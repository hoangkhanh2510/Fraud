\documentclass[11pt]{article}

    \usepackage[breakable]{tcolorbox}
    \usepackage{parskip} % Stop auto-indenting (to mimic markdown behaviour)
    
    \usepackage{iftex}
    \ifPDFTeX
    	\usepackage[T1]{fontenc}
    	\usepackage{mathpazo}
    \else
    	\usepackage{fontspec}
    \fi

    % Basic figure setup, for now with no caption control since it's done
    % automatically by Pandoc (which extracts ![](path) syntax from Markdown).
    \usepackage{graphicx}
    % Maintain compatibility with old templates. Remove in nbconvert 6.0
    \let\Oldincludegraphics\includegraphics
    % Ensure that by default, figures have no caption (until we provide a
    % proper Figure object with a Caption API and a way to capture that
    % in the conversion process - todo).
    \usepackage{caption}
    \DeclareCaptionFormat{nocaption}{}
    \captionsetup{format=nocaption,aboveskip=0pt,belowskip=0pt}

    \usepackage[Export]{adjustbox} % Used to constrain images to a maximum size
    \adjustboxset{max size={0.9\linewidth}{0.9\paperheight}}
    \usepackage{float}
    \floatplacement{figure}{H} % forces figures to be placed at the correct location
    \usepackage{xcolor} % Allow colors to be defined
    \usepackage{enumerate} % Needed for markdown enumerations to work
    \usepackage{geometry} % Used to adjust the document margins
    \usepackage{amsmath} % Equations
    \usepackage{amssymb} % Equations
    \usepackage{textcomp} % defines textquotesingle
    % Hack from http://tex.stackexchange.com/a/47451/13684:
    \AtBeginDocument{%
        \def\PYZsq{\textquotesingle}% Upright quotes in Pygmentized code
    }
    \usepackage{upquote} % Upright quotes for verbatim code
    \usepackage{eurosym} % defines \euro
    \usepackage[mathletters]{ucs} % Extended unicode (utf-8) support
    \usepackage{fancyvrb} % verbatim replacement that allows latex
    \usepackage{grffile} % extends the file name processing of package graphics 
                         % to support a larger range
    \makeatletter % fix for grffile with XeLaTeX
    \def\Gread@@xetex#1{%
      \IfFileExists{"\Gin@base".bb}%
      {\Gread@eps{\Gin@base.bb}}%
      {\Gread@@xetex@aux#1}%
    }
    \makeatother

    % The hyperref package gives us a pdf with properly built
    % internal navigation ('pdf bookmarks' for the table of contents,
    % internal cross-reference links, web links for URLs, etc.)
    \usepackage{hyperref}
    % The default LaTeX title has an obnoxious amount of whitespace. By default,
    % titling removes some of it. It also provides customization options.
    \usepackage{titling}
    \usepackage{longtable} % longtable support required by pandoc >1.10
    \usepackage{booktabs}  % table support for pandoc > 1.12.2
    \usepackage[inline]{enumitem} % IRkernel/repr support (it uses the enumerate* environment)
    \usepackage[normalem]{ulem} % ulem is needed to support strikethroughs (\sout)
                                % normalem makes italics be italics, not underlines
    \usepackage{mathrsfs}
    

    
    % Colors for the hyperref package
    \definecolor{urlcolor}{rgb}{0,.145,.698}
    \definecolor{linkcolor}{rgb}{.71,0.21,0.01}
    \definecolor{citecolor}{rgb}{.12,.54,.11}

    % ANSI colors
    \definecolor{ansi-black}{HTML}{3E424D}
    \definecolor{ansi-black-intense}{HTML}{282C36}
    \definecolor{ansi-red}{HTML}{E75C58}
    \definecolor{ansi-red-intense}{HTML}{B22B31}
    \definecolor{ansi-green}{HTML}{00A250}
    \definecolor{ansi-green-intense}{HTML}{007427}
    \definecolor{ansi-yellow}{HTML}{DDB62B}
    \definecolor{ansi-yellow-intense}{HTML}{B27D12}
    \definecolor{ansi-blue}{HTML}{208FFB}
    \definecolor{ansi-blue-intense}{HTML}{0065CA}
    \definecolor{ansi-magenta}{HTML}{D160C4}
    \definecolor{ansi-magenta-intense}{HTML}{A03196}
    \definecolor{ansi-cyan}{HTML}{60C6C8}
    \definecolor{ansi-cyan-intense}{HTML}{258F8F}
    \definecolor{ansi-white}{HTML}{C5C1B4}
    \definecolor{ansi-white-intense}{HTML}{A1A6B2}
    \definecolor{ansi-default-inverse-fg}{HTML}{FFFFFF}
    \definecolor{ansi-default-inverse-bg}{HTML}{000000}

    % commands and environments needed by pandoc snippets
    % extracted from the output of `pandoc -s`
    \providecommand{\tightlist}{%
      \setlength{\itemsep}{0pt}\setlength{\parskip}{0pt}}
    \DefineVerbatimEnvironment{Highlighting}{Verbatim}{commandchars=\\\{\}}
    % Add ',fontsize=\small' for more characters per line
    \newenvironment{Shaded}{}{}
    \newcommand{\KeywordTok}[1]{\textcolor[rgb]{0.00,0.44,0.13}{\textbf{{#1}}}}
    \newcommand{\DataTypeTok}[1]{\textcolor[rgb]{0.56,0.13,0.00}{{#1}}}
    \newcommand{\DecValTok}[1]{\textcolor[rgb]{0.25,0.63,0.44}{{#1}}}
    \newcommand{\BaseNTok}[1]{\textcolor[rgb]{0.25,0.63,0.44}{{#1}}}
    \newcommand{\FloatTok}[1]{\textcolor[rgb]{0.25,0.63,0.44}{{#1}}}
    \newcommand{\CharTok}[1]{\textcolor[rgb]{0.25,0.44,0.63}{{#1}}}
    \newcommand{\StringTok}[1]{\textcolor[rgb]{0.25,0.44,0.63}{{#1}}}
    \newcommand{\CommentTok}[1]{\textcolor[rgb]{0.38,0.63,0.69}{\textit{{#1}}}}
    \newcommand{\OtherTok}[1]{\textcolor[rgb]{0.00,0.44,0.13}{{#1}}}
    \newcommand{\AlertTok}[1]{\textcolor[rgb]{1.00,0.00,0.00}{\textbf{{#1}}}}
    \newcommand{\FunctionTok}[1]{\textcolor[rgb]{0.02,0.16,0.49}{{#1}}}
    \newcommand{\RegionMarkerTok}[1]{{#1}}
    \newcommand{\ErrorTok}[1]{\textcolor[rgb]{1.00,0.00,0.00}{\textbf{{#1}}}}
    \newcommand{\NormalTok}[1]{{#1}}
    
    % Additional commands for more recent versions of Pandoc
    \newcommand{\ConstantTok}[1]{\textcolor[rgb]{0.53,0.00,0.00}{{#1}}}
    \newcommand{\SpecialCharTok}[1]{\textcolor[rgb]{0.25,0.44,0.63}{{#1}}}
    \newcommand{\VerbatimStringTok}[1]{\textcolor[rgb]{0.25,0.44,0.63}{{#1}}}
    \newcommand{\SpecialStringTok}[1]{\textcolor[rgb]{0.73,0.40,0.53}{{#1}}}
    \newcommand{\ImportTok}[1]{{#1}}
    \newcommand{\DocumentationTok}[1]{\textcolor[rgb]{0.73,0.13,0.13}{\textit{{#1}}}}
    \newcommand{\AnnotationTok}[1]{\textcolor[rgb]{0.38,0.63,0.69}{\textbf{\textit{{#1}}}}}
    \newcommand{\CommentVarTok}[1]{\textcolor[rgb]{0.38,0.63,0.69}{\textbf{\textit{{#1}}}}}
    \newcommand{\VariableTok}[1]{\textcolor[rgb]{0.10,0.09,0.49}{{#1}}}
    \newcommand{\ControlFlowTok}[1]{\textcolor[rgb]{0.00,0.44,0.13}{\textbf{{#1}}}}
    \newcommand{\OperatorTok}[1]{\textcolor[rgb]{0.40,0.40,0.40}{{#1}}}
    \newcommand{\BuiltInTok}[1]{{#1}}
    \newcommand{\ExtensionTok}[1]{{#1}}
    \newcommand{\PreprocessorTok}[1]{\textcolor[rgb]{0.74,0.48,0.00}{{#1}}}
    \newcommand{\AttributeTok}[1]{\textcolor[rgb]{0.49,0.56,0.16}{{#1}}}
    \newcommand{\InformationTok}[1]{\textcolor[rgb]{0.38,0.63,0.69}{\textbf{\textit{{#1}}}}}
    \newcommand{\WarningTok}[1]{\textcolor[rgb]{0.38,0.63,0.69}{\textbf{\textit{{#1}}}}}
    
    
    % Define a nice break command that doesn't care if a line doesn't already
    % exist.
    \def\br{\hspace*{\fill} \\* }
    % Math Jax compatibility definitions
    \def\gt{>}
    \def\lt{<}
    \let\Oldtex\TeX
    \let\Oldlatex\LaTeX
    \renewcommand{\TeX}{\textrm{\Oldtex}}
    \renewcommand{\LaTeX}{\textrm{\Oldlatex}}
    % Document parameters
    % Document title
    \title{BIG DATA}
    
    
    
    
    
% Pygments definitions
\makeatletter
\def\PY@reset{\let\PY@it=\relax \let\PY@bf=\relax%
    \let\PY@ul=\relax \let\PY@tc=\relax%
    \let\PY@bc=\relax \let\PY@ff=\relax}
\def\PY@tok#1{\csname PY@tok@#1\endcsname}
\def\PY@toks#1+{\ifx\relax#1\empty\else%
    \PY@tok{#1}\expandafter\PY@toks\fi}
\def\PY@do#1{\PY@bc{\PY@tc{\PY@ul{%
    \PY@it{\PY@bf{\PY@ff{#1}}}}}}}
\def\PY#1#2{\PY@reset\PY@toks#1+\relax+\PY@do{#2}}

\expandafter\def\csname PY@tok@w\endcsname{\def\PY@tc##1{\textcolor[rgb]{0.73,0.73,0.73}{##1}}}
\expandafter\def\csname PY@tok@c\endcsname{\let\PY@it=\textit\def\PY@tc##1{\textcolor[rgb]{0.25,0.50,0.50}{##1}}}
\expandafter\def\csname PY@tok@cp\endcsname{\def\PY@tc##1{\textcolor[rgb]{0.74,0.48,0.00}{##1}}}
\expandafter\def\csname PY@tok@k\endcsname{\let\PY@bf=\textbf\def\PY@tc##1{\textcolor[rgb]{0.00,0.50,0.00}{##1}}}
\expandafter\def\csname PY@tok@kp\endcsname{\def\PY@tc##1{\textcolor[rgb]{0.00,0.50,0.00}{##1}}}
\expandafter\def\csname PY@tok@kt\endcsname{\def\PY@tc##1{\textcolor[rgb]{0.69,0.00,0.25}{##1}}}
\expandafter\def\csname PY@tok@o\endcsname{\def\PY@tc##1{\textcolor[rgb]{0.40,0.40,0.40}{##1}}}
\expandafter\def\csname PY@tok@ow\endcsname{\let\PY@bf=\textbf\def\PY@tc##1{\textcolor[rgb]{0.67,0.13,1.00}{##1}}}
\expandafter\def\csname PY@tok@nb\endcsname{\def\PY@tc##1{\textcolor[rgb]{0.00,0.50,0.00}{##1}}}
\expandafter\def\csname PY@tok@nf\endcsname{\def\PY@tc##1{\textcolor[rgb]{0.00,0.00,1.00}{##1}}}
\expandafter\def\csname PY@tok@nc\endcsname{\let\PY@bf=\textbf\def\PY@tc##1{\textcolor[rgb]{0.00,0.00,1.00}{##1}}}
\expandafter\def\csname PY@tok@nn\endcsname{\let\PY@bf=\textbf\def\PY@tc##1{\textcolor[rgb]{0.00,0.00,1.00}{##1}}}
\expandafter\def\csname PY@tok@ne\endcsname{\let\PY@bf=\textbf\def\PY@tc##1{\textcolor[rgb]{0.82,0.25,0.23}{##1}}}
\expandafter\def\csname PY@tok@nv\endcsname{\def\PY@tc##1{\textcolor[rgb]{0.10,0.09,0.49}{##1}}}
\expandafter\def\csname PY@tok@no\endcsname{\def\PY@tc##1{\textcolor[rgb]{0.53,0.00,0.00}{##1}}}
\expandafter\def\csname PY@tok@nl\endcsname{\def\PY@tc##1{\textcolor[rgb]{0.63,0.63,0.00}{##1}}}
\expandafter\def\csname PY@tok@ni\endcsname{\let\PY@bf=\textbf\def\PY@tc##1{\textcolor[rgb]{0.60,0.60,0.60}{##1}}}
\expandafter\def\csname PY@tok@na\endcsname{\def\PY@tc##1{\textcolor[rgb]{0.49,0.56,0.16}{##1}}}
\expandafter\def\csname PY@tok@nt\endcsname{\let\PY@bf=\textbf\def\PY@tc##1{\textcolor[rgb]{0.00,0.50,0.00}{##1}}}
\expandafter\def\csname PY@tok@nd\endcsname{\def\PY@tc##1{\textcolor[rgb]{0.67,0.13,1.00}{##1}}}
\expandafter\def\csname PY@tok@s\endcsname{\def\PY@tc##1{\textcolor[rgb]{0.73,0.13,0.13}{##1}}}
\expandafter\def\csname PY@tok@sd\endcsname{\let\PY@it=\textit\def\PY@tc##1{\textcolor[rgb]{0.73,0.13,0.13}{##1}}}
\expandafter\def\csname PY@tok@si\endcsname{\let\PY@bf=\textbf\def\PY@tc##1{\textcolor[rgb]{0.73,0.40,0.53}{##1}}}
\expandafter\def\csname PY@tok@se\endcsname{\let\PY@bf=\textbf\def\PY@tc##1{\textcolor[rgb]{0.73,0.40,0.13}{##1}}}
\expandafter\def\csname PY@tok@sr\endcsname{\def\PY@tc##1{\textcolor[rgb]{0.73,0.40,0.53}{##1}}}
\expandafter\def\csname PY@tok@ss\endcsname{\def\PY@tc##1{\textcolor[rgb]{0.10,0.09,0.49}{##1}}}
\expandafter\def\csname PY@tok@sx\endcsname{\def\PY@tc##1{\textcolor[rgb]{0.00,0.50,0.00}{##1}}}
\expandafter\def\csname PY@tok@m\endcsname{\def\PY@tc##1{\textcolor[rgb]{0.40,0.40,0.40}{##1}}}
\expandafter\def\csname PY@tok@gh\endcsname{\let\PY@bf=\textbf\def\PY@tc##1{\textcolor[rgb]{0.00,0.00,0.50}{##1}}}
\expandafter\def\csname PY@tok@gu\endcsname{\let\PY@bf=\textbf\def\PY@tc##1{\textcolor[rgb]{0.50,0.00,0.50}{##1}}}
\expandafter\def\csname PY@tok@gd\endcsname{\def\PY@tc##1{\textcolor[rgb]{0.63,0.00,0.00}{##1}}}
\expandafter\def\csname PY@tok@gi\endcsname{\def\PY@tc##1{\textcolor[rgb]{0.00,0.63,0.00}{##1}}}
\expandafter\def\csname PY@tok@gr\endcsname{\def\PY@tc##1{\textcolor[rgb]{1.00,0.00,0.00}{##1}}}
\expandafter\def\csname PY@tok@ge\endcsname{\let\PY@it=\textit}
\expandafter\def\csname PY@tok@gs\endcsname{\let\PY@bf=\textbf}
\expandafter\def\csname PY@tok@gp\endcsname{\let\PY@bf=\textbf\def\PY@tc##1{\textcolor[rgb]{0.00,0.00,0.50}{##1}}}
\expandafter\def\csname PY@tok@go\endcsname{\def\PY@tc##1{\textcolor[rgb]{0.53,0.53,0.53}{##1}}}
\expandafter\def\csname PY@tok@gt\endcsname{\def\PY@tc##1{\textcolor[rgb]{0.00,0.27,0.87}{##1}}}
\expandafter\def\csname PY@tok@err\endcsname{\def\PY@bc##1{\setlength{\fboxsep}{0pt}\fcolorbox[rgb]{1.00,0.00,0.00}{1,1,1}{\strut ##1}}}
\expandafter\def\csname PY@tok@kc\endcsname{\let\PY@bf=\textbf\def\PY@tc##1{\textcolor[rgb]{0.00,0.50,0.00}{##1}}}
\expandafter\def\csname PY@tok@kd\endcsname{\let\PY@bf=\textbf\def\PY@tc##1{\textcolor[rgb]{0.00,0.50,0.00}{##1}}}
\expandafter\def\csname PY@tok@kn\endcsname{\let\PY@bf=\textbf\def\PY@tc##1{\textcolor[rgb]{0.00,0.50,0.00}{##1}}}
\expandafter\def\csname PY@tok@kr\endcsname{\let\PY@bf=\textbf\def\PY@tc##1{\textcolor[rgb]{0.00,0.50,0.00}{##1}}}
\expandafter\def\csname PY@tok@bp\endcsname{\def\PY@tc##1{\textcolor[rgb]{0.00,0.50,0.00}{##1}}}
\expandafter\def\csname PY@tok@fm\endcsname{\def\PY@tc##1{\textcolor[rgb]{0.00,0.00,1.00}{##1}}}
\expandafter\def\csname PY@tok@vc\endcsname{\def\PY@tc##1{\textcolor[rgb]{0.10,0.09,0.49}{##1}}}
\expandafter\def\csname PY@tok@vg\endcsname{\def\PY@tc##1{\textcolor[rgb]{0.10,0.09,0.49}{##1}}}
\expandafter\def\csname PY@tok@vi\endcsname{\def\PY@tc##1{\textcolor[rgb]{0.10,0.09,0.49}{##1}}}
\expandafter\def\csname PY@tok@vm\endcsname{\def\PY@tc##1{\textcolor[rgb]{0.10,0.09,0.49}{##1}}}
\expandafter\def\csname PY@tok@sa\endcsname{\def\PY@tc##1{\textcolor[rgb]{0.73,0.13,0.13}{##1}}}
\expandafter\def\csname PY@tok@sb\endcsname{\def\PY@tc##1{\textcolor[rgb]{0.73,0.13,0.13}{##1}}}
\expandafter\def\csname PY@tok@sc\endcsname{\def\PY@tc##1{\textcolor[rgb]{0.73,0.13,0.13}{##1}}}
\expandafter\def\csname PY@tok@dl\endcsname{\def\PY@tc##1{\textcolor[rgb]{0.73,0.13,0.13}{##1}}}
\expandafter\def\csname PY@tok@s2\endcsname{\def\PY@tc##1{\textcolor[rgb]{0.73,0.13,0.13}{##1}}}
\expandafter\def\csname PY@tok@sh\endcsname{\def\PY@tc##1{\textcolor[rgb]{0.73,0.13,0.13}{##1}}}
\expandafter\def\csname PY@tok@s1\endcsname{\def\PY@tc##1{\textcolor[rgb]{0.73,0.13,0.13}{##1}}}
\expandafter\def\csname PY@tok@mb\endcsname{\def\PY@tc##1{\textcolor[rgb]{0.40,0.40,0.40}{##1}}}
\expandafter\def\csname PY@tok@mf\endcsname{\def\PY@tc##1{\textcolor[rgb]{0.40,0.40,0.40}{##1}}}
\expandafter\def\csname PY@tok@mh\endcsname{\def\PY@tc##1{\textcolor[rgb]{0.40,0.40,0.40}{##1}}}
\expandafter\def\csname PY@tok@mi\endcsname{\def\PY@tc##1{\textcolor[rgb]{0.40,0.40,0.40}{##1}}}
\expandafter\def\csname PY@tok@il\endcsname{\def\PY@tc##1{\textcolor[rgb]{0.40,0.40,0.40}{##1}}}
\expandafter\def\csname PY@tok@mo\endcsname{\def\PY@tc##1{\textcolor[rgb]{0.40,0.40,0.40}{##1}}}
\expandafter\def\csname PY@tok@ch\endcsname{\let\PY@it=\textit\def\PY@tc##1{\textcolor[rgb]{0.25,0.50,0.50}{##1}}}
\expandafter\def\csname PY@tok@cm\endcsname{\let\PY@it=\textit\def\PY@tc##1{\textcolor[rgb]{0.25,0.50,0.50}{##1}}}
\expandafter\def\csname PY@tok@cpf\endcsname{\let\PY@it=\textit\def\PY@tc##1{\textcolor[rgb]{0.25,0.50,0.50}{##1}}}
\expandafter\def\csname PY@tok@c1\endcsname{\let\PY@it=\textit\def\PY@tc##1{\textcolor[rgb]{0.25,0.50,0.50}{##1}}}
\expandafter\def\csname PY@tok@cs\endcsname{\let\PY@it=\textit\def\PY@tc##1{\textcolor[rgb]{0.25,0.50,0.50}{##1}}}

\def\PYZbs{\char`\\}
\def\PYZus{\char`\_}
\def\PYZob{\char`\{}
\def\PYZcb{\char`\}}
\def\PYZca{\char`\^}
\def\PYZam{\char`\&}
\def\PYZlt{\char`\<}
\def\PYZgt{\char`\>}
\def\PYZsh{\char`\#}
\def\PYZpc{\char`\%}
\def\PYZdl{\char`\$}
\def\PYZhy{\char`\-}
\def\PYZsq{\char`\'}
\def\PYZdq{\char`\"}
\def\PYZti{\char`\~}
% for compatibility with earlier versions
\def\PYZat{@}
\def\PYZlb{[}
\def\PYZrb{]}
\makeatother


    % For linebreaks inside Verbatim environment from package fancyvrb. 
    \makeatletter
        \newbox\Wrappedcontinuationbox 
        \newbox\Wrappedvisiblespacebox 
        \newcommand*\Wrappedvisiblespace {\textcolor{red}{\textvisiblespace}} 
        \newcommand*\Wrappedcontinuationsymbol {\textcolor{red}{\llap{\tiny$\m@th\hookrightarrow$}}} 
        \newcommand*\Wrappedcontinuationindent {3ex } 
        \newcommand*\Wrappedafterbreak {\kern\Wrappedcontinuationindent\copy\Wrappedcontinuationbox} 
        % Take advantage of the already applied Pygments mark-up to insert 
        % potential linebreaks for TeX processing. 
        %        {, <, #, %, $, ' and ": go to next line. 
        %        _, }, ^, &, >, - and ~: stay at end of broken line. 
        % Use of \textquotesingle for straight quote. 
        \newcommand*\Wrappedbreaksatspecials {% 
            \def\PYGZus{\discretionary{\char`\_}{\Wrappedafterbreak}{\char`\_}}% 
            \def\PYGZob{\discretionary{}{\Wrappedafterbreak\char`\{}{\char`\{}}% 
            \def\PYGZcb{\discretionary{\char`\}}{\Wrappedafterbreak}{\char`\}}}% 
            \def\PYGZca{\discretionary{\char`\^}{\Wrappedafterbreak}{\char`\^}}% 
            \def\PYGZam{\discretionary{\char`\&}{\Wrappedafterbreak}{\char`\&}}% 
            \def\PYGZlt{\discretionary{}{\Wrappedafterbreak\char`\<}{\char`\<}}% 
            \def\PYGZgt{\discretionary{\char`\>}{\Wrappedafterbreak}{\char`\>}}% 
            \def\PYGZsh{\discretionary{}{\Wrappedafterbreak\char`\#}{\char`\#}}% 
            \def\PYGZpc{\discretionary{}{\Wrappedafterbreak\char`\%}{\char`\%}}% 
            \def\PYGZdl{\discretionary{}{\Wrappedafterbreak\char`\$}{\char`\$}}% 
            \def\PYGZhy{\discretionary{\char`\-}{\Wrappedafterbreak}{\char`\-}}% 
            \def\PYGZsq{\discretionary{}{\Wrappedafterbreak\textquotesingle}{\textquotesingle}}% 
            \def\PYGZdq{\discretionary{}{\Wrappedafterbreak\char`\"}{\char`\"}}% 
            \def\PYGZti{\discretionary{\char`\~}{\Wrappedafterbreak}{\char`\~}}% 
        } 
        % Some characters . , ; ? ! / are not pygmentized. 
        % This macro makes them "active" and they will insert potential linebreaks 
        \newcommand*\Wrappedbreaksatpunct {% 
            \lccode`\~`\.\lowercase{\def~}{\discretionary{\hbox{\char`\.}}{\Wrappedafterbreak}{\hbox{\char`\.}}}% 
            \lccode`\~`\,\lowercase{\def~}{\discretionary{\hbox{\char`\,}}{\Wrappedafterbreak}{\hbox{\char`\,}}}% 
            \lccode`\~`\;\lowercase{\def~}{\discretionary{\hbox{\char`\;}}{\Wrappedafterbreak}{\hbox{\char`\;}}}% 
            \lccode`\~`\:\lowercase{\def~}{\discretionary{\hbox{\char`\:}}{\Wrappedafterbreak}{\hbox{\char`\:}}}% 
            \lccode`\~`\?\lowercase{\def~}{\discretionary{\hbox{\char`\?}}{\Wrappedafterbreak}{\hbox{\char`\?}}}% 
            \lccode`\~`\!\lowercase{\def~}{\discretionary{\hbox{\char`\!}}{\Wrappedafterbreak}{\hbox{\char`\!}}}% 
            \lccode`\~`\/\lowercase{\def~}{\discretionary{\hbox{\char`\/}}{\Wrappedafterbreak}{\hbox{\char`\/}}}% 
            \catcode`\.\active
            \catcode`\,\active 
            \catcode`\;\active
            \catcode`\:\active
            \catcode`\?\active
            \catcode`\!\active
            \catcode`\/\active 
            \lccode`\~`\~ 	
        }
    \makeatother

    \let\OriginalVerbatim=\Verbatim
    \makeatletter
    \renewcommand{\Verbatim}[1][1]{%
        %\parskip\z@skip
        \sbox\Wrappedcontinuationbox {\Wrappedcontinuationsymbol}%
        \sbox\Wrappedvisiblespacebox {\FV@SetupFont\Wrappedvisiblespace}%
        \def\FancyVerbFormatLine ##1{\hsize\linewidth
            \vtop{\raggedright\hyphenpenalty\z@\exhyphenpenalty\z@
                \doublehyphendemerits\z@\finalhyphendemerits\z@
                \strut ##1\strut}%
        }%
        % If the linebreak is at a space, the latter will be displayed as visible
        % space at end of first line, and a continuation symbol starts next line.
        % Stretch/shrink are however usually zero for typewriter font.
        \def\FV@Space {%
            \nobreak\hskip\z@ plus\fontdimen3\font minus\fontdimen4\font
            \discretionary{\copy\Wrappedvisiblespacebox}{\Wrappedafterbreak}
            {\kern\fontdimen2\font}%
        }%
        
        % Allow breaks at special characters using \PYG... macros.
        \Wrappedbreaksatspecials
        % Breaks at punctuation characters . , ; ? ! and / need catcode=\active 	
        \OriginalVerbatim[#1,codes*=\Wrappedbreaksatpunct]%
    }
    \makeatother

    % Exact colors from NB
    \definecolor{incolor}{HTML}{303F9F}
    \definecolor{outcolor}{HTML}{D84315}
    \definecolor{cellborder}{HTML}{CFCFCF}
    \definecolor{cellbackground}{HTML}{F7F7F7}
    
    % prompt
    \makeatletter
    \newcommand{\boxspacing}{\kern\kvtcb@left@rule\kern\kvtcb@boxsep}
    \makeatother
    \newcommand{\prompt}[4]{
        \ttfamily\llap{{\color{#2}[#3]:\hspace{3pt}#4}}\vspace{-\baselineskip}
    }
    

    
    % Prevent overflowing lines due to hard-to-break entities
    \sloppy 
    % Setup hyperref package
    \hypersetup{
      breaklinks=true,  % so long urls are correctly broken across lines
      colorlinks=true,
      urlcolor=urlcolor,
      linkcolor=linkcolor,
      citecolor=citecolor,
      }
    % Slightly bigger margins than the latex defaults
    
    \geometry{verbose,tmargin=1in,bmargin=1in,lmargin=1in,rmargin=1in}
    
    

\begin{document}
    
    \maketitle
    
    

    
    \hypertarget{duxe9tection-de-fraude-bancaire}{%
\section{Détection de fraude
bancaire}\label{duxe9tection-de-fraude-bancaire}}

\hypertarget{big-data-mining-m2-sise-pauline-lainuxe9---khuxe1nh-houxe0ng-luxea}{%
\subsubsection{Big Data Mining M2 SISE : Pauline Lainé - Khánh Hoàng
Lê}\label{big-data-mining-m2-sise-pauline-lainuxe9---khuxe1nh-houxe0ng-luxea}}

    \hypertarget{introduction}{%
\subsection{1. Introduction}\label{introduction}}

    Lorsqu'un client paye par chèque, il se peut que celui-ci ne paye jamais
son panier, nous sommes alors dans le cas d'une fraude. Deux raisons
peuvent être dû à une fraude.

Le premier cas, c'est lorsque nous sommes dans le cas d'un chèque dit
impayé, ce cas de figure apparait lorsque la personne ne possède pas le
solde sur son compte en banque. L'autre cas de figure c'est lorsque nous
avons à faire à un faux chèque. Un faux chèque peut se caractériser par
une fausse identité ou lorsque le CMC7 est incorrect. Le CMC7 est une
série de Caractères Magnétiques Codés à 7 bâtonnets situé en bas d'un
chèque.

Sur une période de 10 mois, ces fraudes se compte au nombre de 20 000
soit 0.6\% des transactions réalisées, réprésentant un chiffre
d'affaires de 2 millions d'euros qui équivaut à 1.1\% du chiffre
d'affaires total de l'enseigne. C'est un chiffre d'affaires perdu non
négligable.

Notre étude va alors se porter sur la détection des fraudes par chèque
dans un contexte déséquilibré.

    \hypertarget{les-donnuxe9es}{%
\subsection{2. Les données}\label{les-donnuxe9es}}

    Les données que nous allons utiliser provienne d'une enseigne de grande
distribustion et d'organismes bancaires tels que la FNCI et la Banque de
France. Nous avons une ligne par transaction sur la période du
21/03/2016 au 19/10/2016 soit une période de 7 mois. Chaque transaction
est décrite par les 23 variables suivantes :

\begin{itemize}
\tightlist
\item
  ZIBZIN : identifiant bancaire relatif à la personne
\item
  IDAvisAutorisAtionCheque : identifiant de la transaction en cours
\item
  Montant : montant de la transaction
\item
  DateTransaction : date de la transaction
\item
  CodeDecision : il s'agit d'une variable qui peut prendre ici 4 valeurs

  \begin{itemize}
  \tightlist
  \item
    0 : la transaction a été acceptée par le magasin
  \item
    1 : la transaction et donc le client fait partie d'une liste blanche
    (bons payeurs). Aucun dans cette base de données
  \item
    2 : le client fait d'une partie d'une liste noire, son historique
    indique c'est un mauvais payeur (des impayés en cours ou des
    incidents bancaires en cours), sa transaction est alors
    automatiquement refusée
  \item
    3 : client ayant était arrêté par le système par le passé pour une
    raison plus ou moins fondée
  \end{itemize}
\item
  VérifianceCPT1 : nombre de transactions effectuées par le même
  identifiant bancaire au cours du même jour
\item
  VérifianceCPT2 : nombre de transactions effectuées par le même
  identifiant bancaire au cours des trois derniers jours
\item
  VérifianceCPT3 : nombre de transactions effectuées par le même
  identifiant bancaire au cours des sept derniers jours
\item
  D2CB : durée de connaissance du client (par son identifiant bancaire),
  en jours. Pour des contraintes légales, cette durée de connaissance ne
  peut excéder deux ans
\item
  ScoringFP1 : score d'anormalité du panier relatif à une première
  famille de produits (ex : denrées alimentaires)
\item
  ScoringFP2 : score d'anormalité du panier relatif à une deuxième
  famille de produits(ex : électroniques)
\item
  ScoringFP3 : score d'anormalité du panier relatif à une troisième
  famille de produits (ex : autres)
\item
  TauxImpNb\_RB : taux impayés enregistrés selon la région où a lieu la
  transaction
\item
  TauxImpNB\_CPM : taux d'impayés relatif au magasin où a lieu la
  transaction
\item
  EcartNumCheq : différence entre les numéros de chèques
\item
  NbrMagasin3J : nombre de magasins différents fréquentés les 3 derniers
  jours
\item
  DiffDateTr1 : écart (en jours) à la précédente transaction
\item
  DiffDateTr2 : écart (en jours) à l'avant dernière transaction
\item
  DiffDateTr3 : écart (en jours) à l'antépénultième transaction
\item
  CA3TRetMtt : montant des dernières transactions + montant de la
  transaction en cours
\item
  CA3TR : montant des trois dernières transactions
\item
  Heure : heure de la transaction
\item
  FlagImpaye : acception (0) ou refus de la transaction (1)
\end{itemize}

Cette dernière variable ``FlagImpaye'' est celle que nous cherchons à
prédire. La classe 0 correspondant aux transactions normales et
acceptées et la classe 1 correspond aux fraudes, les transactions qui
sont refusées.

Voyons à présent de plus près à quoi ressemble nos données.

    \hypertarget{importation-des-librairies}{%
\paragraph{Importation des
librairies}\label{importation-des-librairies}}

    \begin{tcolorbox}[breakable, size=fbox, boxrule=1pt, pad at break*=1mm,colback=cellbackground, colframe=cellborder]
\prompt{In}{incolor}{178}{\boxspacing}
\begin{Verbatim}[commandchars=\\\{\}]
\PY{k+kn}{import} \PY{n+nn}{numpy} \PY{k}{as} \PY{n+nn}{np}
\PY{k+kn}{import} \PY{n+nn}{pandas} \PY{k}{as} \PY{n+nn}{pd}
\PY{k+kn}{import} \PY{n+nn}{missingno} \PY{k}{as} \PY{n+nn}{mn}
\PY{k+kn}{import} \PY{n+nn}{matplotlib}\PY{n+nn}{.}\PY{n+nn}{pyplot} \PY{k}{as} \PY{n+nn}{plt}
\PY{k+kn}{import} \PY{n+nn}{plotly}\PY{n+nn}{.}\PY{n+nn}{express} \PY{k}{as} \PY{n+nn}{px}
\PY{k+kn}{from} \PY{n+nn}{sklearn}\PY{n+nn}{.}\PY{n+nn}{svm} \PY{k+kn}{import} \PY{n}{LinearSVC}
\PY{k+kn}{from} \PY{n+nn}{collections} \PY{k+kn}{import} \PY{n}{Counter}
\PY{k+kn}{from} \PY{n+nn}{sklearn}\PY{n+nn}{.}\PY{n+nn}{datasets} \PY{k+kn}{import} \PY{n}{make\PYZus{}classification}
\PY{k+kn}{from} \PY{n+nn}{sklearn}\PY{n+nn}{.}\PY{n+nn}{ensemble} \PY{k+kn}{import} \PY{n}{RandomForestClassifier}\PY{p}{,} \PY{n}{VotingClassifier}
\PY{k+kn}{from} \PY{n+nn}{sklearn}\PY{n+nn}{.}\PY{n+nn}{tree} \PY{k+kn}{import} \PY{n}{DecisionTreeClassifier}
\PY{k+kn}{from} \PY{n+nn}{sklearn}\PY{n+nn}{.}\PY{n+nn}{model\PYZus{}selection} \PY{k+kn}{import} \PY{n}{train\PYZus{}test\PYZus{}split}\PY{p}{,} \PY{n}{GridSearchCV}\PY{p}{,}\PY{n}{TimeSeriesSplit}\PY{p}{,} \PY{n}{cross\PYZus{}val\PYZus{}score}\PY{p}{,}\PY{n}{cross\PYZus{}val\PYZus{}predict}
\PY{k+kn}{from} \PY{n+nn}{sklearn}\PY{n+nn}{.}\PY{n+nn}{metrics} \PY{k+kn}{import} \PY{n}{recall\PYZus{}score}\PY{p}{,} \PY{n}{f1\PYZus{}score}\PY{p}{,} \PY{n}{confusion\PYZus{}matrix}\PY{p}{,} \PY{n}{roc\PYZus{}curve}\PY{p}{,} \PY{n}{roc\PYZus{}auc\PYZus{}score} 
\PY{k+kn}{from} \PY{n+nn}{imblearn}\PY{n+nn}{.}\PY{n+nn}{over\PYZus{}sampling} \PY{k+kn}{import} \PY{n}{SMOTE}
\PY{k+kn}{from} \PY{n+nn}{sklearn}\PY{n+nn}{.}\PY{n+nn}{pipeline} \PY{k+kn}{import} \PY{n}{make\PYZus{}pipeline}
\PY{k+kn}{from} \PY{n+nn}{sklearn}\PY{n+nn}{.}\PY{n+nn}{ensemble} \PY{k+kn}{import} \PY{n}{IsolationForest}
\PY{k+kn}{from} \PY{n+nn}{sklearn}\PY{n+nn}{.}\PY{n+nn}{preprocessing} \PY{k+kn}{import} \PY{n}{StandardScaler}
\PY{k+kn}{from} \PY{n+nn}{sklearn}\PY{n+nn}{.}\PY{n+nn}{discriminant\PYZus{}analysis} \PY{k+kn}{import} \PY{n}{LinearDiscriminantAnalysis}
\PY{k+kn}{from} \PY{n+nn}{sklearn}\PY{n+nn}{.}\PY{n+nn}{neighbors} \PY{k+kn}{import} \PY{n}{KNeighborsClassifier}
\PY{k+kn}{from} \PY{n+nn}{sklearn} \PY{k+kn}{import} \PY{n}{metrics}
\PY{k+kn}{from} \PY{n+nn}{time} \PY{k+kn}{import} \PY{n}{time}
\PY{k+kn}{import} \PY{n+nn}{seaborn} \PY{k}{as} \PY{n+nn}{sns}
\PY{k+kn}{import} \PY{n+nn}{imblearn}
\PY{k+kn}{from} \PY{n+nn}{sklearn}\PY{n+nn}{.}\PY{n+nn}{decomposition} \PY{k+kn}{import} \PY{n}{PCA}
\PY{k+kn}{from} \PY{n+nn}{sklearn}\PY{n+nn}{.}\PY{n+nn}{svm} \PY{k+kn}{import} \PY{n}{OneClassSVM}
\PY{k+kn}{import} \PY{n+nn}{xgboost} \PY{k}{as} \PY{n+nn}{xgb}
\PY{k+kn}{from} \PY{n+nn}{sklearn}\PY{n+nn}{.}\PY{n+nn}{model\PYZus{}selection} \PY{k+kn}{import} \PY{n}{TimeSeriesSplit}
\PY{k+kn}{from} \PY{n+nn}{sklearn}\PY{n+nn}{.}\PY{n+nn}{model\PYZus{}selection} \PY{k+kn}{import} \PY{n}{cross\PYZus{}val\PYZus{}score}
\PY{k+kn}{from} \PY{n+nn}{matplotlib} \PY{k+kn}{import} \PY{n}{pyplot} \PY{k}{as} \PY{n}{plt}
\PY{k+kn}{from} \PY{n+nn}{collections} \PY{k+kn}{import} \PY{n}{OrderedDict}
\PY{n+nb}{print}\PY{p}{(}\PY{n}{imblearn}\PY{o}{.}\PY{n}{\PYZus{}\PYZus{}version\PYZus{}\PYZus{}}\PY{p}{)}
\end{Verbatim}
\end{tcolorbox}

    \begin{Verbatim}[commandchars=\\\{\}]
0.7.0
    \end{Verbatim}

    \hypertarget{importation-du-jeu-de-donnuxe9es}{%
\subsubsection{2.1. Importation du jeu de
données}\label{importation-du-jeu-de-donnuxe9es}}

    \begin{tcolorbox}[breakable, size=fbox, boxrule=1pt, pad at break*=1mm,colback=cellbackground, colframe=cellborder]
\prompt{In}{incolor}{2}{\boxspacing}
\begin{Verbatim}[commandchars=\\\{\}]
\PY{k+kn}{import} \PY{n+nn}{os}
\PY{c+c1}{\PYZsh{}os.chdir(\PYZdq{}D:/OneDrive/Documents/COURS/M2 SISE/Big Data Mining/Projet\PYZdq{})}
\PY{c+c1}{\PYZsh{}df = pd.read\PYZus{}csv(\PYZdq{}dataproject.csv\PYZdq{},encoding=\PYZdq{}utf\PYZhy{}8\PYZdq{},header=0, sep =\PYZdq{};\PYZdq{},decimal=\PYZsq{},\PYZsq{})}
\PY{n}{df} \PY{o}{=}\PY{n}{pd}\PY{o}{.}\PY{n}{read\PYZus{}csv}\PY{p}{(}\PY{l+s+s2}{\PYZdq{}}\PY{l+s+s2}{/Users/hoangkhanhle/Desktop/School/Master 2/Big Data/dataproject.csv}\PY{l+s+s2}{\PYZdq{}}\PY{p}{,}\PY{n}{encoding}\PY{o}{=}\PY{l+s+s2}{\PYZdq{}}\PY{l+s+s2}{utf\PYZhy{}8}\PY{l+s+s2}{\PYZdq{}}\PY{p}{,}\PY{n}{header}\PY{o}{=}\PY{l+m+mi}{0}\PY{p}{,} \PY{n}{sep} \PY{o}{=}\PY{l+s+s2}{\PYZdq{}}\PY{l+s+s2}{;}\PY{l+s+s2}{\PYZdq{}}\PY{p}{,}\PY{n}{decimal}\PY{o}{=}\PY{l+s+s1}{\PYZsq{}}\PY{l+s+s1}{,}\PY{l+s+s1}{\PYZsq{}}\PY{p}{)}
\end{Verbatim}
\end{tcolorbox}

    \begin{tcolorbox}[breakable, size=fbox, boxrule=1pt, pad at break*=1mm,colback=cellbackground, colframe=cellborder]
\prompt{In}{incolor}{17}{\boxspacing}
\begin{Verbatim}[commandchars=\\\{\}]
\PY{n}{df}\PY{o}{.}\PY{n}{head}\PY{p}{(}\PY{p}{)}
\end{Verbatim}
\end{tcolorbox}

            \begin{tcolorbox}[breakable, size=fbox, boxrule=.5pt, pad at break*=1mm, opacityfill=0]
\prompt{Out}{outcolor}{17}{\boxspacing}
\begin{Verbatim}[commandchars=\\\{\}]
                      ZIBZIN  IDAvisAutorisAtionCheque  MontAnt  \textbackslash{}
0  A034010041908012010710730                  71051532    40.17
1  A035010041908006493331734                  71051533    20.00
2  A013010003908005150136747                  71051534    35.00
3  A013010041908025639221029                  71051536    20.00
4  A013010003908005556100283                  71051538    20.00

       DAteTrAnsAction  CodeDecision  VerifiAnceCPT1  VerifiAnceCPT2  \textbackslash{}
0  2016-03-21 07:47:38             0               1               1
1  2016-03-21 08:04:57             0               0               0
2  2016-03-21 08:06:45             0               0               0
3  2016-03-21 08:11:38             0               0               0
4  2016-03-21 08:17:11             0               0               0

   VerifiAnceCPT3  D2CB  ScoringFP1  {\ldots}  TAuxImpNB\_CPM  EcArtNumCheq  \textbackslash{}
0               1   535         0.0  {\ldots}      21.834061             0
1               0   358         0.0  {\ldots}      12.586532             0
2               0   199         0.0  {\ldots}      39.274924             0
3               0    38         0.0  {\ldots}      39.274924             0
4               0    26         0.0  {\ldots}      39.274924             0

   NbrMAgAsin3J  DiffDAteTr1  DiffDAteTr2  DiffDAteTr3  CA3TRetMtt  CA3TR  \textbackslash{}
0             1          4.0          4.0          4.0       40.17    0.0
1             1          4.0          4.0          4.0       20.00    0.0
2             1          4.0          4.0          4.0       35.00    0.0
3             1          4.0          4.0          4.0       20.00    0.0
4             1          4.0          4.0          4.0       20.00    0.0

   Heure  FlAgImpAye
0  28058           0
1  29097           0
2  29205           0
3  29498           0
4  29831           0

[5 rows x 23 columns]
\end{Verbatim}
\end{tcolorbox}
        
    \begin{tcolorbox}[breakable, size=fbox, boxrule=1pt, pad at break*=1mm,colback=cellbackground, colframe=cellborder]
\prompt{In}{incolor}{18}{\boxspacing}
\begin{Verbatim}[commandchars=\\\{\}]
\PY{n+nb}{print}\PY{p}{(}\PY{n}{np}\PY{o}{.}\PY{n}{shape}\PY{p}{(}\PY{n}{df}\PY{p}{)}\PY{p}{)}
\PY{n+nb}{print}\PY{p}{(}\PY{n}{df}\PY{o}{.}\PY{n}{info}\PY{p}{(}\PY{p}{)}\PY{p}{)}
\end{Verbatim}
\end{tcolorbox}

    \begin{Verbatim}[commandchars=\\\{\}]
(2231369, 23)
<class 'pandas.core.frame.DataFrame'>
RangeIndex: 2231369 entries, 0 to 2231368
Data columns (total 23 columns):
 \#   Column                    Dtype
---  ------                    -----
 0   ZIBZIN                    object
 1   IDAvisAutorisAtionCheque  int64
 2   MontAnt                   float64
 3   DAteTrAnsAction           object
 4   CodeDecision              int64
 5   VerifiAnceCPT1            int64
 6   VerifiAnceCPT2            int64
 7   VerifiAnceCPT3            int64
 8   D2CB                      int64
 9   ScoringFP1                float64
 10  ScoringFP2                float64
 11  ScoringFP3                float64
 12  TAuxImpNb\_RB              float64
 13  TAuxImpNB\_CPM             float64
 14  EcArtNumCheq              int64
 15  NbrMAgAsin3J              int64
 16  DiffDAteTr1               float64
 17  DiffDAteTr2               float64
 18  DiffDAteTr3               float64
 19  CA3TRetMtt                float64
 20  CA3TR                     float64
 21  Heure                     int64
 22  FlAgImpAye                int64
dtypes: float64(11), int64(10), object(2)
memory usage: 391.6+ MB
None
    \end{Verbatim}

    Nous avons une base de données de plus de 2 millions de transactions (2
231 369) caractérisé par 23 variables. La plus part sont quantitatives
de type ``int64'' et ``float64''. Deux variables sont de type
qualitative : l'indentifiant de la personne (ZIBZIN) et la date de la
transaction (DAteTrAnsAction).

    \hypertarget{valeurs-manquantes}{%
\subsubsection{2.2. Valeurs manquantes}\label{valeurs-manquantes}}

Avant de faire des analyses sur nos données. Nous allons dans un premier
vérifier si nous avons des valeurs manquantes dans notre jeu de données.

    \begin{tcolorbox}[breakable, size=fbox, boxrule=1pt, pad at break*=1mm,colback=cellbackground, colframe=cellborder]
\prompt{In}{incolor}{19}{\boxspacing}
\begin{Verbatim}[commandchars=\\\{\}]
\PY{c+c1}{\PYZsh{} Matrice des valeurs nulles}
\PY{n}{mn}\PY{o}{.}\PY{n}{matrix}\PY{p}{(}\PY{n}{df}\PY{p}{)}
\end{Verbatim}
\end{tcolorbox}

            \begin{tcolorbox}[breakable, size=fbox, boxrule=.5pt, pad at break*=1mm, opacityfill=0]
\prompt{Out}{outcolor}{19}{\boxspacing}
\begin{Verbatim}[commandchars=\\\{\}]
<matplotlib.axes.\_subplots.AxesSubplot at 0x1869330fa60>
\end{Verbatim}
\end{tcolorbox}
        
    \begin{center}
    \adjustimage{max size={0.9\linewidth}{0.9\paperheight}}{output_13_1.png}
    \end{center}
    { \hspace*{\fill} \\}
    
    \begin{tcolorbox}[breakable, size=fbox, boxrule=1pt, pad at break*=1mm,colback=cellbackground, colframe=cellborder]
\prompt{In}{incolor}{21}{\boxspacing}
\begin{Verbatim}[commandchars=\\\{\}]
\PY{c+c1}{\PYZsh{}Nombre de valeurs nulles par variable}
\PY{n+nb}{print}\PY{p}{(}\PY{n}{df}\PY{o}{.}\PY{n}{isnull}\PY{p}{(}\PY{p}{)}\PY{o}{.}\PY{n}{sum}\PY{p}{(}\PY{p}{)}\PY{p}{)}
\end{Verbatim}
\end{tcolorbox}

    \begin{Verbatim}[commandchars=\\\{\}]
ZIBZIN                      0
IDAvisAutorisAtionCheque    0
MontAnt                     0
DAteTrAnsAction             0
CodeDecision                0
VerifiAnceCPT1              0
VerifiAnceCPT2              0
VerifiAnceCPT3              0
D2CB                        0
ScoringFP1                  0
ScoringFP2                  0
ScoringFP3                  0
TAuxImpNb\_RB                0
TAuxImpNB\_CPM               0
EcArtNumCheq                0
NbrMAgAsin3J                0
DiffDAteTr1                 0
DiffDAteTr2                 0
DiffDAteTr3                 0
CA3TRetMtt                  0
CA3TR                       0
Heure                       0
FlAgImpAye                  0
dtype: int64
    \end{Verbatim}

    Notre jeu de données comporte aucune valeur nulle. Nous allons
maintenant pouvoir faire quelques analyses sur nos données.

    \hypertarget{analyse-univariuxe9}{%
\subsubsection{2.3. Analyse univarié}\label{analyse-univariuxe9}}

\hypertarget{matrice-de-corruxe9lation}{%
\subparagraph{Matrice de corrélation}\label{matrice-de-corruxe9lation}}

    \begin{tcolorbox}[breakable, size=fbox, boxrule=1pt, pad at break*=1mm,colback=cellbackground, colframe=cellborder]
\prompt{In}{incolor}{22}{\boxspacing}
\begin{Verbatim}[commandchars=\\\{\}]
\PY{k+kn}{from} \PY{n+nn}{matplotlib} \PY{k+kn}{import} \PY{n}{pyplot}
\PY{k+kn}{import} \PY{n+nn}{seaborn} \PY{k}{as} \PY{n+nn}{sns}
\PY{c+c1}{\PYZsh{}Matrice de corrélation}
\PY{n}{sns}\PY{o}{.}\PY{n}{set\PYZus{}style}\PY{p}{(}\PY{l+s+s2}{\PYZdq{}}\PY{l+s+s2}{darkgrid}\PY{l+s+s2}{\PYZdq{}}\PY{p}{)}
\PY{n}{pyplot}\PY{o}{.}\PY{n}{figure}\PY{p}{(}\PY{n}{figsize}\PY{o}{=}\PY{p}{(}\PY{l+m+mi}{20}\PY{p}{,} \PY{l+m+mi}{20}\PY{p}{)}\PY{p}{)}
\PY{n}{sns}\PY{o}{.}\PY{n}{heatmap}\PY{p}{(}\PY{n}{df}\PY{o}{.}\PY{n}{corr}\PY{p}{(}\PY{p}{)}\PY{p}{,} \PY{n}{square}\PY{o}{=}\PY{k+kc}{True}\PY{p}{,} \PY{n}{annot}\PY{o}{=}\PY{k+kc}{True}\PY{p}{)}
\end{Verbatim}
\end{tcolorbox}

            \begin{tcolorbox}[breakable, size=fbox, boxrule=.5pt, pad at break*=1mm, opacityfill=0]
\prompt{Out}{outcolor}{22}{\boxspacing}
\begin{Verbatim}[commandchars=\\\{\}]
<matplotlib.axes.\_subplots.AxesSubplot at 0x186efebc730>
\end{Verbatim}
\end{tcolorbox}
        
    \begin{center}
    \adjustimage{max size={0.9\linewidth}{0.9\paperheight}}{output_17_1.png}
    \end{center}
    { \hspace*{\fill} \\}
    
    Lorsqu'on regarde la matrice de correlation on voit qu'aucune variables
est corrélées à notre variables cible. Le plus gros score qu'on observe
est 0.041 ce qui est très faible.

\hypertarget{description-des-variables}{%
\subparagraph{Description des
variables}\label{description-des-variables}}

    \begin{tcolorbox}[breakable, size=fbox, boxrule=1pt, pad at break*=1mm,colback=cellbackground, colframe=cellborder]
\prompt{In}{incolor}{23}{\boxspacing}
\begin{Verbatim}[commandchars=\\\{\}]
\PY{n}{df}\PY{o}{.}\PY{n}{describe}\PY{p}{(}\PY{p}{)}
\end{Verbatim}
\end{tcolorbox}

            \begin{tcolorbox}[breakable, size=fbox, boxrule=.5pt, pad at break*=1mm, opacityfill=0]
\prompt{Out}{outcolor}{23}{\boxspacing}
\begin{Verbatim}[commandchars=\\\{\}]
       IDAvisAutorisAtionCheque       MontAnt  CodeDecision  VerifiAnceCPT1  \textbackslash{}
count              2.231369e+06  2.231369e+06  2.231369e+06    2.231369e+06
mean               7.356762e+07  5.991771e+01  1.288043e-02    3.991402e-01
std                1.458268e+06  7.980922e+01  1.682062e-01    5.250070e-01
min                7.105153e+07  1.000000e-02  0.000000e+00    0.000000e+00
25\%                7.232987e+07  2.500000e+01  0.000000e+00    0.000000e+00
50\%                7.353719e+07  4.235000e+01  0.000000e+00    0.000000e+00
75\%                7.480167e+07  7.205000e+01  0.000000e+00    1.000000e+00
max                7.619241e+07  1.698534e+04  3.000000e+00    1.500000e+01

       VerifiAnceCPT2  VerifiAnceCPT3          D2CB    ScoringFP1  \textbackslash{}
count    2.231369e+06    2.231369e+06  2.231369e+06  2.231369e+06
mean     4.296752e-01    4.816980e-01  2.477938e+02  3.100542e+00
std      6.056457e-01    7.339127e-01  2.131088e+02  3.047480e+01
min      0.000000e+00    0.000000e+00  1.000000e+00  0.000000e+00
25\%      0.000000e+00    0.000000e+00  1.000000e+00  5.874265e-03
50\%      0.000000e+00    0.000000e+00  2.430000e+02  1.543502e-02
75\%      1.000000e+00    1.000000e+00  4.680000e+02  1.370299e-01
max      4.000000e+01    4.000000e+01  5.510000e+02  1.680000e+04

         ScoringFP2    ScoringFP3  {\ldots}  TAuxImpNB\_CPM  EcArtNumCheq  \textbackslash{}
count  2.231369e+06  2.231369e+06  {\ldots}   2.231369e+06  2.231369e+06
mean   6.882686e-01  1.769416e-01  {\ldots}   3.256122e+01  6.724334e+03
std    9.336748e+00  5.662696e-01  {\ldots}   3.475789e+01  1.742520e+05
min   -4.969986e+01  0.000000e+00  {\ldots}   0.000000e+00  0.000000e+00
25\%   -2.869431e+00  1.179973e-03  {\ldots}   1.390821e+01  0.000000e+00
50\%    0.000000e+00  2.131271e-03  {\ldots}   2.526529e+01  0.000000e+00
75\%    5.289664e+00  1.533859e-02  {\ldots}   3.927492e+01  0.000000e+00
max    4.702998e+01  1.839140e+01  {\ldots}   7.142857e+02  9.993474e+06

       NbrMAgAsin3J   DiffDAteTr1   DiffDAteTr2   DiffDAteTr3    CA3TRetMtt  \textbackslash{}
count  2.231369e+06  2.231369e+06  2.231369e+06  2.231369e+06  2.231369e+06
mean   1.034200e+00  3.947808e+00  4.234163e+00  4.266234e+00  6.683475e+01
std    1.862585e-01  1.078915e+00  5.510923e-01  4.643674e-01  8.929606e+01
min    1.000000e+00  0.000000e+00  4.745370e-04  1.365741e-03  1.000000e-02
25\%    1.000000e+00  4.000000e+00  4.000000e+00  4.000000e+00  2.792000e+01
50\%    1.000000e+00  4.000000e+00  4.000000e+00  4.000000e+00  4.732000e+01
75\%    1.000000e+00  5.000000e+00  5.000000e+00  5.000000e+00  8.130000e+01
max    1.000000e+01  5.000000e+00  5.000000e+00  5.000000e+00  1.698534e+04

              CA3TR         Heure    FlAgImpAye
count  2.231369e+06  2.231369e+06  2.231369e+06
mean   6.917041e+00  5.347306e+04  2.804108e-03
std    3.550385e+01  1.200173e+04  5.287955e-02
min    0.000000e+00  1.336000e+03  0.000000e+00
25\%    0.000000e+00  4.248500e+04  0.000000e+00
50\%    0.000000e+00  5.488200e+04  0.000000e+00
75\%    0.000000e+00  6.390700e+04  0.000000e+00
max    1.060000e+04  8.256400e+04  1.000000e+00

[8 rows x 21 columns]
\end{Verbatim}
\end{tcolorbox}
        
    L'ordre de grandeur des variables varie enormement, il faudra penser à
standardiser les données pour nos prédictions. Regardons le montant de
la transaction, qui est une variable importante pour notre
problématique. On peut voir que le montant moyen d'une transaction est
proche de 60€. Avec la plus petite transaction qui est inférieur à 1€ et
la transaction la plus élevé qui est d'environ 17K€. On voit donc un
très grand écart entre c'est deux valeurs. La dispersion de cette
variable est élevé. Regardons la distribution de cette variable.

    \begin{tcolorbox}[breakable, size=fbox, boxrule=1pt, pad at break*=1mm,colback=cellbackground, colframe=cellborder]
\prompt{In}{incolor}{24}{\boxspacing}
\begin{Verbatim}[commandchars=\\\{\}]
\PY{c+c1}{\PYZsh{}Distribution de la variable Montant}
\PY{n}{plt}\PY{o}{.}\PY{n}{figure}\PY{p}{(}\PY{n}{figsize}\PY{o}{=}\PY{p}{(}\PY{l+m+mi}{16}\PY{p}{,} \PY{l+m+mi}{10}\PY{p}{)}\PY{p}{)}
\PY{n}{ax} \PY{o}{=} \PY{n}{sns}\PY{o}{.}\PY{n}{distplot}\PY{p}{(}\PY{n}{df}\PY{p}{[}\PY{l+s+s2}{\PYZdq{}}\PY{l+s+s2}{MontAnt}\PY{l+s+s2}{\PYZdq{}}\PY{p}{]}\PY{p}{)}
\PY{n}{ax}\PY{o}{.}\PY{n}{set}\PY{p}{(}\PY{n}{ylabel}\PY{o}{=}\PY{l+s+s2}{\PYZdq{}}\PY{l+s+s2}{Distribution}\PY{l+s+s2}{\PYZdq{}}\PY{p}{)}  
\end{Verbatim}
\end{tcolorbox}

            \begin{tcolorbox}[breakable, size=fbox, boxrule=.5pt, pad at break*=1mm, opacityfill=0]
\prompt{Out}{outcolor}{24}{\boxspacing}
\begin{Verbatim}[commandchars=\\\{\}]
[Text(0, 0.5, 'Distribution')]
\end{Verbatim}
\end{tcolorbox}
        
    \begin{center}
    \adjustimage{max size={0.9\linewidth}{0.9\paperheight}}{output_21_1.png}
    \end{center}
    { \hspace*{\fill} \\}
    
    Quand on regarde la distribution du montant des transactions on voit que
la plus part des transactions sont inférieur à 500€. On a très peu de
transactions avec une montant très élevé. Regardons à présent les deux
variables quantitatives discrètes ``FlAgImpAye'' et ``CodeDecision''
afin de voir la répartion de ces variables dans cette base.

\hypertarget{code-duxe9cision}{%
\subparagraph{Code Décision :}\label{code-duxe9cision}}

Cette variable rappelons-le elle permet de definir si la transaction à
été accepter ou refuser soit parce que le client fait partie d'une liste
noir soit parce que le système à déjà arrêtés le clients. Cette variable
peut prendre 4 valeurs possible (0,1,2 ou 3). Sachant qu'il n'y a
normalement aucune valeur 1 dans cette base de données. Nous souhaitons
voir la répartition de chacune des valeurs possibles.

    \begin{tcolorbox}[breakable, size=fbox, boxrule=1pt, pad at break*=1mm,colback=cellbackground, colframe=cellborder]
\prompt{In}{incolor}{25}{\boxspacing}
\begin{Verbatim}[commandchars=\\\{\}]
\PY{c+c1}{\PYZsh{}Répartion de la variable CodeDecision}
\PY{n}{pd}\PY{o}{.}\PY{n}{value\PYZus{}counts}\PY{p}{(}\PY{n}{df}\PY{p}{[}\PY{l+s+s1}{\PYZsq{}}\PY{l+s+s1}{CodeDecision}\PY{l+s+s1}{\PYZsq{}}\PY{p}{]}\PY{p}{)}
\end{Verbatim}
\end{tcolorbox}

            \begin{tcolorbox}[breakable, size=fbox, boxrule=.5pt, pad at break*=1mm, opacityfill=0]
\prompt{Out}{outcolor}{25}{\boxspacing}
\begin{Verbatim}[commandchars=\\\{\}]
0    2218002
2      11360
3       2007
Name: CodeDecision, dtype: int64
\end{Verbatim}
\end{tcolorbox}
        
    \begin{tcolorbox}[breakable, size=fbox, boxrule=1pt, pad at break*=1mm,colback=cellbackground, colframe=cellborder]
\prompt{In}{incolor}{26}{\boxspacing}
\begin{Verbatim}[commandchars=\\\{\}]
\PY{c+c1}{\PYZsh{}Barplot}
\PY{n}{ax} \PY{o}{=} \PY{n}{sns}\PY{o}{.}\PY{n}{countplot}\PY{p}{(}\PY{n}{x}\PY{o}{=}\PY{l+s+s2}{\PYZdq{}}\PY{l+s+s2}{CodeDecision}\PY{l+s+s2}{\PYZdq{}}\PY{p}{,} \PY{n}{data}\PY{o}{=}\PY{n}{df}\PY{p}{)}
\end{Verbatim}
\end{tcolorbox}

    \begin{center}
    \adjustimage{max size={0.9\linewidth}{0.9\paperheight}}{output_24_0.png}
    \end{center}
    { \hspace*{\fill} \\}
    
    On voit donc que en effet la valeur 1 concernant les clients bon payeurs
qui font partie d'une liste blanche ne fait pas partie de cette base. De
plus on peut voir que notre jeu de données est composé en majorité de
transactions qui ont été aceptées par l'enseigne.

\hypertarget{variable-cible-flagimpayes}{%
\subparagraph{Variable cible :
FlAgImpAyes}\label{variable-cible-flagimpayes}}

Comme dit précedemment la variable que nous cherchons à prédire est
``FlAgImpAyes'', il semble donc important de voir la répartition de
cette variable, sachant qu'elle peut prendre que 2 valeurs possibles (0
ou 1).

    \begin{tcolorbox}[breakable, size=fbox, boxrule=1pt, pad at break*=1mm,colback=cellbackground, colframe=cellborder]
\prompt{In}{incolor}{27}{\boxspacing}
\begin{Verbatim}[commandchars=\\\{\}]
\PY{c+c1}{\PYZsh{}Répartion de la variable cible FlAgImpAye en \PYZpc{}}
\PY{n+nb}{print}\PY{p}{(}\PY{n}{pd}\PY{o}{.}\PY{n}{value\PYZus{}counts}\PY{p}{(}\PY{n}{df}\PY{p}{[}\PY{l+s+s1}{\PYZsq{}}\PY{l+s+s1}{FlAgImpAye}\PY{l+s+s1}{\PYZsq{}}\PY{p}{]}\PY{p}{)}\PY{o}{/}\PY{n+nb}{len}\PY{p}{(}\PY{n}{df}\PY{p}{)}\PY{p}{)} 
\end{Verbatim}
\end{tcolorbox}

    \begin{Verbatim}[commandchars=\\\{\}]
0    0.997196
1    0.002804
Name: FlAgImpAye, dtype: float64
    \end{Verbatim}

    \begin{tcolorbox}[breakable, size=fbox, boxrule=1pt, pad at break*=1mm,colback=cellbackground, colframe=cellborder]
\prompt{In}{incolor}{28}{\boxspacing}
\begin{Verbatim}[commandchars=\\\{\}]
\PY{c+c1}{\PYZsh{}Barplot}
\PY{n}{ax} \PY{o}{=} \PY{n}{sns}\PY{o}{.}\PY{n}{countplot}\PY{p}{(}\PY{n}{x}\PY{o}{=}\PY{l+s+s2}{\PYZdq{}}\PY{l+s+s2}{FlAgImpAye}\PY{l+s+s2}{\PYZdq{}}\PY{p}{,} \PY{n}{data}\PY{o}{=}\PY{n}{df}\PY{p}{)}
\end{Verbatim}
\end{tcolorbox}

    \begin{center}
    \adjustimage{max size={0.9\linewidth}{0.9\paperheight}}{output_27_0.png}
    \end{center}
    { \hspace*{\fill} \\}
    
    On voit donc voit que la classe 1 dite ``frauduleuse'' est extremement
petite par rapport à la classe normale (0). Elle représente 0.28\% de
notre jeu de données. Plus de 99\% de nos données sont donc de la classe
normale. Nous sommes dans une environnement extrêmement déséquilibré.
Nos prédictions seront faussées et risque de prédire tout le temps la
classe majoritaire. Nous devons prendre en compte cet élément pour
palier à ça. On va donc procéder à un oversampling pour équilibré nos
classes sans perdre des informations importantes sur les classes. Nous
allons également tester sur les modèles qui le permette une méthode qui
permet de pondérer les données en fonction de sa classe.

On se demande si le montant des transactions est différent en fonction
de la classe. C'est ce que nous allons voir.

\hypertarget{relation-entre-le-type-de-transaction-et-le-montant-de-celle-ci}{%
\subparagraph{Relation entre le type de transaction et le montant de
celle
ci}\label{relation-entre-le-type-de-transaction-et-le-montant-de-celle-ci}}

    \begin{tcolorbox}[breakable, size=fbox, boxrule=1pt, pad at break*=1mm,colback=cellbackground, colframe=cellborder]
\prompt{In}{incolor}{29}{\boxspacing}
\begin{Verbatim}[commandchars=\\\{\}]
\PY{c+c1}{\PYZsh{}Relation entre la classe et le montant}
\PY{n}{sns}\PY{o}{.}\PY{n}{catplot}\PY{p}{(}\PY{n}{x}\PY{o}{=}\PY{l+s+s2}{\PYZdq{}}\PY{l+s+s2}{FlAgImpAye}\PY{l+s+s2}{\PYZdq{}}\PY{p}{,} \PY{n}{y}\PY{o}{=}\PY{l+s+s2}{\PYZdq{}}\PY{l+s+s2}{MontAnt}\PY{l+s+s2}{\PYZdq{}}\PY{p}{,} \PY{n}{kind}\PY{o}{=}\PY{l+s+s2}{\PYZdq{}}\PY{l+s+s2}{box}\PY{l+s+s2}{\PYZdq{}}\PY{p}{,} \PY{n}{data}\PY{o}{=}\PY{n}{df}\PY{p}{)}
\end{Verbatim}
\end{tcolorbox}

            \begin{tcolorbox}[breakable, size=fbox, boxrule=.5pt, pad at break*=1mm, opacityfill=0]
\prompt{Out}{outcolor}{29}{\boxspacing}
\begin{Verbatim}[commandchars=\\\{\}]
<seaborn.axisgrid.FacetGrid at 0x18693868c10>
\end{Verbatim}
\end{tcolorbox}
        
    \begin{center}
    \adjustimage{max size={0.9\linewidth}{0.9\paperheight}}{output_29_1.png}
    \end{center}
    { \hspace*{\fill} \\}
    
    Il ne semble pas y avoir une grande différence entre le montant moyen
des transaction de ces deux classes. Néanmoins on voit que les
transactions les plus élevées sont des transactions qui ne sont pas
frauduleuses. Les transactions de la classe frauduleuses sont moins
dispersées.

\hypertarget{date-transaction}{%
\subparagraph{Date transaction}\label{date-transaction}}

Il nous reste une variable importante à regarder de plus près. C'est la
variable de la date de transaction (DAteTrAnsAction). Nous allons
commencer par la transformer pour quel soit au bon format ``datetime''.

    \begin{tcolorbox}[breakable, size=fbox, boxrule=1pt, pad at break*=1mm,colback=cellbackground, colframe=cellborder]
\prompt{In}{incolor}{3}{\boxspacing}
\begin{Verbatim}[commandchars=\\\{\}]
\PY{c+c1}{\PYZsh{}Transformation de la date au format datetime}
\PY{n}{df}\PY{p}{[}\PY{l+s+s1}{\PYZsq{}}\PY{l+s+s1}{DAteTrAnsAction}\PY{l+s+s1}{\PYZsq{}}\PY{p}{]} \PY{o}{=} \PY{n}{pd}\PY{o}{.}\PY{n}{to\PYZus{}datetime}\PY{p}{(}\PY{n}{df}\PY{p}{[}\PY{l+s+s1}{\PYZsq{}}\PY{l+s+s1}{DAteTrAnsAction}\PY{l+s+s1}{\PYZsq{}}\PY{p}{]}\PY{p}{)}
\PY{n+nb}{print}\PY{p}{(}\PY{n}{df}\PY{o}{.}\PY{n}{info}\PY{p}{(}\PY{p}{)}\PY{p}{)}
\end{Verbatim}
\end{tcolorbox}

    \begin{Verbatim}[commandchars=\\\{\}]
<class 'pandas.core.frame.DataFrame'>
RangeIndex: 2231369 entries, 0 to 2231368
Data columns (total 23 columns):
 \#   Column                    Dtype
---  ------                    -----
 0   ZIBZIN                    object
 1   IDAvisAutorisAtionCheque  int64
 2   MontAnt                   float64
 3   DAteTrAnsAction           datetime64[ns]
 4   CodeDecision              int64
 5   VerifiAnceCPT1            int64
 6   VerifiAnceCPT2            int64
 7   VerifiAnceCPT3            int64
 8   D2CB                      int64
 9   ScoringFP1                float64
 10  ScoringFP2                float64
 11  ScoringFP3                float64
 12  TAuxImpNb\_RB              float64
 13  TAuxImpNB\_CPM             float64
 14  EcArtNumCheq              int64
 15  NbrMAgAsin3J              int64
 16  DiffDAteTr1               float64
 17  DiffDAteTr2               float64
 18  DiffDAteTr3               float64
 19  CA3TRetMtt                float64
 20  CA3TR                     float64
 21  Heure                     int64
 22  FlAgImpAye                int64
dtypes: datetime64[ns](1), float64(11), int64(10), object(1)
memory usage: 391.6+ MB
None
    \end{Verbatim}

    \begin{tcolorbox}[breakable, size=fbox, boxrule=1pt, pad at break*=1mm,colback=cellbackground, colframe=cellborder]
\prompt{In}{incolor}{4}{\boxspacing}
\begin{Verbatim}[commandchars=\\\{\}]
\PY{c+c1}{\PYZsh{}Extraction du mois}
\PY{n}{df}\PY{p}{[}\PY{l+s+s1}{\PYZsq{}}\PY{l+s+s1}{Month}\PY{l+s+s1}{\PYZsq{}}\PY{p}{]}\PY{o}{=}\PY{n}{df}\PY{p}{[}\PY{l+s+s1}{\PYZsq{}}\PY{l+s+s1}{DAteTrAnsAction}\PY{l+s+s1}{\PYZsq{}}\PY{p}{]}\PY{o}{.}\PY{n}{dt}\PY{o}{.}\PY{n}{month}
\PY{n}{a}\PY{o}{=}\PY{n}{pd}\PY{o}{.}\PY{n}{value\PYZus{}counts}\PY{p}{(}\PY{n}{df}\PY{p}{[}\PY{l+s+s1}{\PYZsq{}}\PY{l+s+s1}{Month}\PY{l+s+s1}{\PYZsq{}}\PY{p}{]}\PY{p}{)}
\PY{c+c1}{\PYZsh{}Les différents mois present dans la base de données}
\PY{n+nb}{print}\PY{p}{(}\PY{n}{df}\PY{p}{[}\PY{l+s+s1}{\PYZsq{}}\PY{l+s+s1}{Month}\PY{l+s+s1}{\PYZsq{}}\PY{p}{]}\PY{o}{.}\PY{n}{unique}\PY{p}{(}\PY{p}{)}\PY{p}{)}
\end{Verbatim}
\end{tcolorbox}

    \begin{Verbatim}[commandchars=\\\{\}]
[ 3  4  5  6  7  8  9 10]
    \end{Verbatim}

    \begin{tcolorbox}[breakable, size=fbox, boxrule=1pt, pad at break*=1mm,colback=cellbackground, colframe=cellborder]
\prompt{In}{incolor}{5}{\boxspacing}
\begin{Verbatim}[commandchars=\\\{\}]
\PY{c+c1}{\PYZsh{}Répartion par mois}
\PY{n}{plot} \PY{o}{=} \PY{n}{a}\PY{o}{.}\PY{n}{plot}\PY{o}{.}\PY{n}{pie}\PY{p}{(}\PY{n}{y}\PY{o}{=}\PY{l+s+s1}{\PYZsq{}}\PY{l+s+s1}{Month}\PY{l+s+s1}{\PYZsq{}}\PY{p}{,} \PY{n}{figsize}\PY{o}{=}\PY{p}{(}\PY{l+m+mi}{5}\PY{p}{,} \PY{l+m+mi}{5}\PY{p}{)}\PY{p}{,}\PY{n}{autopct}\PY{o}{=}\PY{l+s+s1}{\PYZsq{}}\PY{l+s+si}{\PYZpc{}1.1f}\PY{l+s+si}{\PYZpc{}\PYZpc{}}\PY{l+s+s1}{\PYZsq{}}\PY{p}{)}
\end{Verbatim}
\end{tcolorbox}

    \begin{center}
    \adjustimage{max size={0.9\linewidth}{0.9\paperheight}}{output_33_0.png}
    \end{center}
    { \hspace*{\fill} \\}
    
    \begin{tcolorbox}[breakable, size=fbox, boxrule=1pt, pad at break*=1mm,colback=cellbackground, colframe=cellborder]
\prompt{In}{incolor}{8}{\boxspacing}
\begin{Verbatim}[commandchars=\\\{\}]
\PY{n}{df}\PY{o}{.}\PY{n}{info}\PY{p}{(}\PY{p}{)}
\end{Verbatim}
\end{tcolorbox}

    \begin{Verbatim}[commandchars=\\\{\}]
<class 'pandas.core.frame.DataFrame'>
RangeIndex: 2231369 entries, 0 to 2231368
Data columns (total 24 columns):
 \#   Column                    Dtype
---  ------                    -----
 0   ZIBZIN                    object
 1   IDAvisAutorisAtionCheque  int64
 2   MontAnt                   float64
 3   DAteTrAnsAction           datetime64[ns]
 4   CodeDecision              int64
 5   VerifiAnceCPT1            int64
 6   VerifiAnceCPT2            int64
 7   VerifiAnceCPT3            int64
 8   D2CB                      int64
 9   ScoringFP1                float64
 10  ScoringFP2                float64
 11  ScoringFP3                float64
 12  TAuxImpNb\_RB              float64
 13  TAuxImpNB\_CPM             float64
 14  EcArtNumCheq              int64
 15  NbrMAgAsin3J              int64
 16  DiffDAteTr1               float64
 17  DiffDAteTr2               float64
 18  DiffDAteTr3               float64
 19  CA3TRetMtt                float64
 20  CA3TR                     float64
 21  Heure                     int64
 22  FlAgImpAye                int64
 23  Month                     int64
dtypes: datetime64[ns](1), float64(11), int64(11), object(1)
memory usage: 408.6+ MB
    \end{Verbatim}

    Donc notre base de données répresente 8 mois différents de mars à
octobre. Ce graphique nous montre la répartition des transactions en
fonction du mois. Le mois qui enregistre le plus de transactions est le
mois de juillet. A contrario les mois qui ont le moins d'activité est le
mois et mars suivit de près par octobre, ce qui s'explique nottament par
le fait que nous n'avons pas les transactions sur le mois complet.

La première étape d'analyse de la base est fini. Nous allons pouvoir
passer à la création de nos bases de test et d'entrainement.

\hypertarget{cruxe9ation-des-ensembles-de-donnuxe9es}{%
\subsection{3. Création des ensembles de
données}\label{cruxe9ation-des-ensembles-de-donnuxe9es}}

L'objectif ici est de creer notre base d'apprentissage et de test en
fonction de la date de la transaction. Puis à partir de la base
d'apprentissage nous créerons deux sous-ensembles. Un ensemble
d'entrainement et un ensemble de validation.

\hypertarget{ensembles-dapprentissage-et-de-test}{%
\subsubsection{3.1. Ensembles d'apprentissage et de
test}\label{ensembles-dapprentissage-et-de-test}}

\begin{itemize}
\tightlist
\item
  Notre base d'apprentissage comprendra les transactions ayant eu lieu
  entre le ``2016-03-21'' et le ``2016-09-19''.
\item
  Notre base de test les transactions ayant eu lieu entre le
  ``2016-09-20'' et le ``2016-10-19''.
\end{itemize}

    \begin{tcolorbox}[breakable, size=fbox, boxrule=1pt, pad at break*=1mm,colback=cellbackground, colframe=cellborder]
\prompt{In}{incolor}{110}{\boxspacing}
\begin{Verbatim}[commandchars=\\\{\}]
\PY{c+c1}{\PYZsh{}Apprentissage}
\PY{n}{Apprenti} \PY{o}{=} \PY{n}{df}\PY{o}{.}\PY{n}{loc}\PY{p}{[}\PY{n}{df}\PY{p}{[}\PY{l+s+s1}{\PYZsq{}}\PY{l+s+s1}{DAteTrAnsAction}\PY{l+s+s1}{\PYZsq{}}\PY{p}{]} \PY{o}{\PYZlt{}} \PY{l+s+s1}{\PYZsq{}}\PY{l+s+s1}{2016\PYZhy{}09\PYZhy{}20}\PY{l+s+s1}{\PYZsq{}}\PY{p}{]} 
\PY{c+c1}{\PYZsh{}Test}
\PY{n}{Test} \PY{o}{=} \PY{n}{df}\PY{o}{.}\PY{n}{loc}\PY{p}{[}\PY{n}{df}\PY{p}{[}\PY{l+s+s1}{\PYZsq{}}\PY{l+s+s1}{DAteTrAnsAction}\PY{l+s+s1}{\PYZsq{}}\PY{p}{]} \PY{o}{\PYZgt{}}\PY{o}{=} \PY{l+s+s1}{\PYZsq{}}\PY{l+s+s1}{2016\PYZhy{}09\PYZhy{}20}\PY{l+s+s1}{\PYZsq{}}\PY{p}{]} 
\end{Verbatim}
\end{tcolorbox}

    Une fois notre base divisé en deux ensembles. On créé notre ensemble de
variables explicatives et notre variable cible.

Nous allons garder seulement les variables qui nous semble les plus
utiles. Nous avons donc décidés d'enlever les variables d'identifiants,
la variable de date ainsi que l'heure des transactions.

Nous regroupons ces variables dans une base nommée
``training\_features'' pour la base d'apprentissage et
``test\_features'' pour le test. Dans une base ``training\_target'' et
``test\_target'' nous mettons notre variable cible qui représente la
classe de la transaction (FlAgImpAye).

    \begin{tcolorbox}[breakable, size=fbox, boxrule=1pt, pad at break*=1mm,colback=cellbackground, colframe=cellborder]
\prompt{In}{incolor}{33}{\boxspacing}
\begin{Verbatim}[commandchars=\\\{\}]
\PY{c+c1}{\PYZsh{}Apprentissage}
\PY{c+c1}{\PYZsh{}Variables explicatives}
\PY{n}{training\PYZus{}features}\PY{o}{=} \PY{n}{Apprenti}\PY{o}{.}\PY{n}{drop}\PY{p}{(}\PY{n}{Apprenti}\PY{o}{.}\PY{n}{columns}\PY{p}{[}\PY{p}{[}\PY{l+m+mi}{0}\PY{p}{,} \PY{l+m+mi}{1}\PY{p}{,}\PY{l+m+mi}{3}\PY{p}{,}\PY{l+m+mi}{21}\PY{p}{,}\PY{l+m+mi}{22}\PY{p}{,}\PY{l+m+mi}{23}\PY{p}{]}\PY{p}{]}\PY{p}{,} \PY{n}{axis}\PY{o}{=}\PY{l+m+mi}{1}\PY{p}{)} 
\PY{c+c1}{\PYZsh{}Variables cible}
\PY{n}{training\PYZus{}target} \PY{o}{=} \PY{n}{Apprenti}\PY{p}{[}\PY{l+s+s1}{\PYZsq{}}\PY{l+s+s1}{FlAgImpAye}\PY{l+s+s1}{\PYZsq{}}\PY{p}{]}
\end{Verbatim}
\end{tcolorbox}

    \begin{tcolorbox}[breakable, size=fbox, boxrule=1pt, pad at break*=1mm,colback=cellbackground, colframe=cellborder]
\prompt{In}{incolor}{111}{\boxspacing}
\begin{Verbatim}[commandchars=\\\{\}]
\PY{c+c1}{\PYZsh{}Test}
\PY{c+c1}{\PYZsh{}Variables explicatives}
\PY{n}{test\PYZus{}features}\PY{o}{=} \PY{n}{Test}\PY{o}{.}\PY{n}{drop}\PY{p}{(}\PY{n}{Apprenti}\PY{o}{.}\PY{n}{columns}\PY{p}{[}\PY{p}{[}\PY{l+m+mi}{0}\PY{p}{,} \PY{l+m+mi}{1}\PY{p}{,} \PY{l+m+mi}{3}\PY{p}{,}\PY{l+m+mi}{21}\PY{p}{,}\PY{l+m+mi}{22}\PY{p}{]}\PY{p}{]}\PY{p}{,} \PY{n}{axis}\PY{o}{=}\PY{l+m+mi}{1}\PY{p}{)} 
\PY{c+c1}{\PYZsh{}Variable cible}
\PY{n}{test\PYZus{}target} \PY{o}{=} \PY{n}{Test}\PY{p}{[}\PY{l+s+s1}{\PYZsq{}}\PY{l+s+s1}{FlAgImpAye}\PY{l+s+s1}{\PYZsq{}}\PY{p}{]}
\end{Verbatim}
\end{tcolorbox}

    \hypertarget{ensembles-dentrainement-et-de-validation}{%
\subsubsection{3.2. Ensembles d'entrainement et de
validation}\label{ensembles-dentrainement-et-de-validation}}

A partir de notre base d'apprentissage on va pouvoir créer notre
ensemble d'entrainement et notre ensemble de validation. On a dans un
premier temps décidé de répartir les données en 70\% pour l'entrainement
et 30\% pour le test. La répartition était donc la suivante : -
Transaction du 2016-03-21 au 2016-07-22 pour l'ensemble d'entrainement.
- Transaction du 2016-07-22 au 2016-09-19 pour la base de validation.
Avec cette répartition nous obtenions au maximun une mesure f1 de 13\%
lors de la phase de validation.

Nous avons cherché à améliorer cette mesure en modifiant notre manière
de diviser la base. Nous sommes donc partis sur une répartition en
fonction de la date. En prenant 1 mois de transactions pour l'ensemble
de validation et le reste pour l'ensemble d'entrainement, soit 5 mois.
Cette répartition permet d'obtenir de meilleurs résultats nous avons
donc choisit cette option et nous obtenons la répartition suivantes : -
Transaction du 2016-03-21 au 2016-08-19 pour l'ensemble d'entrainement.
- Transaction du 2016-08-20 au 2016-09-19 pour la base de validation.

    \begin{tcolorbox}[breakable, size=fbox, boxrule=1pt, pad at break*=1mm,colback=cellbackground, colframe=cellborder]
\prompt{In}{incolor}{103}{\boxspacing}
\begin{Verbatim}[commandchars=\\\{\}]
\PY{c+c1}{\PYZsh{}Entrainement}
\PY{n}{entrainement} \PY{o}{=} \PY{n}{Apprenti}\PY{o}{.}\PY{n}{loc}\PY{p}{[}\PY{n}{df}\PY{p}{[}\PY{l+s+s1}{\PYZsq{}}\PY{l+s+s1}{DAteTrAnsAction}\PY{l+s+s1}{\PYZsq{}}\PY{p}{]} \PY{o}{\PYZlt{}} \PY{l+s+s1}{\PYZsq{}}\PY{l+s+s1}{2016\PYZhy{}08\PYZhy{}20}\PY{l+s+s1}{\PYZsq{}}\PY{p}{]} 
\PY{c+c1}{\PYZsh{}Validation}
\PY{n}{validation} \PY{o}{=} \PY{n}{Apprenti}\PY{o}{.}\PY{n}{loc}\PY{p}{[}\PY{n}{df}\PY{p}{[}\PY{l+s+s1}{\PYZsq{}}\PY{l+s+s1}{DAteTrAnsAction}\PY{l+s+s1}{\PYZsq{}}\PY{p}{]} \PY{o}{\PYZgt{}}\PY{o}{=} \PY{l+s+s1}{\PYZsq{}}\PY{l+s+s1}{2016\PYZhy{}08\PYZhy{}20}\PY{l+s+s1}{\PYZsq{}}\PY{p}{]} 
\end{Verbatim}
\end{tcolorbox}

    \begin{tcolorbox}[breakable, size=fbox, boxrule=1pt, pad at break*=1mm,colback=cellbackground, colframe=cellborder]
\prompt{In}{incolor}{104}{\boxspacing}
\begin{Verbatim}[commandchars=\\\{\}]
\PY{n}{x\PYZus{}train}\PY{o}{=} \PY{n}{entrainement}\PY{o}{.}\PY{n}{drop}\PY{p}{(}\PY{n}{Apprenti}\PY{o}{.}\PY{n}{columns}\PY{p}{[}\PY{p}{[}\PY{l+m+mi}{0}\PY{p}{,} \PY{l+m+mi}{1}\PY{p}{,} \PY{l+m+mi}{3}\PY{p}{,}\PY{l+m+mi}{21}\PY{p}{,}\PY{l+m+mi}{22}\PY{p}{]}\PY{p}{]}\PY{p}{,} \PY{n}{axis}\PY{o}{=}\PY{l+m+mi}{1}\PY{p}{)} 
\PY{n}{y\PYZus{}train} \PY{o}{=} \PY{n}{entrainement}\PY{p}{[}\PY{l+s+s1}{\PYZsq{}}\PY{l+s+s1}{FlAgImpAye}\PY{l+s+s1}{\PYZsq{}}\PY{p}{]}
\PY{n}{x\PYZus{}val}\PY{o}{=} \PY{n}{validation}\PY{o}{.}\PY{n}{drop}\PY{p}{(}\PY{n}{Apprenti}\PY{o}{.}\PY{n}{columns}\PY{p}{[}\PY{p}{[}\PY{l+m+mi}{0}\PY{p}{,} \PY{l+m+mi}{1}\PY{p}{,} \PY{l+m+mi}{3}\PY{p}{,}\PY{l+m+mi}{21}\PY{p}{,}\PY{l+m+mi}{22}\PY{p}{]}\PY{p}{]}\PY{p}{,} \PY{n}{axis}\PY{o}{=}\PY{l+m+mi}{1}\PY{p}{)} 
\PY{n}{y\PYZus{}val} \PY{o}{=} \PY{n}{validation}\PY{p}{[}\PY{l+s+s1}{\PYZsq{}}\PY{l+s+s1}{FlAgImpAye}\PY{l+s+s1}{\PYZsq{}}\PY{p}{]}
\end{Verbatim}
\end{tcolorbox}

    \begin{tcolorbox}[breakable, size=fbox, boxrule=1pt, pad at break*=1mm,colback=cellbackground, colframe=cellborder]
\prompt{In}{incolor}{14}{\boxspacing}
\begin{Verbatim}[commandchars=\\\{\}]
\PY{c+c1}{\PYZsh{}information sur train}
\PY{n+nb}{print}\PY{p}{(}\PY{n}{pd}\PY{o}{.}\PY{n}{value\PYZus{}counts}\PY{p}{(}\PY{n}{y\PYZus{}train}\PY{p}{)}\PY{p}{)}
\PY{n+nb}{print}\PY{p}{(}\PY{n}{np}\PY{o}{.}\PY{n}{shape}\PY{p}{(}\PY{n}{x\PYZus{}train}\PY{p}{)}\PY{p}{)}
\end{Verbatim}
\end{tcolorbox}

    \begin{Verbatim}[commandchars=\\\{\}]
0    1673087
1       4454
Name: FlAgImpAye, dtype: int64
(1677541, 19)
    \end{Verbatim}

    \hypertarget{mesure-duxe9valuation-le-f1-score}{%
\subsection{4. Mesure d'évaluation le F1
Score}\label{mesure-duxe9valuation-le-f1-score}}

En classification plusieurs mesures peuvent être utiliser pour evaluer
la qualité d'un modèle. Par exemple la precision, le rappel, la courbe
ROC, l'accuracy ou bien la f-mesure. Toutes ces mesures ne sont pas
forcement adaptées à notre problème.

Par exemple l'accuracy n'est pas une indicateur fiable lorsque nous
somme dans un environnement déséquilibré. En effet avec un jeu de
données où plus de 99\% de nos données appartienne à la première classe
si notre algorithme prédit toutes nos transaction comme appartenant à la
première classe nous aurons une accuracy très élevée mais pourtant le
modèle sera mauvais car il predira toujours la même classe.

La précision permet de calculer la proportion de predictions positives
qui sont réellement positives. Permet ainsi de réduire l'erreur de
prédiction des transactions frauduleuses. Le rappel lui nous donne la
proportion de données positives qui ont été correctement predites. C'est
à dire la part des fraudes qui ont été détecter.

Dans notre problème il est important de detecter le maximum de fraudes
mais il est également important de ne pas prédire des fausses fraudes.
En effet accusé un client de fraudes alors que ce n'est pas la cas peut
avoir des conséquences. Ainsi pour maximiser la prédiction de fraudes et
minimiser l'erreur de prédiction, une mesure existe, c'est la f-mesure.
L'objectif va être de maximiser celle-ci pour obtenir le meilleur
modèle.

D'ailleurs, nous pouvons utiliser les measures suivant pour les bases de
données imbalance:

\begin{itemize}
\tightlist
\item
  Recall
\item
  Precision
\item
  F1-score
\item
  Area Under ROC curve.
\end{itemize}

    Formulaire: - Recall = TP / (TP + FN) - Precision = TP / (TP + FP) -
F1-Score = 2\emph{P}R / (P + R) here P for Precision, R for Recall

    \hypertarget{uxe9chantillonnage}{%
\subsection{5. Échantillonnage}\label{uxe9chantillonnage}}

Afin de résoudre le problème de l'apprentissage dééquilibrés, le
ré-échantiollanage est le méthode le plus utilisé. L'objectif est de
modifier notre base de données avant d'entraîner le modèle prédictif
afin d'avoir des données plus équilibrées.

Il éxiste deux catégories d'échantillonnages : les sous-echantillonnage
(undersampling) qui consiste à supprimer des données de la classe
majoritaire et sur-echantillonnage (oversampling) qui lui va rajouter
des données dans la classe minoritaire.

\hypertarget{smote}{%
\subsubsection{5.1. SMOTE}\label{smote}}

Comme nous l'avons vu la classe majoritaire contient plus de 99\% des
données. Réaliser un undersampling supprimerai alors presque la totalité
des données. Ainsi pour garder le maximum de d'information, nous avons
décidé de réaliser un oversampling avec la méthode SMOTE.

Le principe de SMOTE est de générer de nouveaux échantillons en
combinant les données de la classe minoritaire avec celles de leurs
voisins proches. Techniquement, on peut décomposer SMOTE en 5 étapes :

\begin{itemize}
\item
  Choix d'un vecteur caractéristique de notre classe minoritaire que
  nous appellerons vc ;
\item
  Sélection des k-voisins les plus proches (k=5 par défaut) et choix de
  l'un d'eux au hasard que l'on appellera pv ;
\item
  Calcul de la différence pour chaque valeur caractéristique (feature
  value) i, vc{[}i{]}-pv{[}i{]} et multiplication de celle-ci par un
  nombre aléatoire entre {[}0,1{]};
\item
  Ajout du résultat précédent à la valeur de la caractéristique i du
  vecteur vc afin d'obtenir un nouveau point (une nouvelle donnée) dans
  l'espace des caractéristiques\,;
\item
  Répétition de ces opérations pour chaque point de données de la classe
  minoritaire\,;
\end{itemize}

    \begin{tcolorbox}[breakable, size=fbox, boxrule=1pt, pad at break*=1mm,colback=cellbackground, colframe=cellborder]
\prompt{In}{incolor}{15}{\boxspacing}
\begin{Verbatim}[commandchars=\\\{\}]
\PY{c+c1}{\PYZsh{}Réalisation du SMOTE}
\PY{n}{sm} \PY{o}{=} \PY{n}{SMOTE}\PY{p}{(}\PY{n}{random\PYZus{}state}\PY{o}{=}\PY{l+m+mi}{12}\PY{p}{)}
\PY{n}{x\PYZus{}train\PYZus{}res}\PY{p}{,} \PY{n}{y\PYZus{}train\PYZus{}res} \PY{o}{=} \PY{n}{sm}\PY{o}{.}\PY{n}{fit\PYZus{}sample}\PY{p}{(}\PY{n}{x\PYZus{}train}\PY{p}{,} \PY{n}{y\PYZus{}train}\PY{p}{)}
\end{Verbatim}
\end{tcolorbox}

    \begin{tcolorbox}[breakable, size=fbox, boxrule=1pt, pad at break*=1mm,colback=cellbackground, colframe=cellborder]
\prompt{In}{incolor}{16}{\boxspacing}
\begin{Verbatim}[commandchars=\\\{\}]
\PY{c+c1}{\PYZsh{}Vérification}
\PY{n+nb}{print}\PY{p}{(}\PY{n}{np}\PY{o}{.}\PY{n}{shape}\PY{p}{(}\PY{n}{x\PYZus{}train\PYZus{}res}\PY{p}{)}\PY{p}{)}
\PY{n+nb}{print}\PY{p}{(}\PY{n}{pd}\PY{o}{.}\PY{n}{value\PYZus{}counts}\PY{p}{(}\PY{n}{y\PYZus{}train\PYZus{}res}\PY{p}{)}\PY{o}{/}\PY{n+nb}{len}\PY{p}{(}\PY{n}{y\PYZus{}train\PYZus{}res}\PY{p}{)}\PY{p}{)}
\end{Verbatim}
\end{tcolorbox}

    \begin{Verbatim}[commandchars=\\\{\}]
(3346174, 19)
1    0.5
0    0.5
Name: FlAgImpAye, dtype: float64
    \end{Verbatim}

    L'algorithme smote est exécuter sur l'algorithme d'entrainement. Notre
variable cible est à présent équilibré, nous avons bien 50/50 dans
chaque classe et plus de 3 millions de données.

    \hypertarget{adasyn}{%
\subsubsection{5.1. ADASYN}\label{adasyn}}

    A part de Randon Sampling, nous avons aussi ADASYN, la meilleur version
de SMOTE. La principale différence entre SMOTE et ADASYN est la
différence dans la génération de points d'échantillonnage synthétiques
pour les points de données minoritaires. Dans ADASYN, on considère une
distribution de densité rₓ qui décide ainsi du nombre d'échantillons
synthétiques à générer pour un point particulier, alors que dans SMOTE,
il y a un poids uniforme pour tous les points minoritaires. A cause de
niveau de temps d'éxecution donc nous n'avons pas avancé ce partie

    \hypertarget{analyse-composant-principal}{%
\section{6. Analyse Composant
Principal}\label{analyse-composant-principal}}

    L'ACP nous permet de représenter les données en fonction des variables
choisies, grâce à une projection dans un espace de deux dimensions. Les
deux premiers axes de cette représentation sont composés de deux
premières composantes principales de l'ACP, (ce sont Dim 1 et Dim 2). La
troisième coordonnée de cette représentation est la valeur de la
variable cible (nous utilisons la valeur brute normalisé de cette
variable).

    Sans SMOTE

    \begin{tcolorbox}[breakable, size=fbox, boxrule=1pt, pad at break*=1mm,colback=cellbackground, colframe=cellborder]
\prompt{In}{incolor}{43}{\boxspacing}
\begin{Verbatim}[commandchars=\\\{\}]
\PY{n+nb}{print}\PY{p}{(}\PY{n}{pd}\PY{o}{.}\PY{n}{value\PYZus{}counts}\PY{p}{(}\PY{n}{y\PYZus{}train}\PY{p}{)}\PY{o}{/}\PY{n+nb}{len}\PY{p}{(}\PY{n}{y\PYZus{}train}\PY{p}{)}\PY{p}{)}
\end{Verbatim}
\end{tcolorbox}

    \begin{Verbatim}[commandchars=\\\{\}]
0    0.997345
1    0.002655
Name: FlAgImpAye, dtype: float64
    \end{Verbatim}

    \begin{tcolorbox}[breakable, size=fbox, boxrule=1pt, pad at break*=1mm,colback=cellbackground, colframe=cellborder]
\prompt{In}{incolor}{44}{\boxspacing}
\begin{Verbatim}[commandchars=\\\{\}]
\PY{n}{pca} \PY{o}{=} \PY{n}{PCA}\PY{p}{(}\PY{n}{n\PYZus{}components}\PY{o}{=}\PY{l+m+mi}{2}\PY{p}{)}
\PY{n}{sc} \PY{o}{=} \PY{n}{StandardScaler}\PY{p}{(}\PY{p}{)} 
\PY{n}{X\PYZus{}normalized} \PY{o}{=} \PY{n}{sc}\PY{o}{.}\PY{n}{fit\PYZus{}transform}\PY{p}{(}\PY{n}{x\PYZus{}train}\PY{p}{)}  
\PY{n}{components} \PY{o}{=} \PY{n}{pca}\PY{o}{.}\PY{n}{fit\PYZus{}transform}\PY{p}{(}\PY{n}{X\PYZus{}normalized}\PY{p}{)}
\PY{c+c1}{\PYZsh{} Transformation en DataFrame pandas}
\PY{n}{fraud\PYZus{}pca\PYZus{}df} \PY{o}{=} \PY{n}{pd}\PY{o}{.}\PY{n}{DataFrame}\PY{p}{(}\PY{p}{\PYZob{}}
    \PY{l+s+s2}{\PYZdq{}}\PY{l+s+s2}{Dim1}\PY{l+s+s2}{\PYZdq{}} \PY{p}{:} \PY{n}{components}\PY{p}{[}\PY{p}{:}\PY{p}{,}\PY{l+m+mi}{0}\PY{p}{]}\PY{p}{,} 
    \PY{l+s+s2}{\PYZdq{}}\PY{l+s+s2}{Dim2}\PY{l+s+s2}{\PYZdq{}} \PY{p}{:} \PY{n}{components}\PY{p}{[}\PY{p}{:}\PY{p}{,}\PY{l+m+mi}{1}\PY{p}{]}\PY{p}{,} 
    \PY{l+s+s2}{\PYZdq{}}\PY{l+s+s2}{Flag}\PY{l+s+s2}{\PYZdq{}} \PY{p}{:} \PY{n}{y\PYZus{}train}
\PY{p}{\PYZcb{}}\PY{p}{)}
\PY{n}{g\PYZus{}pca} \PY{o}{=} \PY{n}{sns}\PY{o}{.}\PY{n}{lmplot}\PY{p}{(}\PY{l+s+s2}{\PYZdq{}}\PY{l+s+s2}{Dim1}\PY{l+s+s2}{\PYZdq{}}\PY{p}{,} \PY{l+s+s2}{\PYZdq{}}\PY{l+s+s2}{Dim2}\PY{l+s+s2}{\PYZdq{}}\PY{p}{,} \PY{n}{hue} \PY{o}{=} \PY{l+s+s2}{\PYZdq{}}\PY{l+s+s2}{Flag}\PY{l+s+s2}{\PYZdq{}}\PY{p}{,} \PY{n}{data} \PY{o}{=} \PY{n}{fraud\PYZus{}pca\PYZus{}df}\PY{p}{,}\PY{n}{fit\PYZus{}reg} \PY{o}{=} \PY{k+kc}{False}\PY{p}{)}
\PY{n}{g\PYZus{}pca}\PY{o}{.}\PY{n}{set}\PY{p}{(}\PY{n}{xlabel} \PY{o}{=} \PY{l+s+s2}{\PYZdq{}}\PY{l+s+s2}{Dimension 1}\PY{l+s+s2}{\PYZdq{}}\PY{p}{,} \PY{n}{ylabel} \PY{o}{=} \PY{l+s+s2}{\PYZdq{}}\PY{l+s+s2}{Dimension 2}\PY{l+s+s2}{\PYZdq{}}\PY{p}{)}
\PY{n}{g\PYZus{}pca}\PY{o}{.}\PY{n}{fig}\PY{o}{.}\PY{n}{suptitle}\PY{p}{(}\PY{l+s+s2}{\PYZdq{}}\PY{l+s+s2}{Premier plan factoriel}\PY{l+s+s2}{\PYZdq{}}\PY{p}{)}
\end{Verbatim}
\end{tcolorbox}

            \begin{tcolorbox}[breakable, size=fbox, boxrule=.5pt, pad at break*=1mm, opacityfill=0]
\prompt{Out}{outcolor}{44}{\boxspacing}
\begin{Verbatim}[commandchars=\\\{\}]
Text(0.5, 0.98, 'Premier plan factoriel')
\end{Verbatim}
\end{tcolorbox}
        
    \begin{center}
    \adjustimage{max size={0.9\linewidth}{0.9\paperheight}}{output_56_1.png}
    \end{center}
    { \hspace*{\fill} \\}
    
    \begin{tcolorbox}[breakable, size=fbox, boxrule=1pt, pad at break*=1mm,colback=cellbackground, colframe=cellborder]
\prompt{In}{incolor}{ }{\boxspacing}
\begin{Verbatim}[commandchars=\\\{\}]
\PY{n}{pca} \PY{o}{=} \PY{n}{PCA}\PY{p}{(}\PY{n}{n\PYZus{}components}\PY{o}{=}\PY{l+m+mi}{2}\PY{p}{)}
\PY{n}{components} \PY{o}{=} \PY{n}{pca}\PY{o}{.}\PY{n}{fit\PYZus{}transform}\PY{p}{(}\PY{n}{X\PYZus{}normalized}\PY{p}{)} 
\PY{n}{loadings} \PY{o}{=} \PY{n}{pca}\PY{o}{.}\PY{n}{components\PYZus{}}\PY{o}{.}\PY{n}{T} \PY{o}{*} \PY{n}{np}\PY{o}{.}\PY{n}{sqrt}\PY{p}{(}\PY{n}{pca}\PY{o}{.}\PY{n}{explained\PYZus{}variance\PYZus{}}\PY{p}{)}
\PY{n}{features}\PY{o}{=}\PY{n}{X}\PY{o}{.}\PY{n}{columns}
\PY{n}{fig}\PY{o}{=}\PY{n}{px}\PY{o}{.}\PY{n}{scatter}\PY{p}{(}\PY{n}{components}\PY{p}{,} \PY{n}{color}\PY{o}{=}\PY{n}{df}\PY{p}{[}\PY{n+nb}{str}\PY{p}{(}\PY{n}{value1}\PY{p}{)}\PY{p}{]}\PY{p}{,} \PY{n}{x}\PY{o}{=}\PY{l+m+mi}{0}\PY{p}{,} \PY{n}{y}\PY{o}{=}\PY{l+m+mi}{1}\PY{p}{,} \PY{n}{labels}\PY{o}{=}\PY{p}{\PYZob{}}\PY{l+s+s1}{\PYZsq{}}\PY{l+s+s1}{0}\PY{l+s+s1}{\PYZsq{}}\PY{p}{:}\PY{l+s+s1}{\PYZsq{}}\PY{l+s+s1}{PC1}\PY{l+s+s1}{\PYZsq{}}\PY{p}{,} \PY{l+s+s1}{\PYZsq{}}\PY{l+s+s1}{1}\PY{l+s+s1}{\PYZsq{}}\PY{p}{:}\PY{l+s+s1}{\PYZsq{}}\PY{l+s+s1}{PC2}\PY{l+s+s1}{\PYZsq{}}\PY{p}{\PYZcb{}}\PY{p}{)}
\PY{n}{features}\PY{o}{=}\PY{n}{y\PYZus{}train}                
\PY{c+c1}{\PYZsh{}Ajout de la sémantique des axes avec les variables d\PYZsq{}origine }
\PY{k}{for} \PY{n}{i}\PY{p}{,} \PY{n}{feature} \PY{o+ow}{in} \PY{n+nb}{enumerate}\PY{p}{(}\PY{n}{features}\PY{p}{)}\PY{p}{:}
    \PY{c+c1}{\PYZsh{}on ne prend en compte que les variables quanti}
    \PY{k}{if} \PY{n+nb}{str}\PY{p}{(}\PY{n}{X1}\PY{o}{.}\PY{n}{dtypes}\PY{p}{[}\PY{n+nb}{str}\PY{p}{(}\PY{n}{feature}\PY{p}{)}\PY{p}{]}\PY{p}{)}\PY{o}{!=}\PY{l+s+s1}{\PYZsq{}}\PY{l+s+s1}{object}\PY{l+s+s1}{\PYZsq{}}\PY{p}{:}
        \PY{n}{fig}\PY{o}{.}\PY{n}{add\PYZus{}shape}\PY{p}{(}
        \PY{n+nb}{type}\PY{o}{=}\PY{l+s+s1}{\PYZsq{}}\PY{l+s+s1}{line}\PY{l+s+s1}{\PYZsq{}}\PY{p}{,}
        \PY{n}{x0}\PY{o}{=}\PY{l+m+mi}{0}\PY{p}{,} \PY{n}{y0}\PY{o}{=}\PY{l+m+mi}{0}\PY{p}{,}
        \PY{n}{x1}\PY{o}{=}\PY{n}{loadings}\PY{p}{[}\PY{n}{i}\PY{p}{,} \PY{l+m+mi}{0}\PY{p}{]}\PY{p}{,}
        \PY{n}{y1}\PY{o}{=}\PY{n}{loadings}\PY{p}{[}\PY{n}{i}\PY{p}{,} \PY{l+m+mi}{1}\PY{p}{]}
                        \PY{p}{)}
        \PY{n}{fig}\PY{o}{.}\PY{n}{add\PYZus{}annotation}\PY{p}{(}
        \PY{n}{x}\PY{o}{=}\PY{n}{loadings}\PY{p}{[}\PY{n}{i}\PY{p}{,} \PY{l+m+mi}{0}\PY{p}{]}\PY{p}{,}
        \PY{n}{y}\PY{o}{=}\PY{n}{loadings}\PY{p}{[}\PY{n}{i}\PY{p}{,} \PY{l+m+mi}{1}\PY{p}{]}\PY{p}{,}
        \PY{n}{ax}\PY{o}{=}\PY{l+m+mi}{0}\PY{p}{,} \PY{n}{ay}\PY{o}{=}\PY{l+m+mi}{0}\PY{p}{,}
        \PY{n}{xanchor}\PY{o}{=}\PY{l+s+s2}{\PYZdq{}}\PY{l+s+s2}{center}\PY{l+s+s2}{\PYZdq{}}\PY{p}{,}
        \PY{n}{yanchor}\PY{o}{=}\PY{l+s+s2}{\PYZdq{}}\PY{l+s+s2}{bottom}\PY{l+s+s2}{\PYZdq{}}\PY{p}{,}
        \PY{n}{text}\PY{o}{=}\PY{n}{feature}\PY{p}{,}
                        \PY{p}{)}
\end{Verbatim}
\end{tcolorbox}

    Les données ne sont pas très clair à distiguer les classes à cause de
nombreux classes. Nous pouvons voir qu'il y a beaucoup plus de classe 0
que classe 1. Les classe sont assez concentré mais aussi mélangé de l'un
à l'autre

    Avec SMOTE:

    \begin{tcolorbox}[breakable, size=fbox, boxrule=1pt, pad at break*=1mm,colback=cellbackground, colframe=cellborder]
\prompt{In}{incolor}{45}{\boxspacing}
\begin{Verbatim}[commandchars=\\\{\}]
\PY{n}{pca} \PY{o}{=} \PY{n}{PCA}\PY{p}{(}\PY{n}{n\PYZus{}components}\PY{o}{=}\PY{l+m+mi}{2}\PY{p}{)}
\PY{n}{sc} \PY{o}{=} \PY{n}{StandardScaler}\PY{p}{(}\PY{p}{)} 
\PY{n}{X\PYZus{}normalized\PYZus{}res} \PY{o}{=} \PY{n}{sc}\PY{o}{.}\PY{n}{fit\PYZus{}transform}\PY{p}{(}\PY{n}{x\PYZus{}train\PYZus{}res}\PY{p}{)}  
\PY{n}{components\PYZus{}res} \PY{o}{=} \PY{n}{pca}\PY{o}{.}\PY{n}{fit\PYZus{}transform}\PY{p}{(}\PY{n}{X\PYZus{}normalized\PYZus{}res}\PY{p}{)} 
\PY{c+c1}{\PYZsh{} Transformation en DataFrame pandas}
\PY{n}{fraud\PYZus{}pca\PYZus{}df\PYZus{}res} \PY{o}{=} \PY{n}{pd}\PY{o}{.}\PY{n}{DataFrame}\PY{p}{(}\PY{p}{\PYZob{}}
    \PY{l+s+s2}{\PYZdq{}}\PY{l+s+s2}{Dim1}\PY{l+s+s2}{\PYZdq{}} \PY{p}{:} \PY{n}{components\PYZus{}res}\PY{p}{[}\PY{p}{:}\PY{p}{,}\PY{l+m+mi}{0}\PY{p}{]}\PY{p}{,} 
    \PY{l+s+s2}{\PYZdq{}}\PY{l+s+s2}{Dim2}\PY{l+s+s2}{\PYZdq{}} \PY{p}{:} \PY{n}{components\PYZus{}res}\PY{p}{[}\PY{p}{:}\PY{p}{,}\PY{l+m+mi}{1}\PY{p}{]}\PY{p}{,} 
    \PY{l+s+s2}{\PYZdq{}}\PY{l+s+s2}{Flag}\PY{l+s+s2}{\PYZdq{}} \PY{p}{:} \PY{n}{y\PYZus{}train\PYZus{}res}
\PY{p}{\PYZcb{}}\PY{p}{)}
\PY{n}{g\PYZus{}pca\PYZus{}res} \PY{o}{=} \PY{n}{sns}\PY{o}{.}\PY{n}{lmplot}\PY{p}{(}\PY{l+s+s2}{\PYZdq{}}\PY{l+s+s2}{Dim1}\PY{l+s+s2}{\PYZdq{}}\PY{p}{,} \PY{l+s+s2}{\PYZdq{}}\PY{l+s+s2}{Dim2}\PY{l+s+s2}{\PYZdq{}}\PY{p}{,} \PY{n}{hue} \PY{o}{=} \PY{l+s+s2}{\PYZdq{}}\PY{l+s+s2}{Flag}\PY{l+s+s2}{\PYZdq{}}\PY{p}{,} \PY{n}{data} \PY{o}{=} \PY{n}{fraud\PYZus{}pca\PYZus{}df\PYZus{}res}\PY{p}{,} \PY{n}{fit\PYZus{}reg} \PY{o}{=} \PY{k+kc}{False}\PY{p}{)}
\PY{n}{g\PYZus{}pca\PYZus{}res}\PY{o}{.}\PY{n}{set}\PY{p}{(}\PY{n}{xlabel} \PY{o}{=} \PY{l+s+s2}{\PYZdq{}}\PY{l+s+s2}{Dimension 1}\PY{l+s+s2}{\PYZdq{}}\PY{p}{,} \PY{n}{ylabel} \PY{o}{=} \PY{l+s+s2}{\PYZdq{}}\PY{l+s+s2}{Dimension 2}\PY{l+s+s2}{\PYZdq{}}\PY{p}{)}
\PY{n}{g\PYZus{}pca\PYZus{}res}\PY{o}{.}\PY{n}{fig}\PY{o}{.}\PY{n}{suptitle}\PY{p}{(}\PY{l+s+s2}{\PYZdq{}}\PY{l+s+s2}{Premier plan factoriel}\PY{l+s+s2}{\PYZdq{}}\PY{p}{)}
\end{Verbatim}
\end{tcolorbox}

            \begin{tcolorbox}[breakable, size=fbox, boxrule=.5pt, pad at break*=1mm, opacityfill=0]
\prompt{Out}{outcolor}{45}{\boxspacing}
\begin{Verbatim}[commandchars=\\\{\}]
Text(0.5, 0.98, 'Premier plan factoriel')
\end{Verbatim}
\end{tcolorbox}
        
    \begin{center}
    \adjustimage{max size={0.9\linewidth}{0.9\paperheight}}{output_60_1.png}
    \end{center}
    { \hspace*{\fill} \\}
    
    Les classes sont encore dur à observer.

    \hypertarget{algorithmes}{%
\subsection{7. Algorithmes}\label{algorithmes}}

Maintenant, nous pouvons construire les modèles en utilisant les
algorithmes différents de machine learning. Avant de créer ce modèle,
nous devons trouver le statement de notre problème, c'est à dire les
algortihmes supervisé et non - supervisé.

En fait, notre problème tombe sur le supervisé qui suppose que la base
de données a une value cible pour chaque ligne ou échantillon.

Les algorithmes supervisé ont deux types: - Classification - Regression

Crédit card fraud détection est un problème de classification. La valeur
cible est integer donc 1 ou 0 ou catégorique donc fraud ou non - fraud.

Pour résoudre notre problème nous avons décidé de tester plusieurs
modèles de prédiction afin de trouver le meilleur qui maximisera notre
valeur de f1.

Les algorithmes que nous allons appliquer sont :

\begin{itemize}
\tightlist
\item
  La forêt aléatoire (random forest)
\item
  Les K plus proche voisins (k Nearest Neighbor)
\item
  L'arbre de décision (Decision Tree)
\item
  L'analyse discriminent linéaire
\item
  Des modèles ensembliste
\item
  Xgboost
\item
  Modèle de one class classification : isolation forest
\end{itemize}

Pour chacun des algorithmes nous avons tester plusieurs valeurs pour les
hyper-paramètres afin de trouver les meilleurs si le temps de calcul est
optimisé.

Nous n'allons pas faire une cross-validation pour les données comme les
transaction sont chronologiques. Nous pourrions utiliser TimeSeriesSplit
pour que ca repeste le chronologie de train et validation, cependant les
taches semble plus compliqué et le résultat ne sera pas beaucoup plus
améliorer

    Sorties communes : - Temps de calcul: Très simplement, il s'agit du
temps en secondes qu'a mis le modèle en question pour trouver les
résultats. Il démarre lorsque les paramètres sont choisis (optimaux ou
manuel).

\begin{itemize}
\item
  Matrice de confusion sous forme graphique: La matrice de confusion est
  une matrice qui mesure la qualité d'un système de classification.
  Chaque ligne correspond à une classe réelle, chaque colonne correspond
  à une classe estimée. La cellule ligne L, colonne C contient le nombre
  d'éléments de la classe réelle L qui ont été estimés comme appartenant
  à la classe C1. Un des intérêts de la matrice de confusion est qu'elle
  montre rapidement si un système de classification parvient à
  classifier correctement.
\item
  F1-score: La mesure F est la moyenne harmonique de la précision et de
  la sensibilité. Si vous optimisez votre classificateur pour augmenter
  l'un et défavoriser l'autre, la moyenne des harmoniques diminue
  rapidement.
\item
  Courbe ROC et AUC: La courbe ROC est un outil d'évaluation et de
  comparaison des modèles. L'idée est de faire varier le « seuil » de 1
  à 0. AUC indique la probabilité pour que la fonction SCORE place un
  positif devant négatif (dans le meilleur des cas AUC =1
\end{itemize}

    \hypertarget{dictionaire}{%
\paragraph{Dictionaire}\label{dictionaire}}

    \begin{tcolorbox}[breakable, size=fbox, boxrule=1pt, pad at break*=1mm,colback=cellbackground, colframe=cellborder]
\prompt{In}{incolor}{162}{\boxspacing}
\begin{Verbatim}[commandchars=\\\{\}]
\PY{c+c1}{\PYZsh{}Ajout dictionnaire}
\PY{n}{test\PYZus{}result} \PY{o}{=} \PY{p}{\PYZob{}}\PY{p}{\PYZcb{}}
\PY{n}{valid\PYZus{}result} \PY{o}{=} \PY{p}{\PYZob{}}\PY{p}{\PYZcb{}}
\PY{n}{auc\PYZus{}result} \PY{o}{=} \PY{p}{\PYZob{}}\PY{p}{\PYZcb{}}
\PY{n}{time\PYZus{}result} \PY{o}{=} \PY{p}{\PYZob{}}\PY{p}{\PYZcb{}}
\end{Verbatim}
\end{tcolorbox}

    \hypertarget{k-nearest-neighbor}{%
\subsubsection{7.1 K Nearest Neighbor}\label{k-nearest-neighbor}}

    Knn ou méthode des k plus proches voisins est un algorithme standard de
classification qui repose exclusivement sur le choix de la métrique de
classification. Il est ``non paramétrique'' (seul k doit être fixé) et
se base uniquement sur les données d'entraînement.

L'idée est la suivante : à partir d'une base de données étiquetées, on
peut estimer la classe d'une nouvelle donnée en regardant quelle est la
classe majoritaire des k données voisines les plus proches (d'où le nom
de l'algorithme). Le seul paramètre à fixer est k, le nombre de voisins
à considérer.

    \begin{tcolorbox}[breakable, size=fbox, boxrule=1pt, pad at break*=1mm,colback=cellbackground, colframe=cellborder]
\prompt{In}{incolor}{92}{\boxspacing}
\begin{Verbatim}[commandchars=\\\{\}]
\PY{n}{pipeKNN} \PY{o}{=} \PY{n}{make\PYZus{}pipeline}\PY{p}{(}\PY{n}{StandardScaler}\PY{p}{(}\PY{p}{)}\PY{p}{,} \PY{n}{KNeighborsClassifier}\PY{p}{(}\PY{p}{)}\PY{p}{)}
\PY{n}{startKNN} \PY{o}{=} \PY{n}{time}\PY{p}{(}\PY{p}{)}
\PY{n}{pipeKNN}\PY{o}{.}\PY{n}{fit}\PY{p}{(}\PY{n}{x\PYZus{}train\PYZus{}res}\PY{p}{,} \PY{n}{y\PYZus{}train\PYZus{}res}\PY{p}{)}  
\PY{n}{endKNN} \PY{o}{=} \PY{n}{time}\PY{p}{(}\PY{p}{)}
\PY{n}{KNN}\PY{o}{=} \PY{p}{(}\PY{n}{endKNN}\PY{o}{\PYZhy{}}\PY{n}{startKNN}\PY{p}{)}\PY{o}{/}\PY{l+m+mi}{60}
\PY{n+nb}{print}\PY{p}{(}\PY{l+s+s2}{\PYZdq{}}\PY{l+s+s2}{temps d}\PY{l+s+s2}{\PYZsq{}}\PY{l+s+s2}{execution}\PY{l+s+s2}{\PYZdq{}}\PY{p}{,}\PY{n}{KNN}\PY{p}{)}
\end{Verbatim}
\end{tcolorbox}

    \begin{Verbatim}[commandchars=\\\{\}]
temps d'execution 85.573710501194
    \end{Verbatim}

    \hypertarget{validation}{%
\subparagraph{Validation :}\label{validation}}

    \begin{tcolorbox}[breakable, size=fbox, boxrule=1pt, pad at break*=1mm,colback=cellbackground, colframe=cellborder]
\prompt{In}{incolor}{106}{\boxspacing}
\begin{Verbatim}[commandchars=\\\{\}]
\PY{c+c1}{\PYZsh{}Validation}
\PY{n}{f1\PYZus{}knn} \PY{o}{=} \PY{n+nb}{round}\PY{p}{(}\PY{n}{f1\PYZus{}score}\PY{p}{(}\PY{n}{y\PYZus{}val}\PY{p}{,} \PY{n}{pipeKNN}\PY{o}{.}\PY{n}{predict}\PY{p}{(}\PY{n}{x\PYZus{}val}\PY{p}{)}\PY{p}{)}\PY{p}{,}\PY{l+m+mi}{6}\PY{p}{)}
\PY{n+nb}{print}\PY{p}{(}\PY{n}{metrics}\PY{o}{.}\PY{n}{classification\PYZus{}report}\PY{p}{(}\PY{n}{y\PYZus{}val}\PY{p}{,} \PY{n}{pipeKNN}\PY{o}{.}\PY{n}{predict}\PY{p}{(}\PY{n}{x\PYZus{}val}\PY{p}{)}\PY{p}{)}\PY{p}{)}
\PY{n+nb}{print}\PY{p}{(}\PY{l+s+s2}{\PYZdq{}}\PY{l+s+s2}{Validation score f1 : }\PY{l+s+s2}{\PYZdq{}}\PY{p}{,} \PY{n}{f1\PYZus{}knn}\PY{p}{)}
\end{Verbatim}
\end{tcolorbox}

    \begin{Verbatim}[commandchars=\\\{\}]
              precision    recall  f1-score   support

           0       1.00      0.99      0.99    288737
           1       0.03      0.11      0.05       948

    accuracy                           0.99    289685
   macro avg       0.51      0.55      0.52    289685
weighted avg       0.99      0.99      0.99    289685

Validation score f1 :  0.048721
    \end{Verbatim}

    \hypertarget{test}{%
\subparagraph{Test :}\label{test}}

    \begin{tcolorbox}[breakable, size=fbox, boxrule=1pt, pad at break*=1mm,colback=cellbackground, colframe=cellborder]
\prompt{In}{incolor}{113}{\boxspacing}
\begin{Verbatim}[commandchars=\\\{\}]
\PY{c+c1}{\PYZsh{}Matrice de confusion}
\PY{n}{knn\PYZus{}matrix\PYZus{}val} \PY{o}{=} \PY{n}{confusion\PYZus{}matrix}\PY{p}{(}\PY{n}{test\PYZus{}target}\PY{p}{,}\PY{n}{pipeKNN}\PY{o}{.}\PY{n}{predict}\PY{p}{(}\PY{n}{test\PYZus{}features}\PY{p}{)}\PY{p}{)}
\PY{n}{group\PYZus{}names} \PY{o}{=} \PY{p}{[}\PY{l+s+s1}{\PYZsq{}}\PY{l+s+s1}{True Neg}\PY{l+s+s1}{\PYZsq{}}\PY{p}{,}\PY{l+s+s1}{\PYZsq{}}\PY{l+s+s1}{False Pos}\PY{l+s+s1}{\PYZsq{}}\PY{p}{,}\PY{l+s+s1}{\PYZsq{}}\PY{l+s+s1}{False Neg}\PY{l+s+s1}{\PYZsq{}}\PY{p}{,}\PY{l+s+s1}{\PYZsq{}}\PY{l+s+s1}{True Pos}\PY{l+s+s1}{\PYZsq{}}\PY{p}{]}
\PY{n}{group\PYZus{}counts} \PY{o}{=} \PY{p}{[}\PY{l+s+s2}{\PYZdq{}}\PY{l+s+si}{\PYZob{}0:0.0f\PYZcb{}}\PY{l+s+s2}{\PYZdq{}}\PY{o}{.}\PY{n}{format}\PY{p}{(}\PY{n}{value}\PY{p}{)} \PY{k}{for} \PY{n}{value} \PY{o+ow}{in}
                \PY{n}{knn\PYZus{}matrix\PYZus{}val}\PY{o}{.}\PY{n}{flatten}\PY{p}{(}\PY{p}{)}\PY{p}{]}
\PY{n}{group\PYZus{}percentages} \PY{o}{=} \PY{p}{[}\PY{l+s+s2}{\PYZdq{}}\PY{l+s+si}{\PYZob{}0:.2\PYZpc{}\PYZcb{}}\PY{l+s+s2}{\PYZdq{}}\PY{o}{.}\PY{n}{format}\PY{p}{(}\PY{n}{value}\PY{p}{)} \PY{k}{for} \PY{n}{value} \PY{o+ow}{in}
                     \PY{n}{knn\PYZus{}matrix\PYZus{}val}\PY{o}{.}\PY{n}{flatten}\PY{p}{(}\PY{p}{)}\PY{o}{/}\PY{n}{np}\PY{o}{.}\PY{n}{sum}\PY{p}{(}\PY{n}{knn\PYZus{}matrix\PYZus{}val}\PY{p}{)}\PY{p}{]}
\PY{n}{labels} \PY{o}{=} \PY{p}{[}\PY{l+s+sa}{f}\PY{l+s+s2}{\PYZdq{}}\PY{l+s+si}{\PYZob{}}\PY{n}{v1}\PY{l+s+si}{\PYZcb{}}\PY{l+s+se}{\PYZbs{}n}\PY{l+s+si}{\PYZob{}}\PY{n}{v2}\PY{l+s+si}{\PYZcb{}}\PY{l+s+se}{\PYZbs{}n}\PY{l+s+si}{\PYZob{}}\PY{n}{v3}\PY{l+s+si}{\PYZcb{}}\PY{l+s+s2}{\PYZdq{}} \PY{k}{for} \PY{n}{v1}\PY{p}{,} \PY{n}{v2}\PY{p}{,} \PY{n}{v3} \PY{o+ow}{in}
          \PY{n+nb}{zip}\PY{p}{(}\PY{n}{group\PYZus{}names}\PY{p}{,}\PY{n}{group\PYZus{}counts}\PY{p}{,}\PY{n}{group\PYZus{}percentages}\PY{p}{)}\PY{p}{]}
\PY{n}{labels} \PY{o}{=} \PY{n}{np}\PY{o}{.}\PY{n}{asarray}\PY{p}{(}\PY{n}{labels}\PY{p}{)}\PY{o}{.}\PY{n}{reshape}\PY{p}{(}\PY{l+m+mi}{2}\PY{p}{,}\PY{l+m+mi}{2}\PY{p}{)}
\PY{n}{sns}\PY{o}{.}\PY{n}{heatmap}\PY{p}{(}\PY{n}{knn\PYZus{}matrix\PYZus{}val}\PY{p}{,} \PY{n}{annot}\PY{o}{=}\PY{n}{labels}\PY{p}{,} \PY{n}{fmt}\PY{o}{=}\PY{l+s+s1}{\PYZsq{}}\PY{l+s+s1}{\PYZsq{}}\PY{p}{,} \PY{n}{cmap}\PY{o}{=}\PY{l+s+s1}{\PYZsq{}}\PY{l+s+s1}{Blues}\PY{l+s+s1}{\PYZsq{}}\PY{p}{)}
\end{Verbatim}
\end{tcolorbox}

            \begin{tcolorbox}[breakable, size=fbox, boxrule=.5pt, pad at break*=1mm, opacityfill=0]
\prompt{Out}{outcolor}{113}{\boxspacing}
\begin{Verbatim}[commandchars=\\\{\}]
<matplotlib.axes.\_subplots.AxesSubplot at 0x7fe4f9c56cd0>
\end{Verbatim}
\end{tcolorbox}
        
    \begin{center}
    \adjustimage{max size={0.9\linewidth}{0.9\paperheight}}{output_72_1.png}
    \end{center}
    { \hspace*{\fill} \\}
    
    \begin{tcolorbox}[breakable, size=fbox, boxrule=1pt, pad at break*=1mm,colback=cellbackground, colframe=cellborder]
\prompt{In}{incolor}{114}{\boxspacing}
\begin{Verbatim}[commandchars=\\\{\}]
\PY{c+c1}{\PYZsh{}Test}
\PY{n}{test\PYZus{}knn} \PY{o}{=} \PY{n+nb}{round}\PY{p}{(}\PY{n}{f1\PYZus{}score}\PY{p}{(}\PY{n}{test\PYZus{}target}\PY{p}{,}\PY{n}{pipeKNN}\PY{o}{.}\PY{n}{predict}\PY{p}{(}\PY{n}{test\PYZus{}features}\PY{p}{)}\PY{p}{)}\PY{p}{,}\PY{l+m+mi}{6}\PY{p}{)}
\PY{n+nb}{print}\PY{p}{(}\PY{n}{metrics}\PY{o}{.}\PY{n}{classification\PYZus{}report}\PY{p}{(}\PY{n}{test\PYZus{}target}\PY{p}{,}\PY{n}{pipeKNN}\PY{o}{.}\PY{n}{predict}\PY{p}{(}\PY{n}{test\PYZus{}features}\PY{p}{)}\PY{p}{)}\PY{p}{)}
\PY{n+nb}{print}\PY{p}{(}\PY{l+s+s2}{\PYZdq{}}\PY{l+s+s2}{Test score f1 : }\PY{l+s+s2}{\PYZdq{}}\PY{p}{,} \PY{n}{test\PYZus{}knn}\PY{p}{)}
\end{Verbatim}
\end{tcolorbox}

    \begin{Verbatim}[commandchars=\\\{\}]
              precision    recall  f1-score   support

           0       1.00      0.99      0.99    263288
           1       0.03      0.08      0.04       855

    accuracy                           0.99    264143
   macro avg       0.51      0.54      0.52    264143
weighted avg       0.99      0.99      0.99    264143

Test score f1 :  0.039632
    \end{Verbatim}

    \begin{tcolorbox}[breakable, size=fbox, boxrule=1pt, pad at break*=1mm,colback=cellbackground, colframe=cellborder]
\prompt{In}{incolor}{130}{\boxspacing}
\begin{Verbatim}[commandchars=\\\{\}]
\PY{c+c1}{\PYZsh{}Courbe ROC}
\PY{n}{fpr}\PY{p}{,} \PY{n}{tpr}\PY{p}{,} \PY{n}{\PYZus{}} \PY{o}{=} \PY{n}{roc\PYZus{}curve}\PY{p}{(}\PY{n}{test\PYZus{}target}\PY{p}{,} \PY{n}{pipeKNN}\PY{o}{.}\PY{n}{predict}\PY{p}{(}\PY{n}{test\PYZus{}features}\PY{p}{)}\PY{p}{)}
\PY{n}{auc\PYZus{}knn} \PY{o}{=} \PY{n}{metrics}\PY{o}{.}\PY{n}{roc\PYZus{}auc\PYZus{}score}\PY{p}{(}\PY{n}{test\PYZus{}target}\PY{p}{,}\PY{n}{pipeKNN}\PY{o}{.}\PY{n}{predict}\PY{p}{(}\PY{n}{test\PYZus{}features}\PY{p}{)}\PY{p}{)}
\PY{n}{plt}\PY{o}{.}\PY{n}{plot}\PY{p}{(}\PY{n}{fpr}\PY{p}{,}\PY{n}{tpr}\PY{p}{,}\PY{n}{label}\PY{o}{=}\PY{l+s+s2}{\PYZdq{}}\PY{l+s+s2}{data 1, auc=}\PY{l+s+s2}{\PYZdq{}}\PY{o}{+}\PY{n+nb}{str}\PY{p}{(}\PY{n}{auc\PYZus{}knn}\PY{p}{)}\PY{p}{)}
\PY{n}{plt}\PY{o}{.}\PY{n}{legend}\PY{p}{(}\PY{n}{loc}\PY{o}{=}\PY{l+m+mi}{4}\PY{p}{)}
\PY{n}{plt}\PY{o}{.}\PY{n}{show}\PY{p}{(}\PY{p}{)}
\end{Verbatim}
\end{tcolorbox}

    \begin{center}
    \adjustimage{max size={0.9\linewidth}{0.9\paperheight}}{output_74_0.png}
    \end{center}
    { \hspace*{\fill} \\}
    
    Nous obtenos le f1 à 4\% ce qui n'est pas très impressionnant. En plus
avec le temps d'exécution est long et seulement 3 \% Vrai Possitif donc
knn n'est pas vraiment un modèle idéal.

    \begin{tcolorbox}[breakable, size=fbox, boxrule=1pt, pad at break*=1mm,colback=cellbackground, colframe=cellborder]
\prompt{In}{incolor}{163}{\boxspacing}
\begin{Verbatim}[commandchars=\\\{\}]
\PY{c+c1}{\PYZsh{}Ajout dictionnaire}
\PY{n}{test\PYZus{}result}\PY{p}{[}\PY{l+s+s1}{\PYZsq{}}\PY{l+s+s1}{KNN}\PY{l+s+s1}{\PYZsq{}}\PY{p}{]} \PY{o}{=} \PY{n}{test\PYZus{}knn}
\PY{n}{valid\PYZus{}result}\PY{p}{[}\PY{l+s+s1}{\PYZsq{}}\PY{l+s+s1}{KNN}\PY{l+s+s1}{\PYZsq{}}\PY{p}{]} \PY{o}{=} \PY{n}{f1\PYZus{}knn}
\PY{n}{auc\PYZus{}result}\PY{p}{[}\PY{l+s+s1}{\PYZsq{}}\PY{l+s+s1}{KNN}\PY{l+s+s1}{\PYZsq{}}\PY{p}{]}\PY{o}{=}\PY{n+nb}{round}\PY{p}{(}\PY{n}{auc\PYZus{}knn}\PY{p}{,}\PY{l+m+mi}{3}\PY{p}{)}
\PY{n}{time\PYZus{}result}\PY{p}{[}\PY{l+s+s1}{\PYZsq{}}\PY{l+s+s1}{KNN}\PY{l+s+s1}{\PYZsq{}}\PY{p}{]}\PY{o}{=}\PY{n+nb}{round}\PY{p}{(}\PY{n}{KNN}\PY{p}{,}\PY{l+m+mi}{3}\PY{p}{)}
\PY{n+nb}{print}\PY{p}{(}\PY{n}{test\PYZus{}result}\PY{p}{)}
\PY{n+nb}{print}\PY{p}{(}\PY{n}{valid\PYZus{}result}\PY{p}{)}
\PY{n+nb}{print}\PY{p}{(}\PY{n}{auc\PYZus{}result}\PY{p}{)}
\PY{n+nb}{print}\PY{p}{(}\PY{n}{time\PYZus{}result}\PY{p}{)}
\end{Verbatim}
\end{tcolorbox}

    \begin{Verbatim}[commandchars=\\\{\}]
\{'KNN': 0.039632\}
\{'KNN': 0.048721\}
\{'KNN': 0.536\}
\{'KNN': 85.574\}
    \end{Verbatim}

    \hypertarget{decision-tree}{%
\subsubsection{7.2 Decision Tree}\label{decision-tree}}

Le second algorithme de classification que nous avons utilisé est
l'arbre de classification. Il permet de prédire une variable cible
discrète. Le principe est le même celui de l'arbre de décision, et
fonctionne comme l'arbre de régression. L'arbre est construit grâce à un
partitionnement récursif des données. Le principe est alors celui de
séparer les données via des partitions, qui vont elles-mêmes se séparer
à leur tour à mesure que l'on va descendre en profondeur dans l'arbre.
Les avantages de cette méthode sont les mêmes que ceux que l'arbre de
régression, les résultats obtenus par cette méthode sont très
satisfaisants quand on les compare avec le rapide temps de calcul et la
faible complexité de l'algorithme.

    \begin{tcolorbox}[breakable, size=fbox, boxrule=1pt, pad at break*=1mm,colback=cellbackground, colframe=cellborder]
\prompt{In}{incolor}{36}{\boxspacing}
\begin{Verbatim}[commandchars=\\\{\}]
\PY{n}{pipelineDT} \PY{o}{=} \PY{n}{make\PYZus{}pipeline}\PY{p}{(}\PY{n}{StandardScaler}\PY{p}{(}\PY{p}{)}\PY{p}{,} \PY{n}{DecisionTreeClassifier}\PY{p}{(}\PY{p}{)}\PY{p}{)}
\end{Verbatim}
\end{tcolorbox}

    \begin{tcolorbox}[breakable, size=fbox, boxrule=1pt, pad at break*=1mm,colback=cellbackground, colframe=cellborder]
\prompt{In}{incolor}{37}{\boxspacing}
\begin{Verbatim}[commandchars=\\\{\}]
\PY{n}{paramsDT} \PY{o}{=} \PY{p}{\PYZob{}}\PY{l+s+s2}{\PYZdq{}}\PY{l+s+s2}{decisiontreeclassifier\PYZus{}\PYZus{}max\PYZus{}depth}\PY{l+s+s2}{\PYZdq{}}\PY{p}{:} \PY{p}{[}\PY{l+m+mi}{3}\PY{p}{,}\PY{l+m+mi}{6}\PY{p}{,}\PY{l+m+mi}{9}\PY{p}{,}\PY{l+m+mi}{12}\PY{p}{,} \PY{k+kc}{None}\PY{p}{]}\PY{p}{,}
              \PY{l+s+s2}{\PYZdq{}}\PY{l+s+s2}{decisiontreeclassifier\PYZus{}\PYZus{}min\PYZus{}samples\PYZus{}leaf}\PY{l+s+s2}{\PYZdq{}}\PY{p}{:} \PY{n}{np}\PY{o}{.}\PY{n}{arange}\PY{p}{(}\PY{l+m+mi}{1}\PY{p}{,}\PY{l+m+mi}{9}\PY{p}{,}\PY{l+m+mi}{1}\PY{p}{)}\PY{p}{,}
              \PY{l+s+s2}{\PYZdq{}}\PY{l+s+s2}{decisiontreeclassifier\PYZus{}\PYZus{}criterion}\PY{l+s+s2}{\PYZdq{}}\PY{p}{:} \PY{p}{[}\PY{l+s+s2}{\PYZdq{}}\PY{l+s+s2}{gini}\PY{l+s+s2}{\PYZdq{}}\PY{p}{,} \PY{l+s+s2}{\PYZdq{}}\PY{l+s+s2}{entropy}\PY{l+s+s2}{\PYZdq{}}\PY{p}{]}\PY{p}{\PYZcb{}}
\PY{c+c1}{\PYZsh{}params = \PYZob{}\PYZsq{}decisiontreeclassifier\PYZus{}\PYZus{}max\PYZus{}depth\PYZsq{}: [3, None]\PYZcb{}}
\end{Verbatim}
\end{tcolorbox}

    \begin{tcolorbox}[breakable, size=fbox, boxrule=1pt, pad at break*=1mm,colback=cellbackground, colframe=cellborder]
\prompt{In}{incolor}{38}{\boxspacing}
\begin{Verbatim}[commandchars=\\\{\}]
\PY{n+nb}{sorted}\PY{p}{(}\PY{n}{pipelineDT}\PY{o}{.}\PY{n}{get\PYZus{}params}\PY{p}{(}\PY{p}{)}\PY{o}{.}\PY{n}{keys}\PY{p}{(}\PY{p}{)}\PY{p}{)}
\end{Verbatim}
\end{tcolorbox}

            \begin{tcolorbox}[breakable, size=fbox, boxrule=.5pt, pad at break*=1mm, opacityfill=0]
\prompt{Out}{outcolor}{38}{\boxspacing}
\begin{Verbatim}[commandchars=\\\{\}]
['decisiontreeclassifier',
 'decisiontreeclassifier\_\_ccp\_alpha',
 'decisiontreeclassifier\_\_class\_weight',
 'decisiontreeclassifier\_\_criterion',
 'decisiontreeclassifier\_\_max\_depth',
 'decisiontreeclassifier\_\_max\_features',
 'decisiontreeclassifier\_\_max\_leaf\_nodes',
 'decisiontreeclassifier\_\_min\_impurity\_decrease',
 'decisiontreeclassifier\_\_min\_impurity\_split',
 'decisiontreeclassifier\_\_min\_samples\_leaf',
 'decisiontreeclassifier\_\_min\_samples\_split',
 'decisiontreeclassifier\_\_min\_weight\_fraction\_leaf',
 'decisiontreeclassifier\_\_presort',
 'decisiontreeclassifier\_\_random\_state',
 'decisiontreeclassifier\_\_splitter',
 'memory',
 'standardscaler',
 'standardscaler\_\_copy',
 'standardscaler\_\_with\_mean',
 'standardscaler\_\_with\_std',
 'steps',
 'verbose']
\end{Verbatim}
\end{tcolorbox}
        
    \begin{tcolorbox}[breakable, size=fbox, boxrule=1pt, pad at break*=1mm,colback=cellbackground, colframe=cellborder]
\prompt{In}{incolor}{39}{\boxspacing}
\begin{Verbatim}[commandchars=\\\{\}]
\PY{n}{gridsearchDT} \PY{o}{=} \PY{n}{GridSearchCV}\PY{p}{(}\PY{n}{estimator}\PY{o}{=}\PY{n}{pipelineDT}\PY{p}{,}
                       \PY{n}{param\PYZus{}grid}\PY{o}{=}\PY{n}{paramsDT}\PY{p}{,}
                       \PY{n}{n\PYZus{}jobs}\PY{o}{=}\PY{o}{\PYZhy{}}\PY{l+m+mi}{1}\PY{p}{,}
                       \PY{n}{scoring}\PY{o}{=}\PY{l+s+s1}{\PYZsq{}}\PY{l+s+s1}{f1\PYZus{}micro}\PY{l+s+s1}{\PYZsq{}}\PY{p}{)}
\PY{n}{startDT} \PY{o}{=} \PY{n}{time}\PY{p}{(}\PY{p}{)}
\PY{n}{gridsearchDT}\PY{o}{.}\PY{n}{fit}\PY{p}{(}\PY{n}{x\PYZus{}train\PYZus{}res}\PY{p}{,} \PY{n}{y\PYZus{}train\PYZus{}res}\PY{p}{)}
\PY{n}{endDT} \PY{o}{=} \PY{n}{time}\PY{p}{(}\PY{p}{)}
\PY{n}{DT} \PY{o}{=} \PY{p}{(}\PY{n}{endDT}\PY{o}{\PYZhy{}}\PY{n}{startDT}\PY{p}{)}\PY{o}{/}\PY{l+m+mi}{60}
\PY{n+nb}{print}\PY{p}{(}\PY{n}{DT}\PY{p}{)}
\end{Verbatim}
\end{tcolorbox}

    \begin{Verbatim}[commandchars=\\\{\}]
63.41144330501557
    \end{Verbatim}

    \begin{tcolorbox}[breakable, size=fbox, boxrule=1pt, pad at break*=1mm,colback=cellbackground, colframe=cellborder]
\prompt{In}{incolor}{40}{\boxspacing}
\begin{Verbatim}[commandchars=\\\{\}]
\PY{n}{gridsearchDT}\PY{o}{.}\PY{n}{best\PYZus{}estimator\PYZus{}}\PY{o}{.}\PY{n}{get\PYZus{}params}\PY{p}{(}\PY{p}{)}
\end{Verbatim}
\end{tcolorbox}

            \begin{tcolorbox}[breakable, size=fbox, boxrule=.5pt, pad at break*=1mm, opacityfill=0]
\prompt{Out}{outcolor}{40}{\boxspacing}
\begin{Verbatim}[commandchars=\\\{\}]
\{'memory': None,
 'steps': [('standardscaler', StandardScaler()),
  ('decisiontreeclassifier',
   DecisionTreeClassifier(criterion='entropy', min\_samples\_leaf=2))],
 'verbose': False,
 'standardscaler': StandardScaler(),
 'decisiontreeclassifier': DecisionTreeClassifier(criterion='entropy',
min\_samples\_leaf=2),
 'standardscaler\_\_copy': True,
 'standardscaler\_\_with\_mean': True,
 'standardscaler\_\_with\_std': True,
 'decisiontreeclassifier\_\_ccp\_alpha': 0.0,
 'decisiontreeclassifier\_\_class\_weight': None,
 'decisiontreeclassifier\_\_criterion': 'entropy',
 'decisiontreeclassifier\_\_max\_depth': None,
 'decisiontreeclassifier\_\_max\_features': None,
 'decisiontreeclassifier\_\_max\_leaf\_nodes': None,
 'decisiontreeclassifier\_\_min\_impurity\_decrease': 0.0,
 'decisiontreeclassifier\_\_min\_impurity\_split': None,
 'decisiontreeclassifier\_\_min\_samples\_leaf': 2,
 'decisiontreeclassifier\_\_min\_samples\_split': 2,
 'decisiontreeclassifier\_\_min\_weight\_fraction\_leaf': 0.0,
 'decisiontreeclassifier\_\_presort': 'deprecated',
 'decisiontreeclassifier\_\_random\_state': None,
 'decisiontreeclassifier\_\_splitter': 'best'\}
\end{Verbatim}
\end{tcolorbox}
        
    \hypertarget{validation}{%
\subparagraph{Validation :}\label{validation}}

    \begin{tcolorbox}[breakable, size=fbox, boxrule=1pt, pad at break*=1mm,colback=cellbackground, colframe=cellborder]
\prompt{In}{incolor}{41}{\boxspacing}
\begin{Verbatim}[commandchars=\\\{\}]
\PY{c+c1}{\PYZsh{}Métriques d\PYZsq{}évaluation}
\PY{n}{f1\PYZus{}dt} \PY{o}{=} \PY{n+nb}{round}\PY{p}{(}\PY{n}{f1\PYZus{}score}\PY{p}{(}\PY{n}{y\PYZus{}val}\PY{p}{,} \PY{n}{gridsearchDT}\PY{o}{.}\PY{n}{predict}\PY{p}{(}\PY{n}{x\PYZus{}val}\PY{p}{)}\PY{p}{)}\PY{p}{,}\PY{l+m+mi}{6}\PY{p}{)}
\PY{n+nb}{print}\PY{p}{(}\PY{n}{metrics}\PY{o}{.}\PY{n}{classification\PYZus{}report}\PY{p}{(}\PY{n}{y\PYZus{}val}\PY{p}{,} \PY{n}{gridsearchDT}\PY{o}{.}\PY{n}{predict}\PY{p}{(}\PY{n}{x\PYZus{}val}\PY{p}{)}\PY{p}{)}\PY{p}{)}
\PY{n+nb}{print}\PY{p}{(}\PY{l+s+s2}{\PYZdq{}}\PY{l+s+s2}{Validation score f1:}\PY{l+s+s2}{\PYZdq{}}\PY{p}{,} \PY{n}{f1\PYZus{}dt} \PY{p}{)}
\end{Verbatim}
\end{tcolorbox}

    \begin{Verbatim}[commandchars=\\\{\}]
              precision    recall  f1-score   support

           0       1.00      1.00      1.00    288737
           1       0.03      0.03      0.03       948

    accuracy                           0.99    289685
   macro avg       0.51      0.51      0.51    289685
weighted avg       0.99      0.99      0.99    289685

Validation score f1: 0.030846
    \end{Verbatim}

    \hypertarget{test}{%
\subparagraph{Test :}\label{test}}

    \begin{tcolorbox}[breakable, size=fbox, boxrule=1pt, pad at break*=1mm,colback=cellbackground, colframe=cellborder]
\prompt{In}{incolor}{42}{\boxspacing}
\begin{Verbatim}[commandchars=\\\{\}]
\PY{c+c1}{\PYZsh{}Matrice de confusion}
\PY{n}{dt\PYZus{}matrix\PYZus{}val} \PY{o}{=} \PY{n}{confusion\PYZus{}matrix}\PY{p}{(}\PY{n}{test\PYZus{}target}\PY{p}{,} \PY{n}{gridsearchDT}\PY{o}{.}\PY{n}{predict}\PY{p}{(}\PY{n}{test\PYZus{}features}\PY{p}{)}\PY{p}{)}
\PY{n}{group\PYZus{}names} \PY{o}{=} \PY{p}{[}\PY{l+s+s1}{\PYZsq{}}\PY{l+s+s1}{True Neg}\PY{l+s+s1}{\PYZsq{}}\PY{p}{,}\PY{l+s+s1}{\PYZsq{}}\PY{l+s+s1}{False Pos}\PY{l+s+s1}{\PYZsq{}}\PY{p}{,}\PY{l+s+s1}{\PYZsq{}}\PY{l+s+s1}{False Neg}\PY{l+s+s1}{\PYZsq{}}\PY{p}{,}\PY{l+s+s1}{\PYZsq{}}\PY{l+s+s1}{True Pos}\PY{l+s+s1}{\PYZsq{}}\PY{p}{]}
\PY{n}{group\PYZus{}counts} \PY{o}{=} \PY{p}{[}\PY{l+s+s2}{\PYZdq{}}\PY{l+s+si}{\PYZob{}0:0.0f\PYZcb{}}\PY{l+s+s2}{\PYZdq{}}\PY{o}{.}\PY{n}{format}\PY{p}{(}\PY{n}{value}\PY{p}{)} \PY{k}{for} \PY{n}{value} \PY{o+ow}{in}
                \PY{n}{dt\PYZus{}matrix\PYZus{}val}\PY{o}{.}\PY{n}{flatten}\PY{p}{(}\PY{p}{)}\PY{p}{]}
\PY{n}{group\PYZus{}percentages} \PY{o}{=} \PY{p}{[}\PY{l+s+s2}{\PYZdq{}}\PY{l+s+si}{\PYZob{}0:.2\PYZpc{}\PYZcb{}}\PY{l+s+s2}{\PYZdq{}}\PY{o}{.}\PY{n}{format}\PY{p}{(}\PY{n}{value}\PY{p}{)} \PY{k}{for} \PY{n}{value} \PY{o+ow}{in}
                     \PY{n}{dt\PYZus{}matrix\PYZus{}val}\PY{o}{.}\PY{n}{flatten}\PY{p}{(}\PY{p}{)}\PY{o}{/}\PY{n}{np}\PY{o}{.}\PY{n}{sum}\PY{p}{(}\PY{n}{dt\PYZus{}matrix\PYZus{}val}\PY{p}{)}\PY{p}{]}
\PY{n}{labels} \PY{o}{=} \PY{p}{[}\PY{l+s+sa}{f}\PY{l+s+s2}{\PYZdq{}}\PY{l+s+si}{\PYZob{}}\PY{n}{v1}\PY{l+s+si}{\PYZcb{}}\PY{l+s+se}{\PYZbs{}n}\PY{l+s+si}{\PYZob{}}\PY{n}{v2}\PY{l+s+si}{\PYZcb{}}\PY{l+s+se}{\PYZbs{}n}\PY{l+s+si}{\PYZob{}}\PY{n}{v3}\PY{l+s+si}{\PYZcb{}}\PY{l+s+s2}{\PYZdq{}} \PY{k}{for} \PY{n}{v1}\PY{p}{,} \PY{n}{v2}\PY{p}{,} \PY{n}{v3} \PY{o+ow}{in}
          \PY{n+nb}{zip}\PY{p}{(}\PY{n}{group\PYZus{}names}\PY{p}{,}\PY{n}{group\PYZus{}counts}\PY{p}{,}\PY{n}{group\PYZus{}percentages}\PY{p}{)}\PY{p}{]}
\PY{n}{labels} \PY{o}{=} \PY{n}{np}\PY{o}{.}\PY{n}{asarray}\PY{p}{(}\PY{n}{labels}\PY{p}{)}\PY{o}{.}\PY{n}{reshape}\PY{p}{(}\PY{l+m+mi}{2}\PY{p}{,}\PY{l+m+mi}{2}\PY{p}{)}
\PY{n}{sns}\PY{o}{.}\PY{n}{heatmap}\PY{p}{(}\PY{n}{dt\PYZus{}matrix\PYZus{}val}\PY{p}{,} \PY{n}{annot}\PY{o}{=}\PY{n}{labels}\PY{p}{,} \PY{n}{fmt}\PY{o}{=}\PY{l+s+s1}{\PYZsq{}}\PY{l+s+s1}{\PYZsq{}}\PY{p}{,} \PY{n}{cmap}\PY{o}{=}\PY{l+s+s1}{\PYZsq{}}\PY{l+s+s1}{Blues}\PY{l+s+s1}{\PYZsq{}}\PY{p}{)}
\end{Verbatim}
\end{tcolorbox}

            \begin{tcolorbox}[breakable, size=fbox, boxrule=.5pt, pad at break*=1mm, opacityfill=0]
\prompt{Out}{outcolor}{42}{\boxspacing}
\begin{Verbatim}[commandchars=\\\{\}]
<matplotlib.axes.\_subplots.AxesSubplot at 0x7fe4eef31910>
\end{Verbatim}
\end{tcolorbox}
        
    \begin{center}
    \adjustimage{max size={0.9\linewidth}{0.9\paperheight}}{output_86_1.png}
    \end{center}
    { \hspace*{\fill} \\}
    
    \begin{tcolorbox}[breakable, size=fbox, boxrule=1pt, pad at break*=1mm,colback=cellbackground, colframe=cellborder]
\prompt{In}{incolor}{43}{\boxspacing}
\begin{Verbatim}[commandchars=\\\{\}]
\PY{c+c1}{\PYZsh{}Test}
\PY{n}{test\PYZus{}dt} \PY{o}{=} \PY{n+nb}{round}\PY{p}{(}\PY{n}{f1\PYZus{}score}\PY{p}{(}\PY{n}{test\PYZus{}target}\PY{p}{,} \PY{n}{gridsearchDT}\PY{o}{.}\PY{n}{predict}\PY{p}{(}\PY{n}{test\PYZus{}features}\PY{p}{)}\PY{p}{)}\PY{p}{,}\PY{l+m+mi}{6}\PY{p}{)}
\PY{n+nb}{print}\PY{p}{(}\PY{n}{metrics}\PY{o}{.}\PY{n}{classification\PYZus{}report}\PY{p}{(}\PY{n}{test\PYZus{}target}\PY{p}{,} \PY{n}{gridsearchDT}\PY{o}{.}\PY{n}{predict}\PY{p}{(}\PY{n}{test\PYZus{}features}\PY{p}{)}\PY{p}{)}\PY{p}{)}
\PY{n+nb}{print}\PY{p}{(}\PY{l+s+s2}{\PYZdq{}}\PY{l+s+s2}{Test score f1 : }\PY{l+s+s2}{\PYZdq{}}\PY{p}{,} \PY{n}{test\PYZus{}dt}\PY{p}{)}
\end{Verbatim}
\end{tcolorbox}

    \begin{Verbatim}[commandchars=\\\{\}]
              precision    recall  f1-score   support

           0       1.00      1.00      1.00    263288
           1       0.03      0.03      0.03       855

    accuracy                           0.99    264143
   macro avg       0.51      0.51      0.51    264143
weighted avg       0.99      0.99      0.99    264143

Test score f1 :  0.026906
    \end{Verbatim}

    \begin{tcolorbox}[breakable, size=fbox, boxrule=1pt, pad at break*=1mm,colback=cellbackground, colframe=cellborder]
\prompt{In}{incolor}{44}{\boxspacing}
\begin{Verbatim}[commandchars=\\\{\}]
\PY{c+c1}{\PYZsh{}Courbe ROC}
\PY{n}{fpr}\PY{p}{,} \PY{n}{tpr}\PY{p}{,} \PY{n}{\PYZus{}} \PY{o}{=} \PY{n}{roc\PYZus{}curve}\PY{p}{(}\PY{n}{test\PYZus{}target}\PY{p}{,} \PY{n}{gridsearchDT}\PY{o}{.}\PY{n}{predict}\PY{p}{(}\PY{n}{test\PYZus{}features}\PY{p}{)}\PY{p}{)}
\PY{n}{auc\PYZus{}dt} \PY{o}{=} \PY{n}{metrics}\PY{o}{.}\PY{n}{roc\PYZus{}auc\PYZus{}score}\PY{p}{(}\PY{n}{test\PYZus{}target}\PY{p}{,} \PY{n}{gridsearchDT}\PY{o}{.}\PY{n}{predict}\PY{p}{(}\PY{n}{test\PYZus{}features}\PY{p}{)}\PY{p}{)}
\PY{n}{plt}\PY{o}{.}\PY{n}{plot}\PY{p}{(}\PY{n}{fpr}\PY{p}{,}\PY{n}{tpr}\PY{p}{,}\PY{n}{label}\PY{o}{=}\PY{l+s+s2}{\PYZdq{}}\PY{l+s+s2}{data 1, auc=}\PY{l+s+s2}{\PYZdq{}}\PY{o}{+}\PY{n+nb}{str}\PY{p}{(}\PY{n}{auc\PYZus{}dt}\PY{p}{)}\PY{p}{)}
\PY{n}{plt}\PY{o}{.}\PY{n}{legend}\PY{p}{(}\PY{n}{loc}\PY{o}{=}\PY{l+m+mi}{4}\PY{p}{)}
\PY{n}{plt}\PY{o}{.}\PY{n}{show}\PY{p}{(}\PY{p}{)}
\end{Verbatim}
\end{tcolorbox}

    \begin{center}
    \adjustimage{max size={0.9\linewidth}{0.9\paperheight}}{output_88_0.png}
    \end{center}
    { \hspace*{\fill} \\}
    
    \begin{tcolorbox}[breakable, size=fbox, boxrule=1pt, pad at break*=1mm,colback=cellbackground, colframe=cellborder]
\prompt{In}{incolor}{164}{\boxspacing}
\begin{Verbatim}[commandchars=\\\{\}]
\PY{n}{test\PYZus{}result}\PY{p}{[}\PY{l+s+s1}{\PYZsq{}}\PY{l+s+s1}{DT}\PY{l+s+s1}{\PYZsq{}}\PY{p}{]} \PY{o}{=} \PY{n}{test\PYZus{}dt}
\PY{n}{valid\PYZus{}result}\PY{p}{[}\PY{l+s+s1}{\PYZsq{}}\PY{l+s+s1}{DT}\PY{l+s+s1}{\PYZsq{}}\PY{p}{]} \PY{o}{=} \PY{n}{f1\PYZus{}dt}
\PY{n}{auc\PYZus{}result}\PY{p}{[}\PY{l+s+s1}{\PYZsq{}}\PY{l+s+s1}{DT}\PY{l+s+s1}{\PYZsq{}}\PY{p}{]}\PY{o}{=}\PY{n+nb}{round}\PY{p}{(}\PY{n}{auc\PYZus{}dt}\PY{p}{,}\PY{l+m+mi}{3}\PY{p}{)}
\PY{n}{time\PYZus{}result}\PY{p}{[}\PY{l+s+s1}{\PYZsq{}}\PY{l+s+s1}{DT}\PY{l+s+s1}{\PYZsq{}}\PY{p}{]}\PY{o}{=}\PY{n+nb}{round}\PY{p}{(}\PY{n}{DT}\PY{p}{,}\PY{l+m+mi}{3}\PY{p}{)}
\PY{n+nb}{print}\PY{p}{(}\PY{n}{test\PYZus{}result}\PY{p}{)}
\PY{n+nb}{print}\PY{p}{(}\PY{n}{valid\PYZus{}result}\PY{p}{)}
\PY{n+nb}{print}\PY{p}{(}\PY{n}{auc\PYZus{}result}\PY{p}{)}
\PY{n+nb}{print}\PY{p}{(}\PY{n}{time\PYZus{}result}\PY{p}{)}
\end{Verbatim}
\end{tcolorbox}

    \begin{Verbatim}[commandchars=\\\{\}]
\{'KNN': 0.039632, 'DT': 0.080354\}
\{'KNN': 0.048721, 'DT': 0.030846\}
\{'KNN': 0.536, 'DT': 0.512\}
\{'KNN': 85.574, 'DT': 63.411\}
    \end{Verbatim}

    \hypertarget{analyse-discriminante-linuxe9aire}{%
\subsubsection{7.3 Analyse discriminante
linéaire}\label{analyse-discriminante-linuxe9aire}}

Nous avons décidé de présenter les résultats de l'analyse discriminante
linéaire. En effet c'est un modèle relativement simple à mettre en œuvre
dont les résultats sont facilement interprétable. De même, le temps de
calcul est plutôt rapide quel que soit la taille des données et est
stable sur des petits échantillons.

    \begin{tcolorbox}[breakable, size=fbox, boxrule=1pt, pad at break*=1mm,colback=cellbackground, colframe=cellborder]
\prompt{In}{incolor}{58}{\boxspacing}
\begin{Verbatim}[commandchars=\\\{\}]
\PY{c+c1}{\PYZsh{}Création de pipeline}
\PY{n}{pipelineADL} \PY{o}{=} \PY{n}{make\PYZus{}pipeline}\PY{p}{(}\PY{n}{StandardScaler}\PY{p}{(}\PY{p}{)}\PY{p}{,}\PY{n}{LinearDiscriminantAnalysis}\PY{p}{(}\PY{p}{)}\PY{p}{)}
\end{Verbatim}
\end{tcolorbox}

    Afin de savoir les paramètres inclus dans l'ADL nous allons utiliser la
fonction get\_params().keys()

    \begin{tcolorbox}[breakable, size=fbox, boxrule=1pt, pad at break*=1mm,colback=cellbackground, colframe=cellborder]
\prompt{In}{incolor}{59}{\boxspacing}
\begin{Verbatim}[commandchars=\\\{\}]
\PY{c+c1}{\PYZsh{}Liste des paramètres}
\PY{n+nb}{sorted}\PY{p}{(}\PY{n}{pipelineADL}\PY{o}{.}\PY{n}{get\PYZus{}params}\PY{p}{(}\PY{p}{)}\PY{o}{.}\PY{n}{keys}\PY{p}{(}\PY{p}{)}\PY{p}{)}
\end{Verbatim}
\end{tcolorbox}

            \begin{tcolorbox}[breakable, size=fbox, boxrule=.5pt, pad at break*=1mm, opacityfill=0]
\prompt{Out}{outcolor}{59}{\boxspacing}
\begin{Verbatim}[commandchars=\\\{\}]
['lineardiscriminantanalysis',
 'lineardiscriminantanalysis\_\_n\_components',
 'lineardiscriminantanalysis\_\_priors',
 'lineardiscriminantanalysis\_\_shrinkage',
 'lineardiscriminantanalysis\_\_solver',
 'lineardiscriminantanalysis\_\_store\_covariance',
 'lineardiscriminantanalysis\_\_tol',
 'memory',
 'standardscaler',
 'standardscaler\_\_copy',
 'standardscaler\_\_with\_mean',
 'standardscaler\_\_with\_std',
 'steps',
 'verbose']
\end{Verbatim}
\end{tcolorbox}
        
    On obtient la liste des paramètres du modèle, que l'on peut essayer dans
le gridsearch. Pour appliquer cette méthode, trois hyper-paramètres
peuvent être utilisés le « solver », le «shrinkage» et le ``tol''. Le
premier permet de changer la méthode de calcul et le second permet une
régularisation lorsque la dimensionnalité augmente.

    \begin{tcolorbox}[breakable, size=fbox, boxrule=1pt, pad at break*=1mm,colback=cellbackground, colframe=cellborder]
\prompt{In}{incolor}{60}{\boxspacing}
\begin{Verbatim}[commandchars=\\\{\}]
\PY{c+c1}{\PYZsh{}Paramètres à tester}
\PY{n}{paramsADL} \PY{o}{=} \PY{p}{\PYZob{}}\PY{l+s+s1}{\PYZsq{}}\PY{l+s+s1}{lineardiscriminantanalysis\PYZus{}\PYZus{}solver}\PY{l+s+s1}{\PYZsq{}}\PY{p}{:}\PY{p}{[}\PY{l+s+s1}{\PYZsq{}}\PY{l+s+s1}{svd}\PY{l+s+s1}{\PYZsq{}}\PY{p}{,} \PY{l+s+s1}{\PYZsq{}}\PY{l+s+s1}{lsqr}\PY{l+s+s1}{\PYZsq{}}\PY{p}{,}\PY{l+s+s1}{\PYZsq{}}\PY{l+s+s1}{eigen}\PY{l+s+s1}{\PYZsq{}}\PY{p}{]}\PY{p}{,}
             \PY{l+s+s1}{\PYZsq{}}\PY{l+s+s1}{lineardiscriminantanalysis\PYZus{}\PYZus{}shrinkage}\PY{l+s+s1}{\PYZsq{}}\PY{p}{:}\PY{p}{[}\PY{k+kc}{None}\PY{p}{,} \PY{l+s+s1}{\PYZsq{}}\PY{l+s+s1}{auto}\PY{l+s+s1}{\PYZsq{}}\PY{p}{]}\PY{p}{,}
             \PY{l+s+s1}{\PYZsq{}}\PY{l+s+s1}{lineardiscriminantanalysis\PYZus{}\PYZus{}tol}\PY{l+s+s1}{\PYZsq{}}\PY{p}{:}\PY{p}{[}\PY{l+m+mf}{0.0001}\PY{p}{,}\PY{l+m+mf}{0.0002}\PY{p}{,}\PY{l+m+mf}{0.0003}\PY{p}{]}\PY{p}{\PYZcb{}} \PY{c+c1}{\PYZsh{}Paramètres à tester}
\end{Verbatim}
\end{tcolorbox}

    \begin{tcolorbox}[breakable, size=fbox, boxrule=1pt, pad at break*=1mm,colback=cellbackground, colframe=cellborder]
\prompt{In}{incolor}{61}{\boxspacing}
\begin{Verbatim}[commandchars=\\\{\}]
\PY{n}{tscv} \PY{o}{=} \PY{n}{TimeSeriesSplit}\PY{p}{(}\PY{n}{n\PYZus{}splits}\PY{o}{=}\PY{l+m+mi}{5}\PY{p}{)}
\PY{n}{startADL} \PY{o}{=} \PY{n}{time}\PY{p}{(}\PY{p}{)}
\PY{c+c1}{\PYZsh{}instanciation \PYZhy{} recherche des hyperparametres optimaux}
\PY{n}{gridsearchADL} \PY{o}{=} \PY{n}{GridSearchCV}\PY{p}{(}\PY{n}{estimator}\PY{o}{=}\PY{n}{pipelineADL}\PY{p}{,}
                       \PY{n}{param\PYZus{}grid}\PY{o}{=}\PY{n}{paramsADL}\PY{p}{,}
                       \PY{n}{n\PYZus{}jobs}\PY{o}{=}\PY{o}{\PYZhy{}}\PY{l+m+mi}{1}\PY{p}{,}
                       \PY{n}{scoring}\PY{o}{=}\PY{l+s+s1}{\PYZsq{}}\PY{l+s+s1}{f1\PYZus{}micro}\PY{l+s+s1}{\PYZsq{}}\PY{p}{)}
\PY{n}{gridsearchADL}\PY{o}{.}\PY{n}{fit}\PY{p}{(}\PY{n}{x\PYZus{}train\PYZus{}res}\PY{p}{,} \PY{n}{y\PYZus{}train\PYZus{}res}\PY{p}{)}
\PY{n}{doneADL} \PY{o}{=} \PY{n}{time}\PY{p}{(}\PY{p}{)}
\PY{n}{ADL} \PY{o}{=} \PY{n+nb}{round}\PY{p}{(}\PY{n}{doneADL} \PY{o}{\PYZhy{}} \PY{n}{startADL}\PY{p}{)}\PY{o}{/}\PY{l+m+mi}{60}
\PY{n+nb}{print}\PY{p}{(}\PY{l+s+s2}{\PYZdq{}}\PY{l+s+s2}{Temps de calcul:}\PY{l+s+s2}{\PYZdq{}}\PY{p}{,}\PY{n}{ADL}\PY{p}{)}
\end{Verbatim}
\end{tcolorbox}

    \begin{Verbatim}[commandchars=\\\{\}]
Temps de calcul: 0.08499999999999999
    \end{Verbatim}

    On obtient le meilleur modèle:

    \begin{tcolorbox}[breakable, size=fbox, boxrule=1pt, pad at break*=1mm,colback=cellbackground, colframe=cellborder]
\prompt{In}{incolor}{62}{\boxspacing}
\begin{Verbatim}[commandchars=\\\{\}]
\PY{c+c1}{\PYZsh{}Meilleurs paramètres}
\PY{n+nb}{print}\PY{p}{(}\PY{l+s+s1}{\PYZsq{}}\PY{l+s+s1}{Best Penalty:}\PY{l+s+s1}{\PYZsq{}}\PY{p}{,} \PY{n}{gridsearchADL}\PY{o}{.}\PY{n}{best\PYZus{}estimator\PYZus{}}\PY{o}{.}\PY{n}{get\PYZus{}params}\PY{p}{(}\PY{p}{)}\PY{p}{[}\PY{l+s+s1}{\PYZsq{}}\PY{l+s+s1}{lineardiscriminantanalysis\PYZus{}\PYZus{}shrinkage}\PY{l+s+s1}{\PYZsq{}}\PY{p}{]}\PY{p}{)}
\PY{n+nb}{print}\PY{p}{(}\PY{l+s+s1}{\PYZsq{}}\PY{l+s+s1}{Best C:}\PY{l+s+s1}{\PYZsq{}}\PY{p}{,} \PY{n}{gridsearchADL}\PY{o}{.}\PY{n}{best\PYZus{}estimator\PYZus{}}\PY{o}{.}\PY{n}{get\PYZus{}params}\PY{p}{(}\PY{p}{)}\PY{p}{[}\PY{l+s+s1}{\PYZsq{}}\PY{l+s+s1}{lineardiscriminantanalysis\PYZus{}\PYZus{}tol}\PY{l+s+s1}{\PYZsq{}}\PY{p}{]}\PY{p}{)}
\PY{n+nb}{print}\PY{p}{(}\PY{l+s+s1}{\PYZsq{}}\PY{l+s+s1}{Best solver:}\PY{l+s+s1}{\PYZsq{}}\PY{p}{,} \PY{n}{gridsearchADL}\PY{o}{.}\PY{n}{best\PYZus{}estimator\PYZus{}}\PY{o}{.}\PY{n}{get\PYZus{}params}\PY{p}{(}\PY{p}{)}\PY{p}{[}\PY{l+s+s1}{\PYZsq{}}\PY{l+s+s1}{lineardiscriminantanalysis\PYZus{}\PYZus{}solver}\PY{l+s+s1}{\PYZsq{}}\PY{p}{]}\PY{p}{)}
\end{Verbatim}
\end{tcolorbox}

    \begin{Verbatim}[commandchars=\\\{\}]
Best Penalty: None
Best C: 0.0001
Best solver: svd
    \end{Verbatim}

    \hypertarget{validation}{%
\subparagraph{Validation :}\label{validation}}

    Les métriques d'evaluation :

    \begin{tcolorbox}[breakable, size=fbox, boxrule=1pt, pad at break*=1mm,colback=cellbackground, colframe=cellborder]
\prompt{In}{incolor}{64}{\boxspacing}
\begin{Verbatim}[commandchars=\\\{\}]
\PY{c+c1}{\PYZsh{}Validation}
\PY{n}{f1\PYZus{}ADL} \PY{o}{=} \PY{n+nb}{round}\PY{p}{(}\PY{n}{f1\PYZus{}score}\PY{p}{(}\PY{n}{y\PYZus{}val}\PY{p}{,} \PY{n}{gridsearchADL} \PY{o}{.}\PY{n}{predict}\PY{p}{(}\PY{n}{x\PYZus{}val}\PY{p}{)}\PY{p}{)}\PY{p}{,}\PY{l+m+mi}{6}\PY{p}{)}
\PY{n+nb}{print}\PY{p}{(}\PY{n}{metrics}\PY{o}{.}\PY{n}{classification\PYZus{}report}\PY{p}{(}\PY{n}{y\PYZus{}val}\PY{p}{,}\PY{n}{gridsearchADL} \PY{o}{.}\PY{n}{predict}\PY{p}{(}\PY{n}{x\PYZus{}val}\PY{p}{)}\PY{p}{)}\PY{p}{)}
\PY{n+nb}{print}\PY{p}{(}\PY{l+s+s2}{\PYZdq{}}\PY{l+s+s2}{Validation score f1 : }\PY{l+s+s2}{\PYZdq{}}\PY{p}{,} \PY{n}{f1\PYZus{}ADL}\PY{p}{)}
\end{Verbatim}
\end{tcolorbox}

    \begin{Verbatim}[commandchars=\\\{\}]
              precision    recall  f1-score   support

           0       1.00      0.88      0.94    288737
           1       0.02      0.57      0.03       948

    accuracy                           0.88    289685
   macro avg       0.51      0.72      0.48    289685
weighted avg       1.00      0.88      0.93    289685

Validation score f1 :  0.029688
    \end{Verbatim}

    \hypertarget{test}{%
\subparagraph{Test :}\label{test}}

    \begin{tcolorbox}[breakable, size=fbox, boxrule=1pt, pad at break*=1mm,colback=cellbackground, colframe=cellborder]
\prompt{In}{incolor}{65}{\boxspacing}
\begin{Verbatim}[commandchars=\\\{\}]
\PY{c+c1}{\PYZsh{}Test}
\PY{c+c1}{\PYZsh{}Matrice de confusion}
\PY{n}{ADL\PYZus{}matrix\PYZus{}test} \PY{o}{=} \PY{n}{confusion\PYZus{}matrix}\PY{p}{(}\PY{n}{test\PYZus{}target}\PY{p}{,} \PY{n}{gridsearchADL}\PY{o}{.}\PY{n}{predict}\PY{p}{(}\PY{n}{test\PYZus{}features}\PY{p}{)}\PY{p}{)}
\PY{n}{group\PYZus{}names} \PY{o}{=} \PY{p}{[}\PY{l+s+s1}{\PYZsq{}}\PY{l+s+s1}{True Neg}\PY{l+s+s1}{\PYZsq{}}\PY{p}{,}\PY{l+s+s1}{\PYZsq{}}\PY{l+s+s1}{False Pos}\PY{l+s+s1}{\PYZsq{}}\PY{p}{,}\PY{l+s+s1}{\PYZsq{}}\PY{l+s+s1}{False Neg}\PY{l+s+s1}{\PYZsq{}}\PY{p}{,}\PY{l+s+s1}{\PYZsq{}}\PY{l+s+s1}{True Pos}\PY{l+s+s1}{\PYZsq{}}\PY{p}{]}
\PY{n}{group\PYZus{}counts} \PY{o}{=} \PY{p}{[}\PY{l+s+s2}{\PYZdq{}}\PY{l+s+si}{\PYZob{}0:0.0f\PYZcb{}}\PY{l+s+s2}{\PYZdq{}}\PY{o}{.}\PY{n}{format}\PY{p}{(}\PY{n}{value}\PY{p}{)} \PY{k}{for} \PY{n}{value} \PY{o+ow}{in}
                \PY{n}{ADL\PYZus{}matrix\PYZus{}test}\PY{o}{.}\PY{n}{flatten}\PY{p}{(}\PY{p}{)}\PY{p}{]}
\PY{n}{group\PYZus{}percentages} \PY{o}{=} \PY{p}{[}\PY{l+s+s2}{\PYZdq{}}\PY{l+s+si}{\PYZob{}0:.2\PYZpc{}\PYZcb{}}\PY{l+s+s2}{\PYZdq{}}\PY{o}{.}\PY{n}{format}\PY{p}{(}\PY{n}{value}\PY{p}{)} \PY{k}{for} \PY{n}{value} \PY{o+ow}{in}
                     \PY{n}{ADL\PYZus{}matrix\PYZus{}test}\PY{o}{.}\PY{n}{flatten}\PY{p}{(}\PY{p}{)}\PY{o}{/}\PY{n}{np}\PY{o}{.}\PY{n}{sum}\PY{p}{(}\PY{n}{ADL\PYZus{}matrix\PYZus{}test}\PY{p}{)}\PY{p}{]}
\PY{n}{labels} \PY{o}{=} \PY{p}{[}\PY{l+s+sa}{f}\PY{l+s+s2}{\PYZdq{}}\PY{l+s+si}{\PYZob{}}\PY{n}{v1}\PY{l+s+si}{\PYZcb{}}\PY{l+s+se}{\PYZbs{}n}\PY{l+s+si}{\PYZob{}}\PY{n}{v2}\PY{l+s+si}{\PYZcb{}}\PY{l+s+se}{\PYZbs{}n}\PY{l+s+si}{\PYZob{}}\PY{n}{v3}\PY{l+s+si}{\PYZcb{}}\PY{l+s+s2}{\PYZdq{}} \PY{k}{for} \PY{n}{v1}\PY{p}{,} \PY{n}{v2}\PY{p}{,} \PY{n}{v3} \PY{o+ow}{in}
          \PY{n+nb}{zip}\PY{p}{(}\PY{n}{group\PYZus{}names}\PY{p}{,}\PY{n}{group\PYZus{}counts}\PY{p}{,}\PY{n}{group\PYZus{}percentages}\PY{p}{)}\PY{p}{]}
\PY{n}{labels} \PY{o}{=} \PY{n}{np}\PY{o}{.}\PY{n}{asarray}\PY{p}{(}\PY{n}{labels}\PY{p}{)}\PY{o}{.}\PY{n}{reshape}\PY{p}{(}\PY{l+m+mi}{2}\PY{p}{,}\PY{l+m+mi}{2}\PY{p}{)}
\PY{n}{sns}\PY{o}{.}\PY{n}{heatmap}\PY{p}{(}\PY{n}{ADL\PYZus{}matrix\PYZus{}test}\PY{p}{,} \PY{n}{annot}\PY{o}{=}\PY{n}{labels}\PY{p}{,} \PY{n}{fmt}\PY{o}{=}\PY{l+s+s1}{\PYZsq{}}\PY{l+s+s1}{\PYZsq{}}\PY{p}{,} \PY{n}{cmap}\PY{o}{=}\PY{l+s+s1}{\PYZsq{}}\PY{l+s+s1}{Blues}\PY{l+s+s1}{\PYZsq{}}\PY{p}{)}
\end{Verbatim}
\end{tcolorbox}

            \begin{tcolorbox}[breakable, size=fbox, boxrule=.5pt, pad at break*=1mm, opacityfill=0]
\prompt{Out}{outcolor}{65}{\boxspacing}
\begin{Verbatim}[commandchars=\\\{\}]
<matplotlib.axes.\_subplots.AxesSubplot at 0x7fe4ee82c340>
\end{Verbatim}
\end{tcolorbox}
        
    \begin{center}
    \adjustimage{max size={0.9\linewidth}{0.9\paperheight}}{output_103_1.png}
    \end{center}
    { \hspace*{\fill} \\}
    
    \begin{tcolorbox}[breakable, size=fbox, boxrule=1pt, pad at break*=1mm,colback=cellbackground, colframe=cellborder]
\prompt{In}{incolor}{68}{\boxspacing}
\begin{Verbatim}[commandchars=\\\{\}]
\PY{c+c1}{\PYZsh{}Test}
\PY{n}{test\PYZus{}ADL} \PY{o}{=} \PY{n+nb}{round}\PY{p}{(}\PY{n}{f1\PYZus{}score}\PY{p}{(}\PY{n}{test\PYZus{}target}\PY{p}{,}\PY{n}{gridsearchADL}\PY{o}{.}\PY{n}{predict}\PY{p}{(}\PY{n}{test\PYZus{}features}\PY{p}{)}\PY{p}{)}\PY{p}{,}\PY{l+m+mi}{6}\PY{p}{)}
\PY{n+nb}{print}\PY{p}{(}\PY{n}{metrics}\PY{o}{.}\PY{n}{classification\PYZus{}report}\PY{p}{(}\PY{n}{test\PYZus{}target}\PY{p}{,}\PY{n}{gridsearchADL}\PY{o}{.}\PY{n}{predict}\PY{p}{(}\PY{n}{test\PYZus{}features}\PY{p}{)}\PY{p}{)}\PY{p}{)}
\PY{n+nb}{print}\PY{p}{(}\PY{l+s+s2}{\PYZdq{}}\PY{l+s+s2}{Validation score f1 : }\PY{l+s+s2}{\PYZdq{}}\PY{p}{,} \PY{n}{test\PYZus{}ADL}\PY{p}{)}
\end{Verbatim}
\end{tcolorbox}

    \begin{Verbatim}[commandchars=\\\{\}]
              precision    recall  f1-score   support

           0       1.00      0.91      0.95    263288
           1       0.02      0.45      0.03       855

    accuracy                           0.91    264143
   macro avg       0.51      0.68      0.49    264143
weighted avg       0.99      0.91      0.95    264143

Validation score f1 :  0.031711
    \end{Verbatim}

    \begin{tcolorbox}[breakable, size=fbox, boxrule=1pt, pad at break*=1mm,colback=cellbackground, colframe=cellborder]
\prompt{In}{incolor}{69}{\boxspacing}
\begin{Verbatim}[commandchars=\\\{\}]
\PY{c+c1}{\PYZsh{}Courbe ROC}
\PY{n}{fpr}\PY{p}{,} \PY{n}{tpr}\PY{p}{,} \PY{n}{\PYZus{}} \PY{o}{=} \PY{n}{roc\PYZus{}curve}\PY{p}{(}\PY{n}{test\PYZus{}target}\PY{p}{,} \PY{n}{gridsearchADL}\PY{o}{.}\PY{n}{predict}\PY{p}{(}\PY{n}{test\PYZus{}features}\PY{p}{)}\PY{p}{)}
\PY{n}{auc\PYZus{}adl} \PY{o}{=} \PY{n}{metrics}\PY{o}{.}\PY{n}{roc\PYZus{}auc\PYZus{}score}\PY{p}{(}\PY{n}{test\PYZus{}target}\PY{p}{,}\PY{n}{gridsearchADL}\PY{o}{.}\PY{n}{predict}\PY{p}{(}\PY{n}{test\PYZus{}features}\PY{p}{)}\PY{p}{)}
\PY{n}{plt}\PY{o}{.}\PY{n}{plot}\PY{p}{(}\PY{n}{fpr}\PY{p}{,}\PY{n}{tpr}\PY{p}{,}\PY{n}{label}\PY{o}{=}\PY{l+s+s2}{\PYZdq{}}\PY{l+s+s2}{data 1, auc=}\PY{l+s+s2}{\PYZdq{}}\PY{o}{+}\PY{n+nb}{str}\PY{p}{(}\PY{n}{auc\PYZus{}adl}\PY{p}{)}\PY{p}{)}
\PY{n}{plt}\PY{o}{.}\PY{n}{legend}\PY{p}{(}\PY{n}{loc}\PY{o}{=}\PY{l+m+mi}{4}\PY{p}{)}
\PY{n}{plt}\PY{o}{.}\PY{n}{show}\PY{p}{(}\PY{p}{)}
\end{Verbatim}
\end{tcolorbox}

    \begin{center}
    \adjustimage{max size={0.9\linewidth}{0.9\paperheight}}{output_105_0.png}
    \end{center}
    { \hspace*{\fill} \\}
    
    Le F1 measure n'est pas très important mais l'AIC est beaucoup élevé.
D'après, le modèle a bien prédit 384 cas frauds sur 855 cas.

    \begin{tcolorbox}[breakable, size=fbox, boxrule=1pt, pad at break*=1mm,colback=cellbackground, colframe=cellborder]
\prompt{In}{incolor}{165}{\boxspacing}
\begin{Verbatim}[commandchars=\\\{\}]
\PY{c+c1}{\PYZsh{}Ajout dans le dico}
\PY{n}{test\PYZus{}result}\PY{p}{[}\PY{l+s+s1}{\PYZsq{}}\PY{l+s+s1}{ADL}\PY{l+s+s1}{\PYZsq{}}\PY{p}{]} \PY{o}{=} \PY{n}{test\PYZus{}ADL}
\PY{n}{valid\PYZus{}result}\PY{p}{[}\PY{l+s+s1}{\PYZsq{}}\PY{l+s+s1}{ADL}\PY{l+s+s1}{\PYZsq{}}\PY{p}{]} \PY{o}{=} \PY{n}{f1\PYZus{}ADL}
\PY{n}{auc\PYZus{}result}\PY{p}{[}\PY{l+s+s1}{\PYZsq{}}\PY{l+s+s1}{ADL}\PY{l+s+s1}{\PYZsq{}}\PY{p}{]} \PY{o}{=} \PY{n+nb}{round}\PY{p}{(}\PY{n}{auc\PYZus{}adl}\PY{p}{,}\PY{l+m+mi}{3}\PY{p}{)}
\PY{n}{time\PYZus{}result}\PY{p}{[}\PY{l+s+s1}{\PYZsq{}}\PY{l+s+s1}{ADL}\PY{l+s+s1}{\PYZsq{}}\PY{p}{]} \PY{o}{=} \PY{n+nb}{round}\PY{p}{(}\PY{n}{ADL}\PY{p}{,}\PY{l+m+mi}{3}\PY{p}{)}
\PY{n+nb}{print}\PY{p}{(}\PY{n}{test\PYZus{}result}\PY{p}{)}
\PY{n+nb}{print}\PY{p}{(}\PY{n}{valid\PYZus{}result}\PY{p}{)}
\PY{n+nb}{print}\PY{p}{(}\PY{n}{auc\PYZus{}result}\PY{p}{)}
\PY{n+nb}{print}\PY{p}{(}\PY{n}{time\PYZus{}result}\PY{p}{)}
\end{Verbatim}
\end{tcolorbox}

    \begin{Verbatim}[commandchars=\\\{\}]
\{'KNN': 0.039632, 'DT': 0.080354, 'ADL': 0.031711\}
\{'KNN': 0.048721, 'DT': 0.030846, 'ADL': 0.029688\}
\{'KNN': 0.536, 'DT': 0.512, 'ADL': 0.681\}
\{'KNN': 85.574, 'DT': 63.411, 'ADL': 5.1\}
    \end{Verbatim}

    Le F1 maximum que l'on obtient avec cette algorithme est de 0.02 en
utilisant les paramètres au-dessus. Le modèle ne semble pas adapter à
notre problème.

    \hypertarget{moduxe8le-ensembliste-de-svm}{%
\subsubsection{7.4 Modèle ensembliste de
SVM}\label{moduxe8le-ensembliste-de-svm}}

Nous avons voulu essayer un SVM, tout d'abord seul et un modèle
combinant plusieurs SVM. Ca n'a pas été concluant, nous n'avons pas
réussi à faire tourné cet algorithme le temps d'execution est trop long.
Les SVM ne sont pas adapté lorsque la base de données est trop
importante.

    \begin{tcolorbox}[breakable, size=fbox, boxrule=1pt, pad at break*=1mm,colback=cellbackground, colframe=cellborder]
\prompt{In}{incolor}{ }{\boxspacing}
\begin{Verbatim}[commandchars=\\\{\}]
\PY{k+kn}{from} \PY{n+nn}{sklearn}\PY{n+nn}{.}\PY{n+nn}{svm} \PY{k+kn}{import} \PY{n}{SVC}
\PY{n}{pipeSVM} \PY{o}{=} \PY{n}{make\PYZus{}pipeline}\PY{p}{(}\PY{n}{StandardScaler}\PY{p}{(}\PY{p}{)}\PY{p}{,} \PY{n}{SVC}\PY{p}{(}\PY{n}{gamma}\PY{o}{=}\PY{l+s+s1}{\PYZsq{}}\PY{l+s+s1}{auto}\PY{l+s+s1}{\PYZsq{}}\PY{p}{)}\PY{p}{)}
\PY{n}{pipeSVM}\PY{o}{.}\PY{n}{fit}\PY{p}{(}\PY{n}{x\PYZus{}train\PYZus{}res}\PY{p}{,} \PY{n}{y\PYZus{}train\PYZus{}res}\PY{p}{)} 
\PY{n+nb}{print} \PY{p}{(}\PY{l+s+s1}{\PYZsq{}}\PY{l+s+s1}{Validation Results}\PY{l+s+s1}{\PYZsq{}}\PY{p}{)}
\PY{n+nb}{print} \PY{p}{(}\PY{n}{f1\PYZus{}score}\PY{p}{(}\PY{n}{y\PYZus{}val}\PY{p}{,} \PY{n}{pipeSVM}\PY{o}{.}\PY{n}{predict}\PY{p}{(}\PY{n}{x\PYZus{}val}\PY{p}{)}\PY{p}{)}\PY{p}{)}
\PY{n+nb}{print} \PY{p}{(}\PY{l+s+s1}{\PYZsq{}}\PY{l+s+se}{\PYZbs{}n}\PY{l+s+s1}{Test Results}\PY{l+s+s1}{\PYZsq{}}\PY{p}{)}
\PY{n+nb}{print} \PY{p}{(}\PY{n}{f1\PYZus{}score}\PY{p}{(}\PY{n}{test\PYZus{}target}\PY{p}{,} \PY{n}{pipeSVM}\PY{o}{.}\PY{n}{predict}\PY{p}{(}\PY{n}{test\PYZus{}features}\PY{p}{)}\PY{p}{)}\PY{p}{)}
\end{Verbatim}
\end{tcolorbox}

    \begin{tcolorbox}[breakable, size=fbox, boxrule=1pt, pad at break*=1mm,colback=cellbackground, colframe=cellborder]
\prompt{In}{incolor}{ }{\boxspacing}
\begin{Verbatim}[commandchars=\\\{\}]
\PY{c+c1}{\PYZsh{}bagging svm}
\PY{k+kn}{from} \PY{n+nn}{sklearn}\PY{n+nn}{.}\PY{n+nn}{svm} \PY{k+kn}{import} \PY{n}{SVC}
\PY{k+kn}{from} \PY{n+nn}{sklearn}\PY{n+nn}{.}\PY{n+nn}{ensemble} \PY{k+kn}{import} \PY{n}{BaggingClassifier}
\PY{n}{bagging} \PY{o}{=} \PY{n}{make\PYZus{}pipeline}\PY{p}{(}\PY{n}{StandardScaler}\PY{p}{(}\PY{p}{)}\PY{p}{,} \PY{n}{BaggingClassifier}\PY{p}{(}\PY{n}{SVC}\PY{p}{(}\PY{p}{)}\PY{p}{,} \PY{n}{max\PYZus{}samples}\PY{o}{=}\PY{l+m+mf}{0.5}\PY{p}{,} \PY{n}{max\PYZus{}features}\PY{o}{=}\PY{l+m+mf}{0.5}\PY{p}{)}\PY{p}{)}
\PY{n}{bagging}\PY{o}{.}\PY{n}{fit}\PY{p}{(}\PY{n}{x\PYZus{}train\PYZus{}res}\PY{p}{,} \PY{n}{y\PYZus{}train\PYZus{}res}\PY{p}{)}
\PY{n+nb}{print} \PY{p}{(}\PY{l+s+s1}{\PYZsq{}}\PY{l+s+s1}{Validation Results}\PY{l+s+s1}{\PYZsq{}}\PY{p}{)}
\PY{n+nb}{print} \PY{p}{(}\PY{n}{f1\PYZus{}score}\PY{p}{(}\PY{n}{y\PYZus{}val}\PY{p}{,} \PY{n}{bagging}\PY{o}{.}\PY{n}{predict}\PY{p}{(}\PY{n}{x\PYZus{}val}\PY{p}{)}\PY{p}{)}\PY{p}{)}
\end{Verbatim}
\end{tcolorbox}

    \hypertarget{xgboost}{%
\subsubsection{7.5 XGboost}\label{xgboost}}

    \begin{tcolorbox}[breakable, size=fbox, boxrule=1pt, pad at break*=1mm,colback=cellbackground, colframe=cellborder]
\prompt{In}{incolor}{91}{\boxspacing}
\begin{Verbatim}[commandchars=\\\{\}]
\PY{c+c1}{\PYZsh{}xgboost}
\PY{c+c1}{\PYZsh{} avec cross\PYZus{}val\PYZus{}score}
\PY{n}{startboost} \PY{o}{=} \PY{n}{time}\PY{p}{(}\PY{p}{)}
\PY{n}{xg\PYZus{}clas} \PY{o}{=} \PY{n}{make\PYZus{}pipeline}\PY{p}{(}\PY{n}{StandardScaler}\PY{p}{(}\PY{p}{)}\PY{p}{,}\PY{n}{xgb}\PY{o}{.}\PY{n}{XGBClassifier}\PY{p}{(}\PY{n}{objective}\PY{o}{=}\PY{l+s+s1}{\PYZsq{}}\PY{l+s+s1}{binary:logistic}\PY{l+s+s1}{\PYZsq{}}\PY{p}{,} \PY{n}{colsample\PYZus{}bytree} \PY{o}{=} \PY{l+m+mf}{0.3}\PY{p}{,} \PY{n}{learning\PYZus{}rate} \PY{o}{=} \PY{l+m+mf}{0.1}\PY{p}{,}\PY{n}{max\PYZus{}depth} \PY{o}{=} \PY{l+m+mi}{5}\PY{p}{,}  \PY{n}{alpha} \PY{o}{=} \PY{l+m+mi}{10}\PY{p}{,} \PY{n}{n\PYZus{}estimators} \PY{o}{=} \PY{l+m+mi}{10}\PY{p}{)}\PY{p}{)}
\PY{c+c1}{\PYZsh{}preds = xg\PYZus{}clas.predict(x\PYZus{}val)}
\PY{n}{tscv} \PY{o}{=} \PY{n}{TimeSeriesSplit}\PY{p}{(}\PY{n}{n\PYZus{}splits}\PY{o}{=}\PY{l+m+mi}{5}\PY{p}{)}
\PY{n}{results} \PY{o}{=} \PY{n}{cross\PYZus{}val\PYZus{}score}\PY{p}{(}\PY{n}{xg\PYZus{}clas}\PY{p}{,}\PY{n}{x\PYZus{}train\PYZus{}res}\PY{p}{,}\PY{n}{y\PYZus{}train\PYZus{}res}\PY{p}{,} \PY{n}{cv}\PY{o}{=}\PY{n}{tscv}\PY{p}{,} \PY{n}{scoring}\PY{o}{=}\PY{l+s+s1}{\PYZsq{}}\PY{l+s+s1}{f1\PYZus{}micro}\PY{l+s+s1}{\PYZsq{}}\PY{p}{)}
\PY{c+c1}{\PYZsh{}print (\PYZsq{}Validation croisé\PYZsq{})}
\PY{c+c1}{\PYZsh{}print(\PYZdq{}resultat validation croisé:\PYZdq{}, results)}
\PY{c+c1}{\PYZsh{}print(\PYZdq{}f1 mesure: \PYZpc{}.2f\PYZpc{}\PYZpc{} \PYZdq{} \PYZpc{} (results.mean()*100))}
\PY{n}{xg\PYZus{}clas}\PY{o}{.}\PY{n}{fit}\PY{p}{(}\PY{n}{x\PYZus{}train\PYZus{}res}\PY{p}{,}\PY{n}{y\PYZus{}train\PYZus{}res}\PY{p}{)}
\PY{n}{doneboost} \PY{o}{=} \PY{n}{time}\PY{p}{(}\PY{p}{)}
\PY{n}{xg} \PY{o}{=} \PY{n+nb}{round}\PY{p}{(}\PY{n}{doneboost} \PY{o}{\PYZhy{}} \PY{n}{startboost}\PY{p}{,}\PY{l+m+mi}{3}\PY{p}{)}\PY{o}{/}\PY{l+m+mi}{60}
\PY{n+nb}{print}\PY{p}{(}\PY{l+s+s2}{\PYZdq{}}\PY{l+s+s2}{Temps de calcul:}\PY{l+s+s2}{\PYZdq{}}\PY{p}{,}\PY{n}{xg}\PY{p}{)}
\end{Verbatim}
\end{tcolorbox}

    \begin{Verbatim}[commandchars=\\\{\}]
/Users/hoangkhanhle/opt/anaconda3/lib/python3.8/site-
packages/xgboost/sklearn.py:888: UserWarning: The use of label encoder in
XGBClassifier is deprecated and will be removed in a future release. To remove
this warning, do the following: 1) Pass option use\_label\_encoder=False when
constructing XGBClassifier object; and 2) Encode your labels (y) as integers
starting with 0, i.e. 0, 1, 2, {\ldots}, [num\_class - 1].
  warnings.warn(label\_encoder\_deprecation\_msg, UserWarning)
    \end{Verbatim}

    \begin{Verbatim}[commandchars=\\\{\}]
[18:36:51] WARNING: /Users/travis/build/dmlc/xgboost/src/learner.cc:1061:
Starting in XGBoost 1.3.0, the default evaluation metric used with the objective
'binary:logistic' was changed from 'error' to 'logloss'. Explicitly set
eval\_metric if you'd like to restore the old behavior.
    \end{Verbatim}

    \begin{Verbatim}[commandchars=\\\{\}]
/Users/hoangkhanhle/opt/anaconda3/lib/python3.8/site-
packages/xgboost/sklearn.py:888: UserWarning: The use of label encoder in
XGBClassifier is deprecated and will be removed in a future release. To remove
this warning, do the following: 1) Pass option use\_label\_encoder=False when
constructing XGBClassifier object; and 2) Encode your labels (y) as integers
starting with 0, i.e. 0, 1, 2, {\ldots}, [num\_class - 1].
  warnings.warn(label\_encoder\_deprecation\_msg, UserWarning)
    \end{Verbatim}

    \begin{Verbatim}[commandchars=\\\{\}]
[18:36:53] WARNING: /Users/travis/build/dmlc/xgboost/src/learner.cc:1061:
Starting in XGBoost 1.3.0, the default evaluation metric used with the objective
'binary:logistic' was changed from 'error' to 'logloss'. Explicitly set
eval\_metric if you'd like to restore the old behavior.
    \end{Verbatim}

    \begin{Verbatim}[commandchars=\\\{\}]
/Users/hoangkhanhle/opt/anaconda3/lib/python3.8/site-
packages/xgboost/sklearn.py:888: UserWarning: The use of label encoder in
XGBClassifier is deprecated and will be removed in a future release. To remove
this warning, do the following: 1) Pass option use\_label\_encoder=False when
constructing XGBClassifier object; and 2) Encode your labels (y) as integers
starting with 0, i.e. 0, 1, 2, {\ldots}, [num\_class - 1].
  warnings.warn(label\_encoder\_deprecation\_msg, UserWarning)
    \end{Verbatim}

    \begin{Verbatim}[commandchars=\\\{\}]
[18:36:57] WARNING: /Users/travis/build/dmlc/xgboost/src/learner.cc:1061:
Starting in XGBoost 1.3.0, the default evaluation metric used with the objective
'binary:logistic' was changed from 'error' to 'logloss'. Explicitly set
eval\_metric if you'd like to restore the old behavior.
    \end{Verbatim}

    \begin{Verbatim}[commandchars=\\\{\}]
/Users/hoangkhanhle/opt/anaconda3/lib/python3.8/site-
packages/xgboost/sklearn.py:888: UserWarning: The use of label encoder in
XGBClassifier is deprecated and will be removed in a future release. To remove
this warning, do the following: 1) Pass option use\_label\_encoder=False when
constructing XGBClassifier object; and 2) Encode your labels (y) as integers
starting with 0, i.e. 0, 1, 2, {\ldots}, [num\_class - 1].
  warnings.warn(label\_encoder\_deprecation\_msg, UserWarning)
    \end{Verbatim}

    \begin{Verbatim}[commandchars=\\\{\}]
[18:37:03] WARNING: /Users/travis/build/dmlc/xgboost/src/learner.cc:1061:
Starting in XGBoost 1.3.0, the default evaluation metric used with the objective
'binary:logistic' was changed from 'error' to 'logloss'. Explicitly set
eval\_metric if you'd like to restore the old behavior.
    \end{Verbatim}

    \begin{Verbatim}[commandchars=\\\{\}]
/Users/hoangkhanhle/opt/anaconda3/lib/python3.8/site-
packages/xgboost/sklearn.py:888: UserWarning: The use of label encoder in
XGBClassifier is deprecated and will be removed in a future release. To remove
this warning, do the following: 1) Pass option use\_label\_encoder=False when
constructing XGBClassifier object; and 2) Encode your labels (y) as integers
starting with 0, i.e. 0, 1, 2, {\ldots}, [num\_class - 1].
  warnings.warn(label\_encoder\_deprecation\_msg, UserWarning)
    \end{Verbatim}

    \begin{Verbatim}[commandchars=\\\{\}]
[18:37:14] WARNING: /Users/travis/build/dmlc/xgboost/src/learner.cc:1061:
Starting in XGBoost 1.3.0, the default evaluation metric used with the objective
'binary:logistic' was changed from 'error' to 'logloss'. Explicitly set
eval\_metric if you'd like to restore the old behavior.
    \end{Verbatim}

    \begin{Verbatim}[commandchars=\\\{\}]
/Users/hoangkhanhle/opt/anaconda3/lib/python3.8/site-
packages/xgboost/sklearn.py:888: UserWarning: The use of label encoder in
XGBClassifier is deprecated and will be removed in a future release. To remove
this warning, do the following: 1) Pass option use\_label\_encoder=False when
constructing XGBClassifier object; and 2) Encode your labels (y) as integers
starting with 0, i.e. 0, 1, 2, {\ldots}, [num\_class - 1].
  warnings.warn(label\_encoder\_deprecation\_msg, UserWarning)
    \end{Verbatim}

    \begin{Verbatim}[commandchars=\\\{\}]
[18:37:27] WARNING: /Users/travis/build/dmlc/xgboost/src/learner.cc:1061:
Starting in XGBoost 1.3.0, the default evaluation metric used with the objective
'binary:logistic' was changed from 'error' to 'logloss'. Explicitly set
eval\_metric if you'd like to restore the old behavior.
Temps de calcul: 0.7562666666666666
    \end{Verbatim}

    \hypertarget{validation}{%
\subparagraph{Validation :}\label{validation}}

    \begin{tcolorbox}[breakable, size=fbox, boxrule=1pt, pad at break*=1mm,colback=cellbackground, colframe=cellborder]
\prompt{In}{incolor}{51}{\boxspacing}
\begin{Verbatim}[commandchars=\\\{\}]
\PY{c+c1}{\PYZsh{}Validation}

\PY{n}{val\PYZus{}XGboost} \PY{o}{=} \PY{n+nb}{round}\PY{p}{(}\PY{n}{f1\PYZus{}score}\PY{p}{(}\PY{n}{y\PYZus{}val}\PY{p}{,} \PY{n}{xg\PYZus{}clas}\PY{o}{.}\PY{n}{predict}\PY{p}{(}\PY{n}{x\PYZus{}val}\PY{p}{)}\PY{p}{)}\PY{p}{,}\PY{l+m+mi}{6}\PY{p}{)}
\PY{n+nb}{print}\PY{p}{(}\PY{n}{metrics}\PY{o}{.}\PY{n}{classification\PYZus{}report}\PY{p}{(}\PY{n}{y\PYZus{}val}\PY{p}{,} \PY{n}{xg\PYZus{}clas}\PY{o}{.}\PY{n}{predict}\PY{p}{(}\PY{n}{x\PYZus{}val}\PY{p}{)}\PY{p}{)}\PY{p}{)}
\PY{n+nb}{print}\PY{p}{(}\PY{l+s+s2}{\PYZdq{}}\PY{l+s+s2}{Validation score f1 : }\PY{l+s+s2}{\PYZdq{}}\PY{p}{,} \PY{n}{val\PYZus{}XGboost}\PY{p}{)}
\end{Verbatim}
\end{tcolorbox}

    \begin{Verbatim}[commandchars=\\\{\}]
              precision    recall  f1-score   support

           0       1.00      0.98      0.99    288737
           1       0.05      0.29      0.09       948

    accuracy                           0.98    289685
   macro avg       0.52      0.64      0.54    289685
weighted avg       0.99      0.98      0.99    289685

Validation score f1 :  0.085355
    \end{Verbatim}

    \hypertarget{test}{%
\subparagraph{Test :}\label{test}}

    \begin{tcolorbox}[breakable, size=fbox, boxrule=1pt, pad at break*=1mm,colback=cellbackground, colframe=cellborder]
\prompt{In}{incolor}{52}{\boxspacing}
\begin{Verbatim}[commandchars=\\\{\}]
\PY{c+c1}{\PYZsh{}Matrice de confusion}
\PY{n}{XG\PYZus{}matrix\PYZus{}test} \PY{o}{=} \PY{n}{confusion\PYZus{}matrix}\PY{p}{(}\PY{n}{test\PYZus{}target}\PY{p}{,} \PY{n}{xg\PYZus{}clas}\PY{o}{.}\PY{n}{predict}\PY{p}{(}\PY{n}{test\PYZus{}features}\PY{p}{)}\PY{p}{)}
\PY{n}{group\PYZus{}names} \PY{o}{=} \PY{p}{[}\PY{l+s+s1}{\PYZsq{}}\PY{l+s+s1}{True Neg}\PY{l+s+s1}{\PYZsq{}}\PY{p}{,}\PY{l+s+s1}{\PYZsq{}}\PY{l+s+s1}{False Pos}\PY{l+s+s1}{\PYZsq{}}\PY{p}{,}\PY{l+s+s1}{\PYZsq{}}\PY{l+s+s1}{False Neg}\PY{l+s+s1}{\PYZsq{}}\PY{p}{,}\PY{l+s+s1}{\PYZsq{}}\PY{l+s+s1}{True Pos}\PY{l+s+s1}{\PYZsq{}}\PY{p}{]}
\PY{n}{group\PYZus{}counts} \PY{o}{=} \PY{p}{[}\PY{l+s+s2}{\PYZdq{}}\PY{l+s+si}{\PYZob{}0:0.0f\PYZcb{}}\PY{l+s+s2}{\PYZdq{}}\PY{o}{.}\PY{n}{format}\PY{p}{(}\PY{n}{value}\PY{p}{)} \PY{k}{for} \PY{n}{value} \PY{o+ow}{in}
                \PY{n}{XG\PYZus{}matrix\PYZus{}test}\PY{o}{.}\PY{n}{flatten}\PY{p}{(}\PY{p}{)}\PY{p}{]}
\PY{n}{group\PYZus{}percentages} \PY{o}{=} \PY{p}{[}\PY{l+s+s2}{\PYZdq{}}\PY{l+s+si}{\PYZob{}0:.2\PYZpc{}\PYZcb{}}\PY{l+s+s2}{\PYZdq{}}\PY{o}{.}\PY{n}{format}\PY{p}{(}\PY{n}{value}\PY{p}{)} \PY{k}{for} \PY{n}{value} \PY{o+ow}{in}
                     \PY{n}{XG\PYZus{}matrix\PYZus{}test}\PY{o}{.}\PY{n}{flatten}\PY{p}{(}\PY{p}{)}\PY{o}{/}\PY{n}{np}\PY{o}{.}\PY{n}{sum}\PY{p}{(}\PY{n}{XG\PYZus{}matrix\PYZus{}test}\PY{p}{)}\PY{p}{]}
\PY{n}{labels} \PY{o}{=} \PY{p}{[}\PY{l+s+sa}{f}\PY{l+s+s2}{\PYZdq{}}\PY{l+s+si}{\PYZob{}}\PY{n}{v1}\PY{l+s+si}{\PYZcb{}}\PY{l+s+se}{\PYZbs{}n}\PY{l+s+si}{\PYZob{}}\PY{n}{v2}\PY{l+s+si}{\PYZcb{}}\PY{l+s+se}{\PYZbs{}n}\PY{l+s+si}{\PYZob{}}\PY{n}{v3}\PY{l+s+si}{\PYZcb{}}\PY{l+s+s2}{\PYZdq{}} \PY{k}{for} \PY{n}{v1}\PY{p}{,} \PY{n}{v2}\PY{p}{,} \PY{n}{v3} \PY{o+ow}{in}
          \PY{n+nb}{zip}\PY{p}{(}\PY{n}{group\PYZus{}names}\PY{p}{,}\PY{n}{group\PYZus{}counts}\PY{p}{,}\PY{n}{group\PYZus{}percentages}\PY{p}{)}\PY{p}{]}
\PY{n}{labels} \PY{o}{=} \PY{n}{np}\PY{o}{.}\PY{n}{asarray}\PY{p}{(}\PY{n}{labels}\PY{p}{)}\PY{o}{.}\PY{n}{reshape}\PY{p}{(}\PY{l+m+mi}{2}\PY{p}{,}\PY{l+m+mi}{2}\PY{p}{)}
\PY{n}{sns}\PY{o}{.}\PY{n}{heatmap}\PY{p}{(}\PY{n}{XG\PYZus{}matrix\PYZus{}test}\PY{p}{,} \PY{n}{annot}\PY{o}{=}\PY{n}{labels}\PY{p}{,} \PY{n}{fmt}\PY{o}{=}\PY{l+s+s1}{\PYZsq{}}\PY{l+s+s1}{\PYZsq{}}\PY{p}{,} \PY{n}{cmap}\PY{o}{=}\PY{l+s+s1}{\PYZsq{}}\PY{l+s+s1}{Blues}\PY{l+s+s1}{\PYZsq{}}\PY{p}{)}
\end{Verbatim}
\end{tcolorbox}

            \begin{tcolorbox}[breakable, size=fbox, boxrule=.5pt, pad at break*=1mm, opacityfill=0]
\prompt{Out}{outcolor}{52}{\boxspacing}
\begin{Verbatim}[commandchars=\\\{\}]
<matplotlib.axes.\_subplots.AxesSubplot at 0x7fe511fbc760>
\end{Verbatim}
\end{tcolorbox}
        
    \begin{center}
    \adjustimage{max size={0.9\linewidth}{0.9\paperheight}}{output_117_1.png}
    \end{center}
    { \hspace*{\fill} \\}
    
    \begin{tcolorbox}[breakable, size=fbox, boxrule=1pt, pad at break*=1mm,colback=cellbackground, colframe=cellborder]
\prompt{In}{incolor}{57}{\boxspacing}
\begin{Verbatim}[commandchars=\\\{\}]
\PY{c+c1}{\PYZsh{}Test}
\PY{n}{test\PYZus{}xgboost} \PY{o}{=} \PY{n+nb}{round}\PY{p}{(}\PY{n}{f1\PYZus{}score}\PY{p}{(}\PY{n}{test\PYZus{}target}\PY{p}{,} \PY{n}{xg\PYZus{}clas}\PY{o}{.}\PY{n}{predict}\PY{p}{(}\PY{n}{test\PYZus{}features}\PY{p}{)}\PY{p}{)}\PY{p}{,}\PY{l+m+mi}{6}\PY{p}{)}
\PY{n+nb}{print}\PY{p}{(}\PY{n}{metrics}\PY{o}{.}\PY{n}{classification\PYZus{}report}\PY{p}{(}\PY{n}{test\PYZus{}target}\PY{p}{,} \PY{n}{xg\PYZus{}clas}\PY{o}{.}\PY{n}{predict}\PY{p}{(}\PY{n}{test\PYZus{}features}\PY{p}{)}\PY{p}{)}\PY{p}{)}
\PY{n+nb}{print}\PY{p}{(}\PY{l+s+s2}{\PYZdq{}}\PY{l+s+s2}{Test score f1 : }\PY{l+s+s2}{\PYZdq{}}\PY{p}{,} \PY{n}{test\PYZus{}XG}\PY{p}{)}
\end{Verbatim}
\end{tcolorbox}

    \begin{Verbatim}[commandchars=\\\{\}]
              precision    recall  f1-score   support

           0       1.00      0.99      0.99    263288
           1       0.05      0.23      0.08       855

    accuracy                           0.98    264143
   macro avg       0.52      0.61      0.54    264143
weighted avg       0.99      0.98      0.99    264143

Test score f1 :  0.080354
    \end{Verbatim}

    \begin{tcolorbox}[breakable, size=fbox, boxrule=1pt, pad at break*=1mm,colback=cellbackground, colframe=cellborder]
\prompt{In}{incolor}{78}{\boxspacing}
\begin{Verbatim}[commandchars=\\\{\}]
\PY{c+c1}{\PYZsh{}Courbe ROC}
\PY{n}{fpr}\PY{p}{,} \PY{n}{tpr}\PY{p}{,} \PY{n}{\PYZus{}} \PY{o}{=} \PY{n}{roc\PYZus{}curve}\PY{p}{(}\PY{n}{test\PYZus{}target}\PY{p}{,} \PY{n}{xg\PYZus{}clas}\PY{o}{.}\PY{n}{predict}\PY{p}{(}\PY{n}{test\PYZus{}features}\PY{p}{)}\PY{p}{)}
\PY{n}{auc\PYZus{}xg}\PY{o}{=} \PY{n}{metrics}\PY{o}{.}\PY{n}{roc\PYZus{}auc\PYZus{}score}\PY{p}{(}\PY{n}{test\PYZus{}target}\PY{p}{,}\PY{n}{xg\PYZus{}clas}\PY{o}{.}\PY{n}{predict}\PY{p}{(}\PY{n}{test\PYZus{}features}\PY{p}{)}\PY{p}{)}
\PY{n}{plt}\PY{o}{.}\PY{n}{plot}\PY{p}{(}\PY{n}{fpr}\PY{p}{,}\PY{n}{tpr}\PY{p}{,}\PY{n}{label}\PY{o}{=}\PY{l+s+s2}{\PYZdq{}}\PY{l+s+s2}{data 1, auc=}\PY{l+s+s2}{\PYZdq{}}\PY{o}{+}\PY{n+nb}{str}\PY{p}{(}\PY{n}{auc\PYZus{}xg}\PY{p}{)}\PY{p}{)}
\PY{n}{plt}\PY{o}{.}\PY{n}{legend}\PY{p}{(}\PY{n}{loc}\PY{o}{=}\PY{l+m+mi}{4}\PY{p}{)}
\PY{n}{plt}\PY{o}{.}\PY{n}{show}\PY{p}{(}\PY{p}{)}
\end{Verbatim}
\end{tcolorbox}

    \begin{center}
    \adjustimage{max size={0.9\linewidth}{0.9\paperheight}}{output_119_0.png}
    \end{center}
    { \hspace*{\fill} \\}
    
    \begin{tcolorbox}[breakable, size=fbox, boxrule=1pt, pad at break*=1mm,colback=cellbackground, colframe=cellborder]
\prompt{In}{incolor}{169}{\boxspacing}
\begin{Verbatim}[commandchars=\\\{\}]
\PY{c+c1}{\PYZsh{}Ajout dans le dico}
\PY{n}{test\PYZus{}result}\PY{p}{[}\PY{l+s+s1}{\PYZsq{}}\PY{l+s+s1}{XGB}\PY{l+s+s1}{\PYZsq{}}\PY{p}{]} \PY{o}{=} \PY{n}{test\PYZus{}xgboost}
\PY{n}{valid\PYZus{}result}\PY{p}{[}\PY{l+s+s1}{\PYZsq{}}\PY{l+s+s1}{XGB}\PY{l+s+s1}{\PYZsq{}}\PY{p}{]} \PY{o}{=} \PY{n}{f1\PYZus{}xgboost}
\PY{n}{auc\PYZus{}result}\PY{p}{[}\PY{l+s+s1}{\PYZsq{}}\PY{l+s+s1}{XGB}\PY{l+s+s1}{\PYZsq{}}\PY{p}{]} \PY{o}{=} \PY{n+nb}{round}\PY{p}{(}\PY{n}{auc\PYZus{}xg}\PY{p}{,}\PY{l+m+mi}{3}\PY{p}{)}
\PY{n}{time\PYZus{}result}\PY{p}{[}\PY{l+s+s1}{\PYZsq{}}\PY{l+s+s1}{XGB}\PY{l+s+s1}{\PYZsq{}}\PY{p}{]} \PY{o}{=} \PY{n+nb}{round}\PY{p}{(}\PY{n}{xg}\PY{p}{,}\PY{l+m+mi}{3}\PY{p}{)}
\PY{n+nb}{print}\PY{p}{(}\PY{n}{test\PYZus{}result}\PY{p}{)}
\PY{n+nb}{print}\PY{p}{(}\PY{n}{valid\PYZus{}result}\PY{p}{)}
\PY{n+nb}{print}\PY{p}{(}\PY{n}{auc\PYZus{}result}\PY{p}{)}
\PY{n+nb}{print}\PY{p}{(}\PY{n}{time\PYZus{}result}\PY{p}{)}
\end{Verbatim}
\end{tcolorbox}

    \begin{Verbatim}[commandchars=\\\{\}]
\{'KNN': 0.039632, 'DT': 0.080354, 'ADL': 0.031711, 'XGB': 0.080354, 'ADXG':
0.083075\}
\{'KNN': 0.048721, 'DT': 0.030846, 'ADL': 0.029688, 'XGB': 0.085355, 'ADXG':
0.106159\}
\{'KNN': 0.536, 'DT': 0.512, 'ADL': 0.681, 'XGB': 0.61, 'ADXG': 0.538\}
\{'KNN': 85.574, 'DT': 63.411, 'ADL': 5.1, 'XGB': 0.756, 'ADXG': 1.934\}
    \end{Verbatim}

    \hypertarget{adl-x-xgboost-ensemble-modeling}{%
\subsubsection{7.6 ADL x XGBoost ( Ensemble modeling
)}\label{adl-x-xgboost-ensemble-modeling}}

    \begin{tcolorbox}[breakable, size=fbox, boxrule=1pt, pad at break*=1mm,colback=cellbackground, colframe=cellborder]
\prompt{In}{incolor}{142}{\boxspacing}
\begin{Verbatim}[commandchars=\\\{\}]
\PY{n}{VotingPredictorAR} \PY{o}{=} \PY{n}{VotingClassifier}\PY{p}{(}\PY{n}{estimators} \PY{o}{=}
                           \PY{p}{[}\PY{p}{(}\PY{l+s+s1}{\PYZsq{}}\PY{l+s+s1}{ADL}\PY{l+s+s1}{\PYZsq{}}\PY{p}{,} \PY{n}{gridsearchADL}\PY{p}{)}\PY{p}{,} 
                            \PY{p}{(}\PY{l+s+s1}{\PYZsq{}}\PY{l+s+s1}{XG}\PY{l+s+s1}{\PYZsq{}}\PY{p}{,} \PY{n}{xg\PYZus{}clas}\PY{p}{)}\PY{p}{]}\PY{p}{,}
                           \PY{n}{voting}\PY{o}{=}\PY{l+s+s1}{\PYZsq{}}\PY{l+s+s1}{soft}\PY{l+s+s1}{\PYZsq{}}\PY{p}{,} \PY{n}{n\PYZus{}jobs} \PY{o}{=} \PY{l+m+mi}{4}\PY{p}{)}
\PY{n}{start} \PY{o}{=} \PY{n}{time}\PY{p}{(}\PY{p}{)}
\PY{n}{VotingPredictorAR} \PY{o}{=} \PY{n}{VotingPredictorAR}\PY{o}{.}\PY{n}{fit}\PY{p}{(}\PY{n}{x\PYZus{}train}\PY{p}{,} \PY{n}{y\PYZus{}train}\PY{p}{)}
\PY{n}{end} \PY{o}{=} \PY{n}{time}\PY{p}{(}\PY{p}{)}
\PY{n}{AR}\PY{o}{=} \PY{p}{(}\PY{n}{end}\PY{o}{\PYZhy{}}\PY{n}{start}\PY{p}{)}\PY{o}{/}\PY{l+m+mi}{60}
\PY{n+nb}{print}\PY{p}{(}\PY{n}{AR}\PY{p}{)}
\end{Verbatim}
\end{tcolorbox}

    \begin{Verbatim}[commandchars=\\\{\}]
1.9340042153994241
    \end{Verbatim}

    \hypertarget{validation}{%
\subparagraph{Validation :}\label{validation}}

    \begin{tcolorbox}[breakable, size=fbox, boxrule=1pt, pad at break*=1mm,colback=cellbackground, colframe=cellborder]
\prompt{In}{incolor}{143}{\boxspacing}
\begin{Verbatim}[commandchars=\\\{\}]
\PY{c+c1}{\PYZsh{}Validation}
\PY{n}{f1\PYZus{}AR} \PY{o}{=} \PY{n+nb}{round}\PY{p}{(}\PY{n}{f1\PYZus{}score}\PY{p}{(}\PY{n}{y\PYZus{}val}\PY{p}{,} \PY{n}{VotingPredictorAR}\PY{o}{.}\PY{n}{predict}\PY{p}{(}\PY{n}{x\PYZus{}val}\PY{p}{)}\PY{p}{)}\PY{p}{,}\PY{l+m+mi}{6}\PY{p}{)}
\PY{n+nb}{print}\PY{p}{(}\PY{n}{metrics}\PY{o}{.}\PY{n}{classification\PYZus{}report}\PY{p}{(}\PY{n}{y\PYZus{}val}\PY{p}{,}\PY{n}{VotingPredictorAR}\PY{o}{.}\PY{n}{predict}\PY{p}{(}\PY{n}{x\PYZus{}val}\PY{p}{)}\PY{p}{)}\PY{p}{)}
\PY{n+nb}{print}\PY{p}{(}\PY{l+s+s2}{\PYZdq{}}\PY{l+s+s2}{Validation score f1 : }\PY{l+s+s2}{\PYZdq{}}\PY{p}{,} \PY{n}{f1\PYZus{}AR}\PY{p}{)}
\end{Verbatim}
\end{tcolorbox}

    \begin{Verbatim}[commandchars=\\\{\}]
              precision    recall  f1-score   support

           0       1.00      1.00      1.00    288737
           1       0.10      0.11      0.11       948

    accuracy                           0.99    289685
   macro avg       0.55      0.55      0.55    289685
weighted avg       0.99      0.99      0.99    289685

Validation score f1 :  0.106159
    \end{Verbatim}

    \hypertarget{test}{%
\subparagraph{Test :}\label{test}}

    \begin{tcolorbox}[breakable, size=fbox, boxrule=1pt, pad at break*=1mm,colback=cellbackground, colframe=cellborder]
\prompt{In}{incolor}{144}{\boxspacing}
\begin{Verbatim}[commandchars=\\\{\}]
\PY{c+c1}{\PYZsh{}Matrice de confusion}
\PY{n}{ADLRF\PYZus{}matrix\PYZus{}test} \PY{o}{=} \PY{n}{confusion\PYZus{}matrix}\PY{p}{(}\PY{n}{test\PYZus{}target}\PY{p}{,} \PY{n}{VotingPredictorAR}\PY{o}{.}\PY{n}{predict}\PY{p}{(}\PY{n}{test\PYZus{}features}\PY{p}{)}\PY{p}{)}
\PY{n}{group\PYZus{}names} \PY{o}{=} \PY{p}{[}\PY{l+s+s1}{\PYZsq{}}\PY{l+s+s1}{True Neg}\PY{l+s+s1}{\PYZsq{}}\PY{p}{,}\PY{l+s+s1}{\PYZsq{}}\PY{l+s+s1}{False Pos}\PY{l+s+s1}{\PYZsq{}}\PY{p}{,}\PY{l+s+s1}{\PYZsq{}}\PY{l+s+s1}{False Neg}\PY{l+s+s1}{\PYZsq{}}\PY{p}{,}\PY{l+s+s1}{\PYZsq{}}\PY{l+s+s1}{True Pos}\PY{l+s+s1}{\PYZsq{}}\PY{p}{]}
\PY{n}{group\PYZus{}counts} \PY{o}{=} \PY{p}{[}\PY{l+s+s2}{\PYZdq{}}\PY{l+s+si}{\PYZob{}0:0.0f\PYZcb{}}\PY{l+s+s2}{\PYZdq{}}\PY{o}{.}\PY{n}{format}\PY{p}{(}\PY{n}{value}\PY{p}{)} \PY{k}{for} \PY{n}{value} \PY{o+ow}{in}
                \PY{n}{ADLRF\PYZus{}matrix\PYZus{}test}\PY{o}{.}\PY{n}{flatten}\PY{p}{(}\PY{p}{)}\PY{p}{]}
\PY{n}{group\PYZus{}percentages} \PY{o}{=} \PY{p}{[}\PY{l+s+s2}{\PYZdq{}}\PY{l+s+si}{\PYZob{}0:.2\PYZpc{}\PYZcb{}}\PY{l+s+s2}{\PYZdq{}}\PY{o}{.}\PY{n}{format}\PY{p}{(}\PY{n}{value}\PY{p}{)} \PY{k}{for} \PY{n}{value} \PY{o+ow}{in}
                     \PY{n}{ADLRF\PYZus{}matrix\PYZus{}test}\PY{o}{.}\PY{n}{flatten}\PY{p}{(}\PY{p}{)}\PY{o}{/}\PY{n}{np}\PY{o}{.}\PY{n}{sum}\PY{p}{(}\PY{n}{ADLRF\PYZus{}matrix\PYZus{}test}\PY{p}{)}\PY{p}{]}
\PY{n}{labels} \PY{o}{=} \PY{p}{[}\PY{l+s+sa}{f}\PY{l+s+s2}{\PYZdq{}}\PY{l+s+si}{\PYZob{}}\PY{n}{v1}\PY{l+s+si}{\PYZcb{}}\PY{l+s+se}{\PYZbs{}n}\PY{l+s+si}{\PYZob{}}\PY{n}{v2}\PY{l+s+si}{\PYZcb{}}\PY{l+s+se}{\PYZbs{}n}\PY{l+s+si}{\PYZob{}}\PY{n}{v3}\PY{l+s+si}{\PYZcb{}}\PY{l+s+s2}{\PYZdq{}} \PY{k}{for} \PY{n}{v1}\PY{p}{,} \PY{n}{v2}\PY{p}{,} \PY{n}{v3} \PY{o+ow}{in}
          \PY{n+nb}{zip}\PY{p}{(}\PY{n}{group\PYZus{}names}\PY{p}{,}\PY{n}{group\PYZus{}counts}\PY{p}{,}\PY{n}{group\PYZus{}percentages}\PY{p}{)}\PY{p}{]}
\PY{n}{labels} \PY{o}{=} \PY{n}{np}\PY{o}{.}\PY{n}{asarray}\PY{p}{(}\PY{n}{labels}\PY{p}{)}\PY{o}{.}\PY{n}{reshape}\PY{p}{(}\PY{l+m+mi}{2}\PY{p}{,}\PY{l+m+mi}{2}\PY{p}{)}
\PY{n}{sns}\PY{o}{.}\PY{n}{heatmap}\PY{p}{(}\PY{n}{ADLRF\PYZus{}matrix\PYZus{}test}\PY{p}{,} \PY{n}{annot}\PY{o}{=}\PY{n}{labels}\PY{p}{,} \PY{n}{fmt}\PY{o}{=}\PY{l+s+s1}{\PYZsq{}}\PY{l+s+s1}{\PYZsq{}}\PY{p}{,} \PY{n}{cmap}\PY{o}{=}\PY{l+s+s1}{\PYZsq{}}\PY{l+s+s1}{Blues}\PY{l+s+s1}{\PYZsq{}}\PY{p}{)}
\end{Verbatim}
\end{tcolorbox}

            \begin{tcolorbox}[breakable, size=fbox, boxrule=.5pt, pad at break*=1mm, opacityfill=0]
\prompt{Out}{outcolor}{144}{\boxspacing}
\begin{Verbatim}[commandchars=\\\{\}]
<matplotlib.axes.\_subplots.AxesSubplot at 0x7fe50f65e250>
\end{Verbatim}
\end{tcolorbox}
        
    \begin{center}
    \adjustimage{max size={0.9\linewidth}{0.9\paperheight}}{output_126_1.png}
    \end{center}
    { \hspace*{\fill} \\}
    
    \begin{tcolorbox}[breakable, size=fbox, boxrule=1pt, pad at break*=1mm,colback=cellbackground, colframe=cellborder]
\prompt{In}{incolor}{145}{\boxspacing}
\begin{Verbatim}[commandchars=\\\{\}]
\PY{c+c1}{\PYZsh{}Test}
\PY{n}{test\PYZus{}AR} \PY{o}{=} \PY{n+nb}{round}\PY{p}{(}\PY{n}{f1\PYZus{}score}\PY{p}{(}\PY{n}{test\PYZus{}target}\PY{p}{,}\PY{n}{VotingPredictorAR}\PY{o}{.}\PY{n}{predict}\PY{p}{(}\PY{n}{test\PYZus{}features}\PY{p}{)}\PY{p}{)}\PY{p}{,}\PY{l+m+mi}{6}\PY{p}{)}
\PY{n+nb}{print}\PY{p}{(}\PY{n}{metrics}\PY{o}{.}\PY{n}{classification\PYZus{}report}\PY{p}{(}\PY{n}{test\PYZus{}target}\PY{p}{,}\PY{n}{VotingPredictorAR}\PY{o}{.}\PY{n}{predict}\PY{p}{(}\PY{n}{test\PYZus{}features}\PY{p}{)}\PY{p}{)}\PY{p}{)}
\PY{n+nb}{print}\PY{p}{(}\PY{l+s+s2}{\PYZdq{}}\PY{l+s+s2}{Validation score f1 : }\PY{l+s+s2}{\PYZdq{}}\PY{p}{,} \PY{n}{test\PYZus{}AR}\PY{p}{)}
\end{Verbatim}
\end{tcolorbox}

    \begin{Verbatim}[commandchars=\\\{\}]
              precision    recall  f1-score   support

           0       1.00      1.00      1.00    263288
           1       0.09      0.08      0.08       855

    accuracy                           0.99    264143
   macro avg       0.54      0.54      0.54    264143
weighted avg       0.99      0.99      0.99    264143

Validation score f1 :  0.083075
    \end{Verbatim}

    \begin{tcolorbox}[breakable, size=fbox, boxrule=1pt, pad at break*=1mm,colback=cellbackground, colframe=cellborder]
\prompt{In}{incolor}{146}{\boxspacing}
\begin{Verbatim}[commandchars=\\\{\}]
\PY{c+c1}{\PYZsh{}Courbe ROC}
\PY{n}{fpr}\PY{p}{,} \PY{n}{tpr}\PY{p}{,} \PY{n}{\PYZus{}} \PY{o}{=} \PY{n}{roc\PYZus{}curve}\PY{p}{(}\PY{n}{test\PYZus{}target}\PY{p}{,} \PY{n}{VotingPredictorAR}\PY{o}{.}\PY{n}{predict}\PY{p}{(}\PY{n}{test\PYZus{}features}\PY{p}{)}\PY{p}{)}
\PY{n}{auc\PYZus{}ar}\PY{o}{=} \PY{n}{metrics}\PY{o}{.}\PY{n}{roc\PYZus{}auc\PYZus{}score}\PY{p}{(}\PY{n}{test\PYZus{}target}\PY{p}{,}\PY{n}{VotingPredictorAR}\PY{o}{.}\PY{n}{predict}\PY{p}{(}\PY{n}{test\PYZus{}features}\PY{p}{)}\PY{p}{)}
\PY{n}{plt}\PY{o}{.}\PY{n}{plot}\PY{p}{(}\PY{n}{fpr}\PY{p}{,}\PY{n}{tpr}\PY{p}{,}\PY{n}{label}\PY{o}{=}\PY{l+s+s2}{\PYZdq{}}\PY{l+s+s2}{data 1, auc=}\PY{l+s+s2}{\PYZdq{}}\PY{o}{+}\PY{n+nb}{str}\PY{p}{(}\PY{n}{auc\PYZus{}ar}\PY{p}{)}\PY{p}{)}
\PY{n}{plt}\PY{o}{.}\PY{n}{legend}\PY{p}{(}\PY{n}{loc}\PY{o}{=}\PY{l+m+mi}{4}\PY{p}{)}
\PY{n}{plt}\PY{o}{.}\PY{n}{show}\PY{p}{(}\PY{p}{)}
\end{Verbatim}
\end{tcolorbox}

    \begin{center}
    \adjustimage{max size={0.9\linewidth}{0.9\paperheight}}{output_128_0.png}
    \end{center}
    { \hspace*{\fill} \\}
    
    \begin{tcolorbox}[breakable, size=fbox, boxrule=1pt, pad at break*=1mm,colback=cellbackground, colframe=cellborder]
\prompt{In}{incolor}{168}{\boxspacing}
\begin{Verbatim}[commandchars=\\\{\}]
\PY{c+c1}{\PYZsh{}Ajout dans le dico}
\PY{n}{test\PYZus{}result}\PY{p}{[}\PY{l+s+s1}{\PYZsq{}}\PY{l+s+s1}{ADXG}\PY{l+s+s1}{\PYZsq{}}\PY{p}{]} \PY{o}{=} \PY{n}{test\PYZus{}AR}
\PY{n}{valid\PYZus{}result}\PY{p}{[}\PY{l+s+s1}{\PYZsq{}}\PY{l+s+s1}{ADXG}\PY{l+s+s1}{\PYZsq{}}\PY{p}{]} \PY{o}{=} \PY{n}{f1\PYZus{}AR}
\PY{n}{auc\PYZus{}result}\PY{p}{[}\PY{l+s+s1}{\PYZsq{}}\PY{l+s+s1}{ADXG}\PY{l+s+s1}{\PYZsq{}}\PY{p}{]} \PY{o}{=} \PY{n+nb}{round}\PY{p}{(}\PY{n}{auc\PYZus{}ar}\PY{p}{,}\PY{l+m+mi}{3}\PY{p}{)}
\PY{n}{time\PYZus{}result}\PY{p}{[}\PY{l+s+s1}{\PYZsq{}}\PY{l+s+s1}{ADXG}\PY{l+s+s1}{\PYZsq{}}\PY{p}{]} \PY{o}{=} \PY{n+nb}{round}\PY{p}{(}\PY{n}{AR}\PY{p}{,}\PY{l+m+mi}{3}\PY{p}{)}
\PY{n+nb}{print}\PY{p}{(}\PY{n}{test\PYZus{}result}\PY{p}{)}
\PY{n+nb}{print}\PY{p}{(}\PY{n}{valid\PYZus{}result}\PY{p}{)}
\PY{n+nb}{print}\PY{p}{(}\PY{n}{auc\PYZus{}result}\PY{p}{)}
\PY{n+nb}{print}\PY{p}{(}\PY{n}{time\PYZus{}result}\PY{p}{)}
\end{Verbatim}
\end{tcolorbox}

    \begin{Verbatim}[commandchars=\\\{\}]
\{'KNN': 0.039632, 'DT': 0.080354, 'ADL': 0.031711, 'XGB': 0.080354, 'ADXG':
0.083075\}
\{'KNN': 0.048721, 'DT': 0.030846, 'ADL': 0.029688, 'XGB': 0.085355, 'ADXG':
0.106159\}
\{'KNN': 0.536, 'DT': 0.512, 'ADL': 0.681, 'XGB': 0.6095090476157476, 'ADXG':
0.538\}
\{'KNN': 85.574, 'DT': 63.411, 'ADL': 5.1, 'XGB': 0.7562666666666666, 'ADXG':
1.934\}
    \end{Verbatim}

    \hypertarget{random-forest}{%
\subsubsection{7.7 Random Forest}\label{random-forest}}

Le premier algorithme que nous avons tester est le random forest appelé
en français la forêt aléatoire. Il consiste à assembler plusieurs arbres
de décision. Cette algorithme donne la possibilité d'affecter un poid en
fonction de la classe. Nous allons donc tester la forêt aléatoire en
utilisant les données ré échantillonné puis les données déséquilibrés en
rajoutant un poids.

\hypertarget{avec-smote}{%
\paragraph{7.7.1 Avec SMOTE}\label{avec-smote}}

    \begin{tcolorbox}[breakable, size=fbox, boxrule=1pt, pad at break*=1mm,colback=cellbackground, colframe=cellborder]
\prompt{In}{incolor}{17}{\boxspacing}
\begin{Verbatim}[commandchars=\\\{\}]
\PY{n}{pipeRFS} \PY{o}{=} \PY{n}{make\PYZus{}pipeline}\PY{p}{(}\PY{n}{StandardScaler}\PY{p}{(}\PY{p}{)}\PY{p}{,} \PY{n}{RandomForestClassifier}\PY{p}{(}\PY{n}{max\PYZus{}depth}\PY{o}{=}\PY{l+m+mi}{100}\PY{p}{,}\PY{n}{min\PYZus{}samples\PYZus{}leaf}\PY{o}{=}\PY{l+m+mi}{20}\PY{p}{,}
\PY{n}{n\PYZus{}estimators}\PY{o}{=}\PY{l+m+mi}{500}\PY{p}{,} \PY{n}{n\PYZus{}jobs}\PY{o}{=}\PY{o}{\PYZhy{}}\PY{l+m+mi}{1}\PY{p}{,} \PY{n}{random\PYZus{}state}\PY{o}{=}\PY{l+m+mi}{5}\PY{p}{,} \PY{n}{bootstrap}\PY{o}{=}\PY{k+kc}{True}\PY{p}{,} \PY{n}{criterion}\PY{o}{=}\PY{l+s+s1}{\PYZsq{}}\PY{l+s+s1}{entropy}\PY{l+s+s1}{\PYZsq{}}\PY{p}{)}\PY{p}{)}
\PY{n}{start} \PY{o}{=} \PY{n}{time}\PY{p}{(}\PY{p}{)}
\PY{n}{pipeRFS}\PY{o}{.}\PY{n}{fit}\PY{p}{(}\PY{n}{x\PYZus{}train\PYZus{}res}\PY{p}{,} \PY{n}{y\PYZus{}train\PYZus{}res}\PY{p}{)} 
\PY{n}{end} \PY{o}{=} \PY{n}{time}\PY{p}{(}\PY{p}{)}
\PY{n}{RFS} \PY{o}{=} \PY{p}{(}\PY{n}{end}\PY{o}{\PYZhy{}}\PY{n}{start}\PY{p}{)}\PY{o}{/}\PY{l+m+mi}{60}
\PY{n+nb}{print}\PY{p}{(}\PY{l+s+s2}{\PYZdq{}}\PY{l+s+s2}{temps de calcul : }\PY{l+s+s2}{\PYZdq{}}\PY{p}{,}\PY{n}{RFS}\PY{p}{)}
\end{Verbatim}
\end{tcolorbox}

    \begin{Verbatim}[commandchars=\\\{\}]
temps de calcul :  48.06304409901301
    \end{Verbatim}

    \hypertarget{validation}{%
\subparagraph{Validation :}\label{validation}}

    \begin{tcolorbox}[breakable, size=fbox, boxrule=1pt, pad at break*=1mm,colback=cellbackground, colframe=cellborder]
\prompt{In}{incolor}{18}{\boxspacing}
\begin{Verbatim}[commandchars=\\\{\}]
\PY{n}{f1\PYZus{}rfas} \PY{o}{=} \PY{n+nb}{round}\PY{p}{(}\PY{n}{f1\PYZus{}score}\PY{p}{(}\PY{n}{y\PYZus{}val}\PY{p}{,} \PY{n}{pipeRFS}\PY{o}{.}\PY{n}{predict}\PY{p}{(}\PY{n}{x\PYZus{}val}\PY{p}{)}\PY{p}{)}\PY{p}{,}\PY{l+m+mi}{6}\PY{p}{)}
\PY{n+nb}{print}\PY{p}{(}\PY{n}{metrics}\PY{o}{.}\PY{n}{classification\PYZus{}report}\PY{p}{(}\PY{n}{y\PYZus{}val}\PY{p}{,} \PY{n}{pipeRFS}\PY{o}{.}\PY{n}{predict}\PY{p}{(}\PY{n}{x\PYZus{}val}\PY{p}{)}\PY{p}{)}\PY{p}{)}
\PY{n+nb}{print} \PY{p}{(}\PY{l+s+s2}{\PYZdq{}}\PY{l+s+s2}{Validation score f1:}\PY{l+s+s2}{\PYZdq{}}\PY{p}{,}\PY{n}{f1\PYZus{}rfas}\PY{p}{)}
\end{Verbatim}
\end{tcolorbox}

    \begin{Verbatim}[commandchars=\\\{\}]
              precision    recall  f1-score   support

           0       1.00      1.00      1.00    288737
           1       0.17      0.06      0.09       948

    accuracy                           1.00    289685
   macro avg       0.59      0.53      0.55    289685
weighted avg       0.99      1.00      1.00    289685

Validation score f1: 0.093918
    \end{Verbatim}

    \hypertarget{test}{%
\subparagraph{Test :}\label{test}}

    \begin{tcolorbox}[breakable, size=fbox, boxrule=1pt, pad at break*=1mm,colback=cellbackground, colframe=cellborder]
\prompt{In}{incolor}{20}{\boxspacing}
\begin{Verbatim}[commandchars=\\\{\}]
\PY{c+c1}{\PYZsh{}Matrice de confusion}
\PY{n}{RFS\PYZus{}matrix\PYZus{}test} \PY{o}{=} \PY{n}{confusion\PYZus{}matrix}\PY{p}{(}\PY{n}{test\PYZus{}target}\PY{p}{,}\PY{n}{pipeRFS}\PY{o}{.}\PY{n}{predict}\PY{p}{(}\PY{n}{test\PYZus{}features}\PY{p}{)}\PY{p}{)}
\PY{n}{group\PYZus{}names} \PY{o}{=} \PY{p}{[}\PY{l+s+s1}{\PYZsq{}}\PY{l+s+s1}{True Neg}\PY{l+s+s1}{\PYZsq{}}\PY{p}{,}\PY{l+s+s1}{\PYZsq{}}\PY{l+s+s1}{False Pos}\PY{l+s+s1}{\PYZsq{}}\PY{p}{,}\PY{l+s+s1}{\PYZsq{}}\PY{l+s+s1}{False Neg}\PY{l+s+s1}{\PYZsq{}}\PY{p}{,}\PY{l+s+s1}{\PYZsq{}}\PY{l+s+s1}{True Pos}\PY{l+s+s1}{\PYZsq{}}\PY{p}{]}
\PY{n}{group\PYZus{}counts} \PY{o}{=} \PY{p}{[}\PY{l+s+s2}{\PYZdq{}}\PY{l+s+si}{\PYZob{}0:0.0f\PYZcb{}}\PY{l+s+s2}{\PYZdq{}}\PY{o}{.}\PY{n}{format}\PY{p}{(}\PY{n}{value}\PY{p}{)} \PY{k}{for} \PY{n}{value} \PY{o+ow}{in}
                \PY{n}{RFS\PYZus{}matrix\PYZus{}test}\PY{o}{.}\PY{n}{flatten}\PY{p}{(}\PY{p}{)}\PY{p}{]}
\PY{n}{group\PYZus{}percentages} \PY{o}{=} \PY{p}{[}\PY{l+s+s2}{\PYZdq{}}\PY{l+s+si}{\PYZob{}0:.2\PYZpc{}\PYZcb{}}\PY{l+s+s2}{\PYZdq{}}\PY{o}{.}\PY{n}{format}\PY{p}{(}\PY{n}{value}\PY{p}{)} \PY{k}{for} \PY{n}{value} \PY{o+ow}{in}
                     \PY{n}{RFS\PYZus{}matrix\PYZus{}test}\PY{o}{.}\PY{n}{flatten}\PY{p}{(}\PY{p}{)}\PY{o}{/}\PY{n}{np}\PY{o}{.}\PY{n}{sum}\PY{p}{(}\PY{n}{RFS\PYZus{}matrix\PYZus{}test}\PY{p}{)}\PY{p}{]}
\PY{n}{labels} \PY{o}{=} \PY{p}{[}\PY{l+s+sa}{f}\PY{l+s+s2}{\PYZdq{}}\PY{l+s+si}{\PYZob{}}\PY{n}{v1}\PY{l+s+si}{\PYZcb{}}\PY{l+s+se}{\PYZbs{}n}\PY{l+s+si}{\PYZob{}}\PY{n}{v2}\PY{l+s+si}{\PYZcb{}}\PY{l+s+se}{\PYZbs{}n}\PY{l+s+si}{\PYZob{}}\PY{n}{v3}\PY{l+s+si}{\PYZcb{}}\PY{l+s+s2}{\PYZdq{}} \PY{k}{for} \PY{n}{v1}\PY{p}{,} \PY{n}{v2}\PY{p}{,} \PY{n}{v3} \PY{o+ow}{in}
          \PY{n+nb}{zip}\PY{p}{(}\PY{n}{group\PYZus{}names}\PY{p}{,}\PY{n}{group\PYZus{}counts}\PY{p}{,}\PY{n}{group\PYZus{}percentages}\PY{p}{)}\PY{p}{]}
\PY{n}{labels} \PY{o}{=} \PY{n}{np}\PY{o}{.}\PY{n}{asarray}\PY{p}{(}\PY{n}{labels}\PY{p}{)}\PY{o}{.}\PY{n}{reshape}\PY{p}{(}\PY{l+m+mi}{2}\PY{p}{,}\PY{l+m+mi}{2}\PY{p}{)}
\PY{n}{sns}\PY{o}{.}\PY{n}{heatmap}\PY{p}{(}\PY{n}{RFS\PYZus{}matrix\PYZus{}test}\PY{p}{,} \PY{n}{annot}\PY{o}{=}\PY{n}{labels}\PY{p}{,} \PY{n}{fmt}\PY{o}{=}\PY{l+s+s1}{\PYZsq{}}\PY{l+s+s1}{\PYZsq{}}\PY{p}{,} \PY{n}{cmap}\PY{o}{=}\PY{l+s+s1}{\PYZsq{}}\PY{l+s+s1}{Blues}\PY{l+s+s1}{\PYZsq{}}\PY{p}{)}
\end{Verbatim}
\end{tcolorbox}

            \begin{tcolorbox}[breakable, size=fbox, boxrule=.5pt, pad at break*=1mm, opacityfill=0]
\prompt{Out}{outcolor}{20}{\boxspacing}
\begin{Verbatim}[commandchars=\\\{\}]
<matplotlib.axes.\_subplots.AxesSubplot at 0x7fe515e41100>
\end{Verbatim}
\end{tcolorbox}
        
    \begin{center}
    \adjustimage{max size={0.9\linewidth}{0.9\paperheight}}{output_135_1.png}
    \end{center}
    { \hspace*{\fill} \\}
    
    \begin{tcolorbox}[breakable, size=fbox, boxrule=1pt, pad at break*=1mm,colback=cellbackground, colframe=cellborder]
\prompt{In}{incolor}{19}{\boxspacing}
\begin{Verbatim}[commandchars=\\\{\}]
\PY{c+c1}{\PYZsh{}ajout des resultats dans le dico}
\PY{n}{test\PYZus{}rfas} \PY{o}{=} \PY{n+nb}{round}\PY{p}{(}\PY{n}{f1\PYZus{}score}\PY{p}{(}\PY{n}{test\PYZus{}target}\PY{p}{,} \PY{n}{pipeRFS}\PY{o}{.}\PY{n}{predict}\PY{p}{(}\PY{n}{test\PYZus{}features}\PY{p}{)}\PY{p}{)}\PY{p}{,}\PY{l+m+mi}{6}\PY{p}{)}
\PY{n+nb}{print}\PY{p}{(}\PY{n}{metrics}\PY{o}{.}\PY{n}{classification\PYZus{}report}\PY{p}{(}\PY{n}{test\PYZus{}target}\PY{p}{,} \PY{n}{pipeRFS}\PY{o}{.}\PY{n}{predict}\PY{p}{(}\PY{n}{test\PYZus{}features}\PY{p}{)}\PY{p}{)}\PY{p}{)}
\PY{n+nb}{print} \PY{p}{(}\PY{l+s+s2}{\PYZdq{}}\PY{l+s+s2}{Test score f1:}\PY{l+s+s2}{\PYZdq{}}\PY{p}{,}\PY{n}{test\PYZus{}rfas}\PY{p}{)}
\end{Verbatim}
\end{tcolorbox}

    \begin{Verbatim}[commandchars=\\\{\}]
              precision    recall  f1-score   support

           0       1.00      1.00      1.00    263288
           1       0.13      0.04      0.07       855

    accuracy                           1.00    264143
   macro avg       0.56      0.52      0.53    264143
weighted avg       0.99      1.00      0.99    264143

Test score f1: 0.065744
    \end{Verbatim}

    Le f1 maximum que l'on a réussi à avoir avec cette algorithme est de
0.06, c'est un score pas très élevé, nous allons essayer de trouver un
meilleur modèle.

    \begin{tcolorbox}[breakable, size=fbox, boxrule=1pt, pad at break*=1mm,colback=cellbackground, colframe=cellborder]
\prompt{In}{incolor}{128}{\boxspacing}
\begin{Verbatim}[commandchars=\\\{\}]
\PY{c+c1}{\PYZsh{}Courbe ROC}
\PY{n}{fpr}\PY{p}{,} \PY{n}{tpr}\PY{p}{,} \PY{n}{\PYZus{}} \PY{o}{=} \PY{n}{roc\PYZus{}curve}\PY{p}{(}\PY{n}{test\PYZus{}target}\PY{p}{,}\PY{n}{pipeRFS}\PY{o}{.}\PY{n}{predict}\PY{p}{(}\PY{n}{test\PYZus{}features}\PY{p}{)}\PY{p}{)}
\PY{n}{auc\PYZus{}rfs} \PY{o}{=} \PY{n}{metrics}\PY{o}{.}\PY{n}{roc\PYZus{}auc\PYZus{}score}\PY{p}{(}\PY{n}{test\PYZus{}target}\PY{p}{,} \PY{n}{pipeRFS}\PY{o}{.}\PY{n}{predict}\PY{p}{(}\PY{n}{test\PYZus{}features}\PY{p}{)}\PY{p}{)}
\PY{n}{plt}\PY{o}{.}\PY{n}{plot}\PY{p}{(}\PY{n}{fpr}\PY{p}{,}\PY{n}{tpr}\PY{p}{,}\PY{n}{label}\PY{o}{=}\PY{l+s+s2}{\PYZdq{}}\PY{l+s+s2}{data 1, auc=}\PY{l+s+s2}{\PYZdq{}}\PY{o}{+}\PY{n+nb}{str}\PY{p}{(}\PY{n}{auc\PYZus{}rfs}\PY{p}{)}\PY{p}{)}
\PY{n}{plt}\PY{o}{.}\PY{n}{legend}\PY{p}{(}\PY{n}{loc}\PY{o}{=}\PY{l+m+mi}{4}\PY{p}{)}
\PY{n}{plt}\PY{o}{.}\PY{n}{show}\PY{p}{(}\PY{p}{)}
\end{Verbatim}
\end{tcolorbox}

    \begin{center}
    \adjustimage{max size={0.9\linewidth}{0.9\paperheight}}{output_138_0.png}
    \end{center}
    { \hspace*{\fill} \\}
    
    \begin{tcolorbox}[breakable, size=fbox, boxrule=1pt, pad at break*=1mm,colback=cellbackground, colframe=cellborder]
\prompt{In}{incolor}{170}{\boxspacing}
\begin{Verbatim}[commandchars=\\\{\}]
\PY{n}{test\PYZus{}result}\PY{p}{[}\PY{l+s+s1}{\PYZsq{}}\PY{l+s+s1}{RFSM}\PY{l+s+s1}{\PYZsq{}}\PY{p}{]} \PY{o}{=} \PY{n}{test\PYZus{}rfas}
\PY{n}{valid\PYZus{}result}\PY{p}{[}\PY{l+s+s1}{\PYZsq{}}\PY{l+s+s1}{RFSM}\PY{l+s+s1}{\PYZsq{}}\PY{p}{]} \PY{o}{=} \PY{n}{f1\PYZus{}rfas}
\PY{n}{auc\PYZus{}result}\PY{p}{[}\PY{l+s+s1}{\PYZsq{}}\PY{l+s+s1}{RFSM}\PY{l+s+s1}{\PYZsq{}}\PY{p}{]} \PY{o}{=} \PY{n+nb}{round}\PY{p}{(}\PY{n}{auc\PYZus{}rfs}\PY{p}{,}\PY{l+m+mi}{3}\PY{p}{)}
\PY{n}{time\PYZus{}result}\PY{p}{[}\PY{l+s+s1}{\PYZsq{}}\PY{l+s+s1}{RFSM}\PY{l+s+s1}{\PYZsq{}}\PY{p}{]} \PY{o}{=} \PY{n+nb}{round}\PY{p}{(}\PY{n}{RFS}\PY{p}{,}\PY{l+m+mi}{3}\PY{p}{)}
\PY{n+nb}{print}\PY{p}{(}\PY{n}{test\PYZus{}result}\PY{p}{)}
\PY{n+nb}{print}\PY{p}{(}\PY{n}{valid\PYZus{}result}\PY{p}{)}
\PY{n+nb}{print}\PY{p}{(}\PY{n}{auc\PYZus{}result}\PY{p}{)}
\PY{n+nb}{print}\PY{p}{(}\PY{n}{time\PYZus{}result}\PY{p}{)}
\end{Verbatim}
\end{tcolorbox}

    \begin{Verbatim}[commandchars=\\\{\}]
\{'KNN': 0.039632, 'DT': 0.080354, 'ADL': 0.031711, 'XGB': 0.080354, 'ADXG':
0.083075, 'RFSM': 0.065744\}
\{'KNN': 0.048721, 'DT': 0.030846, 'ADL': 0.029688, 'XGB': 0.085355, 'ADXG':
0.106159, 'RFSM': 0.093918\}
\{'KNN': 0.536, 'DT': 0.512, 'ADL': 0.681, 'XGB': 0.61, 'ADXG': 0.538, 'RFSM':
0.522\}
\{'KNN': 85.574, 'DT': 63.411, 'ADL': 5.1, 'XGB': 0.756, 'ADXG': 1.934, 'RFSM':
48.063\}
    \end{Verbatim}

    \hypertarget{sans-smote}{%
\paragraph{7.7.2 Sans SMOTE}\label{sans-smote}}

Ici, nous utilison un random forest sans utiliser les données
rééquilibrées avec un smote. On va accorder un poid en fonction de la
classe. Pour ce faire on utilise l'option ``class\_weight'' dans notre
algortihme. La classe négative, non-fraude (0) est 370 fois plus
présente que la classe fraude (1), donc nous allons mettre l'option
``class\_weight=\{0:1, 1:370\}''

    \begin{tcolorbox}[breakable, size=fbox, boxrule=1pt, pad at break*=1mm,colback=cellbackground, colframe=cellborder]
\prompt{In}{incolor}{134}{\boxspacing}
\begin{Verbatim}[commandchars=\\\{\}]
\PY{n}{startRF} \PY{o}{=} \PY{n}{time}\PY{p}{(}\PY{p}{)}
\PY{n}{RDbest1} \PY{o}{=} \PY{n}{RandomForestClassifier}\PY{p}{(}\PY{n}{bootstrap}\PY{o}{=}\PY{k+kc}{True}\PY{p}{,} \PY{n}{class\PYZus{}weight}\PY{o}{=}\PY{p}{\PYZob{}}\PY{l+m+mi}{0}\PY{p}{:}\PY{l+m+mi}{1}\PY{p}{,} \PY{l+m+mi}{1}\PY{p}{:}\PY{l+m+mi}{370}\PY{p}{\PYZcb{}}\PY{p}{,} \PY{n}{criterion}\PY{o}{=}\PY{l+s+s1}{\PYZsq{}}\PY{l+s+s1}{entropy}\PY{l+s+s1}{\PYZsq{}}\PY{p}{,} \PY{n}{max\PYZus{}depth}\PY{o}{=}\PY{l+m+mi}{100}\PY{p}{,}\PY{n}{min\PYZus{}samples\PYZus{}leaf}\PY{o}{=}\PY{l+m+mi}{20}\PY{p}{,}
\PY{n}{n\PYZus{}estimators}\PY{o}{=}\PY{l+m+mi}{500}\PY{p}{,} \PY{n}{n\PYZus{}jobs}\PY{o}{=}\PY{o}{\PYZhy{}}\PY{l+m+mi}{1}\PY{p}{,} \PY{n}{random\PYZus{}state}\PY{o}{=}\PY{l+m+mi}{5}\PY{p}{)}
\PY{n}{RDbest1}\PY{o}{.}\PY{n}{fit}\PY{p}{(}\PY{n}{x\PYZus{}train}\PY{p}{,} \PY{n}{y\PYZus{}train}\PY{p}{)} 
\PY{n}{doneRF} \PY{o}{=} \PY{n}{time}\PY{p}{(}\PY{p}{)}

\PY{n}{tpsRF} \PY{o}{=} \PY{n+nb}{round}\PY{p}{(}\PY{n}{doneRF} \PY{o}{\PYZhy{}} \PY{n}{startRF}\PY{p}{,}\PY{l+m+mi}{3}\PY{p}{)}\PY{o}{/}\PY{l+m+mi}{60}
\PY{n+nb}{print}\PY{p}{(}\PY{l+s+s2}{\PYZdq{}}\PY{l+s+s2}{temps de calcul}\PY{l+s+s2}{\PYZdq{}} \PY{p}{,} \PY{n}{tpsRF}\PY{p}{)}
\end{Verbatim}
\end{tcolorbox}

    \begin{Verbatim}[commandchars=\\\{\}]
temps de calcul 9.095049999999999
    \end{Verbatim}

    \hypertarget{validation}{%
\subparagraph{Validation :}\label{validation}}

    \begin{tcolorbox}[breakable, size=fbox, boxrule=1pt, pad at break*=1mm,colback=cellbackground, colframe=cellborder]
\prompt{In}{incolor}{26}{\boxspacing}
\begin{Verbatim}[commandchars=\\\{\}]
\PY{c+c1}{\PYZsh{}Métriques d\PYZsq{}évaluation}
\PY{n}{f1\PYZus{}rfss} \PY{o}{=} \PY{n+nb}{round}\PY{p}{(}\PY{n}{f1\PYZus{}score}\PY{p}{(}\PY{n}{y\PYZus{}val}\PY{p}{,} \PY{n}{RDbest1}\PY{o}{.}\PY{n}{predict}\PY{p}{(}\PY{n}{x\PYZus{}val}\PY{p}{)}\PY{p}{)}\PY{p}{,}\PY{l+m+mi}{6}\PY{p}{)}
\PY{n+nb}{print}\PY{p}{(}\PY{n}{metrics}\PY{o}{.}\PY{n}{classification\PYZus{}report}\PY{p}{(}\PY{n}{y\PYZus{}val}\PY{p}{,} \PY{n}{RDbest1}\PY{o}{.}\PY{n}{predict}\PY{p}{(}\PY{n}{x\PYZus{}val}\PY{p}{)}\PY{p}{)}\PY{p}{)}
\PY{n+nb}{print}\PY{p}{(}\PY{l+s+s2}{\PYZdq{}}\PY{l+s+s2}{Validation score f1:}\PY{l+s+s2}{\PYZdq{}}\PY{p}{,} \PY{n}{f1\PYZus{}rfss} \PY{p}{)}
\end{Verbatim}
\end{tcolorbox}

    \begin{Verbatim}[commandchars=\\\{\}]
              precision    recall  f1-score   support

           0       1.00      1.00      1.00    288737
           1       0.16      0.16      0.16       948

    accuracy                           0.99    289685
   macro avg       0.58      0.58      0.58    289685
weighted avg       0.99      0.99      0.99    289685

Validation score f1: 0.157978
    \end{Verbatim}

    Le score maximum que nous avons réussi à obtenir est de 16\% ce qui est
un bon score pour ce type de classification, c'est 2 fois plus qu'avec
le modèle précédent. Voyons à présent si il est aussi bon sur
l'échantillon de test.

    \hypertarget{test}{%
\subparagraph{Test :}\label{test}}

    \begin{tcolorbox}[breakable, size=fbox, boxrule=1pt, pad at break*=1mm,colback=cellbackground, colframe=cellborder]
\prompt{In}{incolor}{27}{\boxspacing}
\begin{Verbatim}[commandchars=\\\{\}]
\PY{c+c1}{\PYZsh{}Matrice de confusion}
\PY{n}{RDbest\PYZus{}matrix\PYZus{}test} \PY{o}{=} \PY{n}{confusion\PYZus{}matrix}\PY{p}{(}\PY{n}{test\PYZus{}target}\PY{p}{,}\PY{n}{RDbest1}\PY{o}{.}\PY{n}{predict}\PY{p}{(}\PY{n}{test\PYZus{}features}\PY{p}{)}\PY{p}{)}
\PY{n}{group\PYZus{}names} \PY{o}{=} \PY{p}{[}\PY{l+s+s1}{\PYZsq{}}\PY{l+s+s1}{True Neg}\PY{l+s+s1}{\PYZsq{}}\PY{p}{,}\PY{l+s+s1}{\PYZsq{}}\PY{l+s+s1}{False Pos}\PY{l+s+s1}{\PYZsq{}}\PY{p}{,}\PY{l+s+s1}{\PYZsq{}}\PY{l+s+s1}{False Neg}\PY{l+s+s1}{\PYZsq{}}\PY{p}{,}\PY{l+s+s1}{\PYZsq{}}\PY{l+s+s1}{True Pos}\PY{l+s+s1}{\PYZsq{}}\PY{p}{]}
\PY{n}{group\PYZus{}counts} \PY{o}{=} \PY{p}{[}\PY{l+s+s2}{\PYZdq{}}\PY{l+s+si}{\PYZob{}0:0.0f\PYZcb{}}\PY{l+s+s2}{\PYZdq{}}\PY{o}{.}\PY{n}{format}\PY{p}{(}\PY{n}{value}\PY{p}{)} \PY{k}{for} \PY{n}{value} \PY{o+ow}{in}
                \PY{n}{RDbest\PYZus{}matrix\PYZus{}test}\PY{o}{.}\PY{n}{flatten}\PY{p}{(}\PY{p}{)}\PY{p}{]}
\PY{n}{group\PYZus{}percentages} \PY{o}{=} \PY{p}{[}\PY{l+s+s2}{\PYZdq{}}\PY{l+s+si}{\PYZob{}0:.2\PYZpc{}\PYZcb{}}\PY{l+s+s2}{\PYZdq{}}\PY{o}{.}\PY{n}{format}\PY{p}{(}\PY{n}{value}\PY{p}{)} \PY{k}{for} \PY{n}{value} \PY{o+ow}{in}
                     \PY{n}{RDbest\PYZus{}matrix\PYZus{}test}\PY{o}{.}\PY{n}{flatten}\PY{p}{(}\PY{p}{)}\PY{o}{/}\PY{n}{np}\PY{o}{.}\PY{n}{sum}\PY{p}{(}\PY{n}{RDbest\PYZus{}matrix\PYZus{}test}\PY{p}{)}\PY{p}{]}
\PY{n}{labels} \PY{o}{=} \PY{p}{[}\PY{l+s+sa}{f}\PY{l+s+s2}{\PYZdq{}}\PY{l+s+si}{\PYZob{}}\PY{n}{v1}\PY{l+s+si}{\PYZcb{}}\PY{l+s+se}{\PYZbs{}n}\PY{l+s+si}{\PYZob{}}\PY{n}{v2}\PY{l+s+si}{\PYZcb{}}\PY{l+s+se}{\PYZbs{}n}\PY{l+s+si}{\PYZob{}}\PY{n}{v3}\PY{l+s+si}{\PYZcb{}}\PY{l+s+s2}{\PYZdq{}} \PY{k}{for} \PY{n}{v1}\PY{p}{,} \PY{n}{v2}\PY{p}{,} \PY{n}{v3} \PY{o+ow}{in}
          \PY{n+nb}{zip}\PY{p}{(}\PY{n}{group\PYZus{}names}\PY{p}{,}\PY{n}{group\PYZus{}counts}\PY{p}{,}\PY{n}{group\PYZus{}percentages}\PY{p}{)}\PY{p}{]}
\PY{n}{labels} \PY{o}{=} \PY{n}{np}\PY{o}{.}\PY{n}{asarray}\PY{p}{(}\PY{n}{labels}\PY{p}{)}\PY{o}{.}\PY{n}{reshape}\PY{p}{(}\PY{l+m+mi}{2}\PY{p}{,}\PY{l+m+mi}{2}\PY{p}{)}
\PY{n}{sns}\PY{o}{.}\PY{n}{heatmap}\PY{p}{(}\PY{n}{RDbest\PYZus{}matrix\PYZus{}test}\PY{p}{,} \PY{n}{annot}\PY{o}{=}\PY{n}{labels}\PY{p}{,} \PY{n}{fmt}\PY{o}{=}\PY{l+s+s1}{\PYZsq{}}\PY{l+s+s1}{\PYZsq{}}\PY{p}{,} \PY{n}{cmap}\PY{o}{=}\PY{l+s+s1}{\PYZsq{}}\PY{l+s+s1}{Blues}\PY{l+s+s1}{\PYZsq{}}\PY{p}{)}
\end{Verbatim}
\end{tcolorbox}

            \begin{tcolorbox}[breakable, size=fbox, boxrule=.5pt, pad at break*=1mm, opacityfill=0]
\prompt{Out}{outcolor}{27}{\boxspacing}
\begin{Verbatim}[commandchars=\\\{\}]
<matplotlib.axes.\_subplots.AxesSubplot at 0x7fe5021e7400>
\end{Verbatim}
\end{tcolorbox}
        
    \begin{center}
    \adjustimage{max size={0.9\linewidth}{0.9\paperheight}}{output_146_1.png}
    \end{center}
    { \hspace*{\fill} \\}
    
    Le modèle a detecé 120 cas de fraud sur l'ensemble de 855 fraudes qui
est beaucoup mieux par rapport au modèle avec SMOTE.

    \begin{tcolorbox}[breakable, size=fbox, boxrule=1pt, pad at break*=1mm,colback=cellbackground, colframe=cellborder]
\prompt{In}{incolor}{28}{\boxspacing}
\begin{Verbatim}[commandchars=\\\{\}]
\PY{n}{test\PYZus{}rfss} \PY{o}{=} \PY{n+nb}{round}\PY{p}{(}\PY{n}{f1\PYZus{}score}\PY{p}{(}\PY{n}{test\PYZus{}target}\PY{p}{,}\PY{n}{RDbest1}\PY{o}{.}\PY{n}{predict}\PY{p}{(}\PY{n}{test\PYZus{}features}\PY{p}{)} \PY{p}{)}\PY{p}{,}\PY{l+m+mi}{6}\PY{p}{)}
\PY{n+nb}{print}\PY{p}{(}\PY{n}{metrics}\PY{o}{.}\PY{n}{classification\PYZus{}report}\PY{p}{(}\PY{n}{test\PYZus{}target}\PY{p}{,} \PY{n}{RDbest1}\PY{o}{.}\PY{n}{predict}\PY{p}{(}\PY{n}{test\PYZus{}features}\PY{p}{)}\PY{p}{)}\PY{p}{)}
\PY{n+nb}{print}\PY{p}{(}\PY{l+s+s2}{\PYZdq{}}\PY{l+s+s2}{Test score f1:}\PY{l+s+s2}{\PYZdq{}}\PY{p}{,}\PY{n}{test\PYZus{}rfss}\PY{p}{)}
\end{Verbatim}
\end{tcolorbox}

    \begin{Verbatim}[commandchars=\\\{\}]
              precision    recall  f1-score   support

           0       1.00      1.00      1.00    263288
           1       0.15      0.14      0.15       855

    accuracy                           0.99    264143
   macro avg       0.58      0.57      0.57    264143
weighted avg       0.99      0.99      0.99    264143

Test score f1: 0.14661
    \end{Verbatim}

    \begin{tcolorbox}[breakable, size=fbox, boxrule=1pt, pad at break*=1mm,colback=cellbackground, colframe=cellborder]
\prompt{In}{incolor}{31}{\boxspacing}
\begin{Verbatim}[commandchars=\\\{\}]
\PY{c+c1}{\PYZsh{}Courbe ROC}
\PY{n}{fpr}\PY{p}{,} \PY{n}{tpr}\PY{p}{,} \PY{n}{\PYZus{}} \PY{o}{=} \PY{n}{roc\PYZus{}curve}\PY{p}{(}\PY{n}{test\PYZus{}target}\PY{p}{,}\PY{n}{RDbest1}\PY{o}{.}\PY{n}{predict}\PY{p}{(}\PY{n}{test\PYZus{}features}\PY{p}{)}\PY{p}{)}
\PY{n}{auc\PYZus{}rfss} \PY{o}{=} \PY{n}{metrics}\PY{o}{.}\PY{n}{roc\PYZus{}auc\PYZus{}score}\PY{p}{(}\PY{n}{test\PYZus{}target}\PY{p}{,}\PY{n}{RDbest1}\PY{o}{.}\PY{n}{predict}\PY{p}{(}\PY{n}{test\PYZus{}features}\PY{p}{)}\PY{p}{)}
\PY{n}{plt}\PY{o}{.}\PY{n}{plot}\PY{p}{(}\PY{n}{fpr}\PY{p}{,}\PY{n}{tpr}\PY{p}{,}\PY{n}{label}\PY{o}{=}\PY{l+s+s2}{\PYZdq{}}\PY{l+s+s2}{data 1, auc=}\PY{l+s+s2}{\PYZdq{}}\PY{o}{+}\PY{n+nb}{str}\PY{p}{(}\PY{n}{auc\PYZus{}rfss}\PY{p}{)}\PY{p}{)}
\PY{n}{plt}\PY{o}{.}\PY{n}{legend}\PY{p}{(}\PY{n}{loc}\PY{o}{=}\PY{l+m+mi}{4}\PY{p}{)}
\PY{n}{plt}\PY{o}{.}\PY{n}{show}\PY{p}{(}\PY{p}{)}
\end{Verbatim}
\end{tcolorbox}

    \begin{center}
    \adjustimage{max size={0.9\linewidth}{0.9\paperheight}}{output_149_0.png}
    \end{center}
    { \hspace*{\fill} \\}
    
    Le AUC est à 0,56

    \begin{tcolorbox}[breakable, size=fbox, boxrule=1pt, pad at break*=1mm,colback=cellbackground, colframe=cellborder]
\prompt{In}{incolor}{173}{\boxspacing}
\begin{Verbatim}[commandchars=\\\{\}]
\PY{c+c1}{\PYZsh{}Ajout dictionnaire}
\PY{n}{test\PYZus{}result}\PY{p}{[}\PY{l+s+s1}{\PYZsq{}}\PY{l+s+s1}{RF}\PY{l+s+s1}{\PYZsq{}}\PY{p}{]} \PY{o}{=} \PY{n}{test\PYZus{}rfss}
\PY{n}{valid\PYZus{}result}\PY{p}{[}\PY{l+s+s1}{\PYZsq{}}\PY{l+s+s1}{RF}\PY{l+s+s1}{\PYZsq{}}\PY{p}{]} \PY{o}{=} \PY{n}{f1\PYZus{}rfss}
\PY{n}{auc\PYZus{}result}\PY{p}{[}\PY{l+s+s1}{\PYZsq{}}\PY{l+s+s1}{RF}\PY{l+s+s1}{\PYZsq{}}\PY{p}{]} \PY{o}{=} \PY{n+nb}{round}\PY{p}{(}\PY{n}{auc\PYZus{}rfss}\PY{p}{,}\PY{l+m+mi}{3}\PY{p}{)}
\PY{n}{time\PYZus{}result}\PY{p}{[}\PY{l+s+s1}{\PYZsq{}}\PY{l+s+s1}{RF}\PY{l+s+s1}{\PYZsq{}}\PY{p}{]} \PY{o}{=} \PY{n+nb}{round}\PY{p}{(}\PY{n}{tpsRF}\PY{p}{,}\PY{l+m+mi}{3}\PY{p}{)}
\PY{n+nb}{print}\PY{p}{(}\PY{n}{test\PYZus{}result}\PY{p}{)}
\PY{n+nb}{print}\PY{p}{(}\PY{n}{valid\PYZus{}result}\PY{p}{)}
\PY{n+nb}{print}\PY{p}{(}\PY{n}{auc\PYZus{}result}\PY{p}{)}
\PY{n+nb}{print}\PY{p}{(}\PY{n}{time\PYZus{}result}\PY{p}{)}
\end{Verbatim}
\end{tcolorbox}

    \begin{Verbatim}[commandchars=\\\{\}]
\{'KNN': 0.039632, 'DT': 0.080354, 'ADL': 0.031711, 'XGB': 0.080354, 'ADXG':
0.083075, 'RFSM': 0.065744, 'RF': 0.14661, 'IF': 0.129272\}
\{'KNN': 0.048721, 'DT': 0.030846, 'ADL': 0.029688, 'XGB': 0.085355, 'ADXG':
0.106159, 'RFSM': 0.093918, 'RF': 0.157978, 'IF': 0.031811\}
\{'KNN': 0.536, 'DT': 0.512, 'ADL': 0.681, 'XGB': 0.61, 'ADXG': 0.538, 'RFSM':
0.522, 'RF': 0.569, 'IF': None\}
\{'KNN': 85.574, 'DT': 63.411, 'ADL': 5.1, 'XGB': 0.756, 'ADXG': 1.934, 'RFSM':
48.063, 'RF': 9.095, 'IF': 0.48593333333333333\}
    \end{Verbatim}

    \hypertarget{one-class-classification-isolation-forest}{%
\subsubsection{7.8 One class Classification Isolation
forest}\label{one-class-classification-isolation-forest}}

Le dernier type de modèle que nous avons voulu essayer est un algorithme
de type One-Class Classification. Ces algorithmes on pour objectif de
détecter des anomalies dans les données en les isolants. Ils sont
adaptés au données déséquilibrés comportant aucune ou très peu de
données de la classe positif. Ce qui est le cas dans notre problème.
Jusqu'a présent nos meilleurs résultat ont été obtenu avec la forêt
aléatoire. Nous avons donc choisit le modèle isolation forest, qui
repose également sur une représentation par arbre. C'est un algorithme
d'apprentissage non-supervisé.

    \begin{tcolorbox}[breakable, size=fbox, boxrule=1pt, pad at break*=1mm,colback=cellbackground, colframe=cellborder]
\prompt{In}{incolor}{88}{\boxspacing}
\begin{Verbatim}[commandchars=\\\{\}]
\PY{k+kn}{from} \PY{n+nn}{sklearn}\PY{n+nn}{.}\PY{n+nn}{metrics} \PY{k+kn}{import} \PY{n}{f1\PYZus{}score}
\PY{k+kn}{from} \PY{n+nn}{sklearn}\PY{n+nn}{.}\PY{n+nn}{ensemble} \PY{k+kn}{import} \PY{n}{IsolationForest}

\PY{n}{start} \PY{o}{=} \PY{n}{time}\PY{p}{(}\PY{p}{)}
\PY{c+c1}{\PYZsh{} define outlier detection model}
\PY{n}{model} \PY{o}{=} \PY{n}{make\PYZus{}pipeline}\PY{p}{(}\PY{n}{StandardScaler}\PY{p}{(}\PY{p}{)}\PY{p}{,}\PY{n}{IsolationForest}\PY{p}{(}\PY{n}{contamination}\PY{o}{=}\PY{l+s+s1}{\PYZsq{}}\PY{l+s+s1}{auto}\PY{l+s+s1}{\PYZsq{}}\PY{p}{)}\PY{p}{)}
\PY{n}{trainX} \PY{o}{=} \PY{n}{x\PYZus{}train}\PY{p}{[}\PY{n}{y\PYZus{}train}\PY{o}{==}\PY{l+m+mi}{0}\PY{p}{]}
\PY{n}{model}\PY{o}{.}\PY{n}{fit}\PY{p}{(}\PY{n}{trainX}\PY{p}{)}
\PY{n}{ypred} \PY{o}{=} \PY{n}{model}\PY{o}{.}\PY{n}{predict}\PY{p}{(}\PY{n}{x\PYZus{}val}\PY{p}{)}
\PY{n}{newy} \PY{o}{=} \PY{n}{y\PYZus{}val}
\PY{n}{newy}\PY{p}{[}\PY{n}{newy}\PY{o}{==}\PY{l+m+mi}{1}\PY{p}{]} \PY{o}{=} \PY{o}{\PYZhy{}}\PY{l+m+mi}{1} 
\PY{n}{newy}\PY{p}{[}\PY{n}{newy}\PY{o}{==}\PY{l+m+mi}{0}\PY{p}{]}\PY{o}{=}\PY{l+m+mi}{1}
\PY{n}{done} \PY{o}{=} \PY{n}{time}\PY{p}{(}\PY{p}{)}
\PY{n}{IR} \PY{o}{=} \PY{n+nb}{round}\PY{p}{(}\PY{n}{done} \PY{o}{\PYZhy{}} \PY{n}{start}\PY{p}{,}\PY{l+m+mi}{3}\PY{p}{)}\PY{o}{/}\PY{l+m+mi}{60}
\PY{n+nb}{print}\PY{p}{(}\PY{l+s+s2}{\PYZdq{}}\PY{l+s+s2}{temps de calcul:}\PY{l+s+s2}{\PYZdq{}}\PY{p}{,}\PY{n}{IR}\PY{p}{)}
\end{Verbatim}
\end{tcolorbox}

    \begin{Verbatim}[commandchars=\\\{\}]
temps de calcul: 0.48593333333333333
    \end{Verbatim}

    \begin{Verbatim}[commandchars=\\\{\}]
<ipython-input-88-aca5f0dfdd4b>:11: SettingWithCopyWarning:
A value is trying to be set on a copy of a slice from a DataFrame

See the caveats in the documentation: https://pandas.pydata.org/pandas-
docs/stable/user\_guide/indexing.html\#returning-a-view-versus-a-copy
  newy[newy==1] = -1
/Users/hoangkhanhle/opt/anaconda3/lib/python3.8/site-
packages/pandas/core/generic.py:8765: SettingWithCopyWarning:
A value is trying to be set on a copy of a slice from a DataFrame

See the caveats in the documentation: https://pandas.pydata.org/pandas-
docs/stable/user\_guide/indexing.html\#returning-a-view-versus-a-copy
  self.\_update\_inplace(new\_data)
<ipython-input-88-aca5f0dfdd4b>:12: SettingWithCopyWarning:
A value is trying to be set on a copy of a slice from a DataFrame

See the caveats in the documentation: https://pandas.pydata.org/pandas-
docs/stable/user\_guide/indexing.html\#returning-a-view-versus-a-copy
  newy[newy==0]=1
    \end{Verbatim}

    \hypertarget{validation}{%
\subparagraph{Validation :}\label{validation}}

    \begin{tcolorbox}[breakable, size=fbox, boxrule=1pt, pad at break*=1mm,colback=cellbackground, colframe=cellborder]
\prompt{In}{incolor}{82}{\boxspacing}
\begin{Verbatim}[commandchars=\\\{\}]
\PY{c+c1}{\PYZsh{}Validation}
\PY{n}{f1\PYZus{}iforest} \PY{o}{=} \PY{n+nb}{round}\PY{p}{(}\PY{n}{f1\PYZus{}score}\PY{p}{(}\PY{n}{newy}\PY{p}{,}\PY{n}{ypred}\PY{p}{,} \PY{n}{pos\PYZus{}label}\PY{o}{=}\PY{o}{\PYZhy{}}\PY{l+m+mi}{1}\PY{p}{)}\PY{p}{,}\PY{l+m+mi}{6}\PY{p}{)}
\PY{n+nb}{print}\PY{p}{(}\PY{n}{metrics}\PY{o}{.}\PY{n}{classification\PYZus{}report}\PY{p}{(}\PY{n}{newy}\PY{p}{,} \PY{n}{ypred}\PY{p}{)}\PY{p}{)}
\PY{n+nb}{print}\PY{p}{(}\PY{l+s+s1}{\PYZsq{}}\PY{l+s+s1}{Validation F1 Score:}\PY{l+s+s1}{\PYZsq{}} \PY{p}{,}\PY{n}{f1\PYZus{}iforest}\PY{p}{)}
\end{Verbatim}
\end{tcolorbox}

    \begin{Verbatim}[commandchars=\\\{\}]
              precision    recall  f1-score   support

          -1       0.02      0.40      0.03       948
           1       1.00      0.92      0.96    288737

    accuracy                           0.92    289685
   macro avg       0.51      0.66      0.50    289685
weighted avg       0.99      0.92      0.96    289685

Validation F1 Score: 0.031811
    \end{Verbatim}

    \hypertarget{test}{%
\subparagraph{Test :}\label{test}}

    \begin{tcolorbox}[breakable, size=fbox, boxrule=1pt, pad at break*=1mm,colback=cellbackground, colframe=cellborder]
\prompt{In}{incolor}{85}{\boxspacing}
\begin{Verbatim}[commandchars=\\\{\}]
\PY{c+c1}{\PYZsh{}Matrice de confusion}
\PY{n}{IR\PYZus{}matrix\PYZus{}test} \PY{o}{=} \PY{n}{confusion\PYZus{}matrix}\PY{p}{(}\PY{n}{newy\PYZus{}test}\PY{p}{,}\PY{n}{ypred\PYZus{}test}\PY{p}{)}
\PY{n}{group\PYZus{}names} \PY{o}{=} \PY{p}{[}\PY{l+s+s1}{\PYZsq{}}\PY{l+s+s1}{True Neg}\PY{l+s+s1}{\PYZsq{}}\PY{p}{,}\PY{l+s+s1}{\PYZsq{}}\PY{l+s+s1}{False Pos}\PY{l+s+s1}{\PYZsq{}}\PY{p}{,}\PY{l+s+s1}{\PYZsq{}}\PY{l+s+s1}{False Neg}\PY{l+s+s1}{\PYZsq{}}\PY{p}{,}\PY{l+s+s1}{\PYZsq{}}\PY{l+s+s1}{True Pos}\PY{l+s+s1}{\PYZsq{}}\PY{p}{]}
\PY{n}{group\PYZus{}counts} \PY{o}{=} \PY{p}{[}\PY{l+s+s2}{\PYZdq{}}\PY{l+s+si}{\PYZob{}0:0.0f\PYZcb{}}\PY{l+s+s2}{\PYZdq{}}\PY{o}{.}\PY{n}{format}\PY{p}{(}\PY{n}{value}\PY{p}{)} \PY{k}{for} \PY{n}{value} \PY{o+ow}{in}
                \PY{n}{IR\PYZus{}matrix\PYZus{}test}\PY{o}{.}\PY{n}{flatten}\PY{p}{(}\PY{p}{)}\PY{p}{]}
\PY{n}{group\PYZus{}percentages} \PY{o}{=} \PY{p}{[}\PY{l+s+s2}{\PYZdq{}}\PY{l+s+si}{\PYZob{}0:.2\PYZpc{}\PYZcb{}}\PY{l+s+s2}{\PYZdq{}}\PY{o}{.}\PY{n}{format}\PY{p}{(}\PY{n}{value}\PY{p}{)} \PY{k}{for} \PY{n}{value} \PY{o+ow}{in}
                     \PY{n}{IR\PYZus{}matrix\PYZus{}test}\PY{o}{.}\PY{n}{flatten}\PY{p}{(}\PY{p}{)}\PY{o}{/}\PY{n}{np}\PY{o}{.}\PY{n}{sum}\PY{p}{(}\PY{n}{IR\PYZus{}matrix\PYZus{}test}\PY{p}{)}\PY{p}{]}
\PY{n}{labels} \PY{o}{=} \PY{p}{[}\PY{l+s+sa}{f}\PY{l+s+s2}{\PYZdq{}}\PY{l+s+si}{\PYZob{}}\PY{n}{v1}\PY{l+s+si}{\PYZcb{}}\PY{l+s+se}{\PYZbs{}n}\PY{l+s+si}{\PYZob{}}\PY{n}{v2}\PY{l+s+si}{\PYZcb{}}\PY{l+s+se}{\PYZbs{}n}\PY{l+s+si}{\PYZob{}}\PY{n}{v3}\PY{l+s+si}{\PYZcb{}}\PY{l+s+s2}{\PYZdq{}} \PY{k}{for} \PY{n}{v1}\PY{p}{,} \PY{n}{v2}\PY{p}{,} \PY{n}{v3} \PY{o+ow}{in}
          \PY{n+nb}{zip}\PY{p}{(}\PY{n}{group\PYZus{}names}\PY{p}{,}\PY{n}{group\PYZus{}counts}\PY{p}{,}\PY{n}{group\PYZus{}percentages}\PY{p}{)}\PY{p}{]}
\PY{n}{labels} \PY{o}{=} \PY{n}{np}\PY{o}{.}\PY{n}{asarray}\PY{p}{(}\PY{n}{labels}\PY{p}{)}\PY{o}{.}\PY{n}{reshape}\PY{p}{(}\PY{l+m+mi}{2}\PY{p}{,}\PY{l+m+mi}{2}\PY{p}{)}
\PY{n}{sns}\PY{o}{.}\PY{n}{heatmap}\PY{p}{(}\PY{n}{IR\PYZus{}matrix\PYZus{}test}\PY{p}{,} \PY{n}{annot}\PY{o}{=}\PY{n}{labels}\PY{p}{,} \PY{n}{fmt}\PY{o}{=}\PY{l+s+s1}{\PYZsq{}}\PY{l+s+s1}{\PYZsq{}}\PY{p}{,} \PY{n}{cmap}\PY{o}{=}\PY{l+s+s1}{\PYZsq{}}\PY{l+s+s1}{Blues}\PY{l+s+s1}{\PYZsq{}}\PY{p}{)}
\end{Verbatim}
\end{tcolorbox}

            \begin{tcolorbox}[breakable, size=fbox, boxrule=.5pt, pad at break*=1mm, opacityfill=0]
\prompt{Out}{outcolor}{85}{\boxspacing}
\begin{Verbatim}[commandchars=\\\{\}]
<matplotlib.axes.\_subplots.AxesSubplot at 0x7fe5009fd760>
\end{Verbatim}
\end{tcolorbox}
        
    \begin{center}
    \adjustimage{max size={0.9\linewidth}{0.9\paperheight}}{output_157_1.png}
    \end{center}
    { \hspace*{\fill} \\}
    
    \begin{tcolorbox}[breakable, size=fbox, boxrule=1pt, pad at break*=1mm,colback=cellbackground, colframe=cellborder]
\prompt{In}{incolor}{84}{\boxspacing}
\begin{Verbatim}[commandchars=\\\{\}]
\PY{c+c1}{\PYZsh{}test}
\PY{n}{newy\PYZus{}test} \PY{o}{=} \PY{n}{test\PYZus{}target}
\PY{n}{newy\PYZus{}test}\PY{p}{[}\PY{n}{newy\PYZus{}test}\PY{o}{==}\PY{l+m+mi}{1}\PY{p}{]} \PY{o}{=} \PY{o}{\PYZhy{}}\PY{l+m+mi}{1} 
\PY{n}{newy\PYZus{}test}\PY{p}{[}\PY{n}{newy\PYZus{}test}\PY{o}{==}\PY{l+m+mi}{0}\PY{p}{]}\PY{o}{=}\PY{l+m+mi}{1}
\PY{n}{ypred\PYZus{}test} \PY{o}{=} \PY{n}{model}\PY{o}{.}\PY{n}{predict}\PY{p}{(}\PY{n}{test\PYZus{}features}\PY{p}{)}
\PY{n}{test\PYZus{}iforest} \PY{o}{=} \PY{n+nb}{round}\PY{p}{(}\PY{n}{f1\PYZus{}score}\PY{p}{(}\PY{n}{newy\PYZus{}test}\PY{p}{,}\PY{n}{ypred\PYZus{}test}\PY{p}{,} \PY{n}{pos\PYZus{}label}\PY{o}{=}\PY{o}{\PYZhy{}}\PY{l+m+mi}{1}\PY{p}{)}\PY{p}{,}\PY{l+m+mi}{6}\PY{p}{)}
\PY{n+nb}{print}\PY{p}{(}\PY{n}{metrics}\PY{o}{.}\PY{n}{classification\PYZus{}report}\PY{p}{(}\PY{n}{newy\PYZus{}test}\PY{p}{,}\PY{n}{ypred\PYZus{}test}\PY{p}{)}\PY{p}{)}
\PY{n+nb}{print}\PY{p}{(}\PY{l+s+s1}{\PYZsq{}}\PY{l+s+s1}{Validation F1 Score:}\PY{l+s+s1}{\PYZsq{}} \PY{p}{,}\PY{n}{test\PYZus{}iforest}\PY{p}{)}
\end{Verbatim}
\end{tcolorbox}

    \begin{Verbatim}[commandchars=\\\{\}]
<ipython-input-84-6557a4093bd3>:3: SettingWithCopyWarning:
A value is trying to be set on a copy of a slice from a DataFrame

See the caveats in the documentation: https://pandas.pydata.org/pandas-
docs/stable/user\_guide/indexing.html\#returning-a-view-versus-a-copy
  newy\_test[newy\_test==1] = -1
/Users/hoangkhanhle/opt/anaconda3/lib/python3.8/site-
packages/pandas/core/generic.py:8765: SettingWithCopyWarning:
A value is trying to be set on a copy of a slice from a DataFrame

See the caveats in the documentation: https://pandas.pydata.org/pandas-
docs/stable/user\_guide/indexing.html\#returning-a-view-versus-a-copy
  self.\_update\_inplace(new\_data)
<ipython-input-84-6557a4093bd3>:4: SettingWithCopyWarning:
A value is trying to be set on a copy of a slice from a DataFrame

See the caveats in the documentation: https://pandas.pydata.org/pandas-
docs/stable/user\_guide/indexing.html\#returning-a-view-versus-a-copy
  newy\_test[newy\_test==0]=1
    \end{Verbatim}

    \begin{Verbatim}[commandchars=\\\{\}]
              precision    recall  f1-score   support

          -1       1.00      0.07      0.13    264143
           1       0.00      0.00      0.00         0

    accuracy                           0.07    264143
   macro avg       0.50      0.03      0.06    264143
weighted avg       1.00      0.07      0.13    264143

Validation F1 Score: 0.129272
    \end{Verbatim}

    \begin{Verbatim}[commandchars=\\\{\}]
/Users/hoangkhanhle/opt/anaconda3/lib/python3.8/site-
packages/sklearn/metrics/\_classification.py:1221: UndefinedMetricWarning: Recall
and F-score are ill-defined and being set to 0.0 in labels with no true samples.
Use `zero\_division` parameter to control this behavior.
  \_warn\_prf(average, modifier, msg\_start, len(result))
    \end{Verbatim}

    A cause de l'isolation de la classe 1 donc nous n'avons pas de courbe de
ROC

    Nous pouvons voir que il y a une grande différence entre le F1 de test
et val donc le modèle n'est pas très fiable

    \begin{tcolorbox}[breakable, size=fbox, boxrule=1pt, pad at break*=1mm,colback=cellbackground, colframe=cellborder]
\prompt{In}{incolor}{200}{\boxspacing}
\begin{Verbatim}[commandchars=\\\{\}]
\PY{c+c1}{\PYZsh{}Ajout dans le dico}
\PY{n}{test\PYZus{}result}\PY{p}{[}\PY{l+s+s1}{\PYZsq{}}\PY{l+s+s1}{IF}\PY{l+s+s1}{\PYZsq{}}\PY{p}{]} \PY{o}{=} \PY{n}{test\PYZus{}iforest}
\PY{n}{valid\PYZus{}result}\PY{p}{[}\PY{l+s+s1}{\PYZsq{}}\PY{l+s+s1}{IF}\PY{l+s+s1}{\PYZsq{}}\PY{p}{]} \PY{o}{=} \PY{n}{f1\PYZus{}iforest}
\PY{n}{auc\PYZus{}result}\PY{p}{[}\PY{l+s+s1}{\PYZsq{}}\PY{l+s+s1}{IF}\PY{l+s+s1}{\PYZsq{}}\PY{p}{]} \PY{o}{=} \PY{l+m+mi}{0}
\PY{n}{time\PYZus{}result}\PY{p}{[}\PY{l+s+s1}{\PYZsq{}}\PY{l+s+s1}{IF}\PY{l+s+s1}{\PYZsq{}}\PY{p}{]} \PY{o}{=} \PY{n+nb}{round}\PY{p}{(}\PY{n}{IR}\PY{p}{,}\PY{l+m+mi}{3}\PY{p}{)}
\PY{n+nb}{print}\PY{p}{(}\PY{n}{test\PYZus{}result}\PY{p}{)}
\PY{n+nb}{print}\PY{p}{(}\PY{n}{valid\PYZus{}result}\PY{p}{)}
\PY{n+nb}{print}\PY{p}{(}\PY{n}{auc\PYZus{}result}\PY{p}{)}
\PY{n+nb}{print}\PY{p}{(}\PY{n}{time\PYZus{}result}\PY{p}{)}
\end{Verbatim}
\end{tcolorbox}

    \begin{Verbatim}[commandchars=\\\{\}]
\{'KNN': 0.039632, 'DT': 0.080354, 'ADL': 0.031711, 'XGB': 0.080354, 'ADXG':
0.083075, 'RFSM': 0.065744, 'RF': 0.14661, 'IF': 0.129272\}
\{'KNN': 0.048721, 'DT': 0.030846, 'ADL': 0.029688, 'XGB': 0.085355, 'ADXG':
0.106159, 'RFSM': 0.093918, 'RF': 0.157978, 'IF': 0.031811\}
\{'KNN': 0.536, 'DT': 0.512, 'ADL': 0.681, 'XGB': 0.61, 'ADXG': 0.538, 'RFSM':
0.522, 'RF': 0.569, 'IF': 0\}
\{'KNN': 85.574, 'DT': 63.411, 'ADL': 5.1, 'XGB': 0.756, 'ADXG': 1.934, 'RFSM':
48.063, 'RF': 9.095, 'IF': 0.486\}
    \end{Verbatim}

    \hypertarget{ruxe9sultats-et-comparaison}{%
\subsection{8. Résultats et
comparaison}\label{ruxe9sultats-et-comparaison}}

    \hypertarget{f1---validation}{%
\subparagraph{F1 - Validation :}\label{f1---validation}}

    \begin{tcolorbox}[breakable, size=fbox, boxrule=1pt, pad at break*=1mm,colback=cellbackground, colframe=cellborder]
\prompt{In}{incolor}{185}{\boxspacing}
\begin{Verbatim}[commandchars=\\\{\}]
\PY{c+c1}{\PYZsh{}barre plot validation}
\PY{n}{tab\PYZus{}valid}\PY{o}{=}\PY{n}{OrderedDict}\PY{p}{(}\PY{n+nb}{sorted}\PY{p}{(}\PY{n}{valid\PYZus{}result}\PY{o}{.}\PY{n}{items}\PY{p}{(}\PY{p}{)}\PY{p}{,} \PY{n}{key}\PY{o}{=}\PY{k}{lambda} \PY{n}{t}\PY{p}{:} \PY{n}{t}\PY{p}{[}\PY{l+m+mi}{1}\PY{p}{]}\PY{p}{,} \PY{n}{reverse}\PY{o}{=}\PY{k+kc}{True}\PY{p}{)}\PY{p}{)}
\PY{n+nb}{print}\PY{p}{(}\PY{n}{tab\PYZus{}valid}\PY{p}{)}
\PY{n}{plt}\PY{o}{.}\PY{n}{bar}\PY{p}{(}\PY{n}{tab\PYZus{}valid}\PY{o}{.}\PY{n}{keys}\PY{p}{(}\PY{p}{)}\PY{p}{,}\PY{n}{tab\PYZus{}valid}\PY{o}{.}\PY{n}{values}\PY{p}{(}\PY{p}{)}\PY{p}{,}\PY{n}{color}\PY{o}{=}\PY{p}{[}\PY{l+s+s1}{\PYZsq{}}\PY{l+s+s1}{\PYZsh{}0072BD}\PY{l+s+s1}{\PYZsq{}}\PY{p}{,}\PY{l+s+s1}{\PYZsq{}}\PY{l+s+s1}{\PYZsh{}D95319}\PY{l+s+s1}{\PYZsq{}}\PY{p}{,}\PY{l+s+s1}{\PYZsq{}}\PY{l+s+s1}{\PYZsh{}EDB120}\PY{l+s+s1}{\PYZsq{}}\PY{p}{,}\PY{l+s+s1}{\PYZsq{}}\PY{l+s+s1}{\PYZsh{}7E2F8E}\PY{l+s+s1}{\PYZsq{}}\PY{p}{,}\PY{l+s+s1}{\PYZsq{}}\PY{l+s+s1}{\PYZsh{}77AC30}\PY{l+s+s1}{\PYZsq{}}\PY{p}{,}\PY{l+s+s1}{\PYZsq{}}\PY{l+s+s1}{\PYZsh{}4DBEEE}\PY{l+s+s1}{\PYZsq{}}\PY{p}{,}\PY{l+s+s1}{\PYZsq{}}\PY{l+s+s1}{\PYZsh{}A2142F}\PY{l+s+s1}{\PYZsq{}}\PY{p}{]}\PY{p}{)}
\end{Verbatim}
\end{tcolorbox}

    \begin{Verbatim}[commandchars=\\\{\}]
OrderedDict([('RF', 0.157978), ('ADXG', 0.106159), ('RFSM', 0.093918), ('XGB',
0.085355), ('KNN', 0.048721), ('IF', 0.031811), ('DT', 0.030846), ('ADL',
0.029688)])
    \end{Verbatim}

            \begin{tcolorbox}[breakable, size=fbox, boxrule=.5pt, pad at break*=1mm, opacityfill=0]
\prompt{Out}{outcolor}{185}{\boxspacing}
\begin{Verbatim}[commandchars=\\\{\}]
<BarContainer object of 8 artists>
\end{Verbatim}
\end{tcolorbox}
        
    \begin{center}
    \adjustimage{max size={0.9\linewidth}{0.9\paperheight}}{output_164_2.png}
    \end{center}
    { \hspace*{\fill} \\}
    
    \hypertarget{f1---test}{%
\subparagraph{F1 - Test :}\label{f1---test}}

    \begin{tcolorbox}[breakable, size=fbox, boxrule=1pt, pad at break*=1mm,colback=cellbackground, colframe=cellborder]
\prompt{In}{incolor}{187}{\boxspacing}
\begin{Verbatim}[commandchars=\\\{\}]
\PY{c+c1}{\PYZsh{}barre plot test}
\PY{n}{tab\PYZus{}test}\PY{o}{=}\PY{n}{OrderedDict}\PY{p}{(}\PY{n+nb}{sorted}\PY{p}{(}\PY{n}{test\PYZus{}result}\PY{o}{.}\PY{n}{items}\PY{p}{(}\PY{p}{)}\PY{p}{,} \PY{n}{key}\PY{o}{=}\PY{k}{lambda} \PY{n}{t}\PY{p}{:} \PY{n}{t}\PY{p}{[}\PY{l+m+mi}{1}\PY{p}{]}\PY{p}{,} \PY{n}{reverse}\PY{o}{=}\PY{k+kc}{True}\PY{p}{)}\PY{p}{)}
\PY{n+nb}{print}\PY{p}{(}\PY{n}{tab\PYZus{}test}\PY{p}{)}
\PY{n}{plt}\PY{o}{.}\PY{n}{bar}\PY{p}{(}\PY{n}{tab\PYZus{}test}\PY{o}{.}\PY{n}{keys}\PY{p}{(}\PY{p}{)}\PY{p}{,}\PY{n}{tab\PYZus{}test}\PY{o}{.}\PY{n}{values}\PY{p}{(}\PY{p}{)}\PY{p}{,}\PY{n}{color}\PY{o}{=}\PY{p}{[}\PY{l+s+s1}{\PYZsq{}}\PY{l+s+s1}{\PYZsh{}0072BD}\PY{l+s+s1}{\PYZsq{}}\PY{p}{,}\PY{l+s+s1}{\PYZsq{}}\PY{l+s+s1}{\PYZsh{}D95319}\PY{l+s+s1}{\PYZsq{}}\PY{p}{,}\PY{l+s+s1}{\PYZsq{}}\PY{l+s+s1}{\PYZsh{}EDB120}\PY{l+s+s1}{\PYZsq{}}\PY{p}{,}\PY{l+s+s1}{\PYZsq{}}\PY{l+s+s1}{\PYZsh{}7E2F8E}\PY{l+s+s1}{\PYZsq{}}\PY{p}{,}\PY{l+s+s1}{\PYZsq{}}\PY{l+s+s1}{\PYZsh{}77AC30}\PY{l+s+s1}{\PYZsq{}}\PY{p}{,}\PY{l+s+s1}{\PYZsq{}}\PY{l+s+s1}{\PYZsh{}4DBEEE}\PY{l+s+s1}{\PYZsq{}}\PY{p}{,}\PY{l+s+s1}{\PYZsq{}}\PY{l+s+s1}{\PYZsh{}A2142F}\PY{l+s+s1}{\PYZsq{}}\PY{p}{]}\PY{p}{)}
\end{Verbatim}
\end{tcolorbox}

    \begin{Verbatim}[commandchars=\\\{\}]
OrderedDict([('RF', 0.14661), ('IF', 0.129272), ('ADXG', 0.083075), ('DT',
0.080354), ('XGB', 0.080354), ('RFSM', 0.065744), ('KNN', 0.039632), ('ADL',
0.031711)])
    \end{Verbatim}

            \begin{tcolorbox}[breakable, size=fbox, boxrule=.5pt, pad at break*=1mm, opacityfill=0]
\prompt{Out}{outcolor}{187}{\boxspacing}
\begin{Verbatim}[commandchars=\\\{\}]
<BarContainer object of 8 artists>
\end{Verbatim}
\end{tcolorbox}
        
    \begin{center}
    \adjustimage{max size={0.9\linewidth}{0.9\paperheight}}{output_166_2.png}
    \end{center}
    { \hspace*{\fill} \\}
    
    \begin{tcolorbox}[breakable, size=fbox, boxrule=1pt, pad at break*=1mm,colback=cellbackground, colframe=cellborder]
\prompt{In}{incolor}{196}{\boxspacing}
\begin{Verbatim}[commandchars=\\\{\}]
\PY{n}{M} \PY{o}{=} \PY{n}{np}\PY{o}{.}\PY{n}{arange}\PY{p}{(}\PY{n+nb}{len}\PY{p}{(}\PY{n}{tab\PYZus{}test}\PY{p}{)}\PY{p}{)}
\PY{n}{ax} \PY{o}{=} \PY{n}{plt}\PY{o}{.}\PY{n}{subplot}\PY{p}{(}\PY{l+m+mi}{111}\PY{p}{)}
\PY{n}{ax}\PY{o}{.}\PY{n}{bar}\PY{p}{(}\PY{n}{M}\PY{p}{,} \PY{n}{tab\PYZus{}test}\PY{o}{.}\PY{n}{values}\PY{p}{(}\PY{p}{)}\PY{p}{,} \PY{n}{width}\PY{o}{=}\PY{l+m+mf}{0.4}\PY{p}{,} \PY{n}{color}\PY{o}{=}\PY{l+s+s1}{\PYZsq{}}\PY{l+s+s1}{\PYZsh{}D95319}\PY{l+s+s1}{\PYZsq{}}\PY{p}{,} \PY{n}{align}\PY{o}{=}\PY{l+s+s1}{\PYZsq{}}\PY{l+s+s1}{center}\PY{l+s+s1}{\PYZsq{}}\PY{p}{)}
\PY{n}{ax}\PY{o}{.}\PY{n}{bar}\PY{p}{(}\PY{n}{M}\PY{o}{\PYZhy{}}\PY{l+m+mf}{0.4}\PY{p}{,} \PY{n}{tab\PYZus{}valid}\PY{o}{.}\PY{n}{values}\PY{p}{(}\PY{p}{)}\PY{p}{,} \PY{n}{width}\PY{o}{=}\PY{l+m+mf}{0.4}\PY{p}{,} \PY{n}{color}\PY{o}{=}\PY{l+s+s1}{\PYZsq{}}\PY{l+s+s1}{\PYZsh{}4DBEEE}\PY{l+s+s1}{\PYZsq{}}\PY{p}{,} \PY{n}{align}\PY{o}{=}\PY{l+s+s1}{\PYZsq{}}\PY{l+s+s1}{center}\PY{l+s+s1}{\PYZsq{}}\PY{p}{)}
\PY{n}{ax}\PY{o}{.}\PY{n}{legend}\PY{p}{(}\PY{p}{(}\PY{l+s+s1}{\PYZsq{}}\PY{l+s+s1}{F1 Validation}\PY{l+s+s1}{\PYZsq{}}\PY{p}{,}\PY{l+s+s1}{\PYZsq{}}\PY{l+s+s1}{F1 Test}\PY{l+s+s1}{\PYZsq{}}\PY{p}{)}\PY{p}{)}
\PY{n}{plt}\PY{o}{.}\PY{n}{xticks}\PY{p}{(}\PY{n}{M}\PY{p}{,} \PY{n}{tab\PYZus{}test}\PY{o}{.}\PY{n}{keys}\PY{p}{(}\PY{p}{)}\PY{p}{)}
\PY{n}{plt}\PY{o}{.}\PY{n}{title}\PY{p}{(}\PY{l+s+s2}{\PYZdq{}}\PY{l+s+s2}{F1 Measure}\PY{l+s+s2}{\PYZdq{}}\PY{p}{,} \PY{n}{fontsize}\PY{o}{=}\PY{l+m+mi}{17}\PY{p}{)}
\PY{n}{plt}\PY{o}{.}\PY{n}{show}\PY{p}{(}\PY{p}{)}
\end{Verbatim}
\end{tcolorbox}

    \begin{center}
    \adjustimage{max size={0.9\linewidth}{0.9\paperheight}}{output_167_0.png}
    \end{center}
    { \hspace*{\fill} \\}
    
    \hypertarget{auc}{%
\subparagraph{AUC :}\label{auc}}

    \begin{tcolorbox}[breakable, size=fbox, boxrule=1pt, pad at break*=1mm,colback=cellbackground, colframe=cellborder]
\prompt{In}{incolor}{202}{\boxspacing}
\begin{Verbatim}[commandchars=\\\{\}]
\PY{c+c1}{\PYZsh{}barre plot test}
\PY{n}{tab\PYZus{}AUC}\PY{o}{=}\PY{n}{OrderedDict}\PY{p}{(}\PY{n+nb}{sorted}\PY{p}{(}\PY{n}{auc\PYZus{}result}\PY{o}{.}\PY{n}{items}\PY{p}{(}\PY{p}{)}\PY{p}{,} \PY{n}{key}\PY{o}{=}\PY{k}{lambda} \PY{n}{t}\PY{p}{:} \PY{n}{t}\PY{p}{[}\PY{l+m+mi}{1}\PY{p}{]}\PY{p}{,} \PY{n}{reverse}\PY{o}{=}\PY{k+kc}{True}\PY{p}{)}\PY{p}{)}
\PY{n+nb}{print}\PY{p}{(}\PY{n}{tab\PYZus{}AUC}\PY{p}{)}
\PY{n}{plt}\PY{o}{.}\PY{n}{bar}\PY{p}{(}\PY{n}{auc\PYZus{}result}\PY{o}{.}\PY{n}{keys}\PY{p}{(}\PY{p}{)}\PY{p}{,}\PY{n}{tab\PYZus{}AUC}\PY{o}{.}\PY{n}{values}\PY{p}{(}\PY{p}{)}\PY{p}{,}\PY{n}{color}\PY{o}{=}\PY{p}{[}\PY{l+s+s1}{\PYZsq{}}\PY{l+s+s1}{\PYZsh{}0072BD}\PY{l+s+s1}{\PYZsq{}}\PY{p}{,}\PY{l+s+s1}{\PYZsq{}}\PY{l+s+s1}{\PYZsh{}D95319}\PY{l+s+s1}{\PYZsq{}}\PY{p}{,}\PY{l+s+s1}{\PYZsq{}}\PY{l+s+s1}{\PYZsh{}EDB120}\PY{l+s+s1}{\PYZsq{}}\PY{p}{,}\PY{l+s+s1}{\PYZsq{}}\PY{l+s+s1}{\PYZsh{}7E2F8E}\PY{l+s+s1}{\PYZsq{}}\PY{p}{,}\PY{l+s+s1}{\PYZsq{}}\PY{l+s+s1}{\PYZsh{}77AC30}\PY{l+s+s1}{\PYZsq{}}\PY{p}{,}\PY{l+s+s1}{\PYZsq{}}\PY{l+s+s1}{\PYZsh{}4DBEEE}\PY{l+s+s1}{\PYZsq{}}\PY{p}{,}\PY{l+s+s1}{\PYZsq{}}\PY{l+s+s1}{\PYZsh{}A2142F}\PY{l+s+s1}{\PYZsq{}}\PY{p}{]}\PY{p}{)}
\end{Verbatim}
\end{tcolorbox}

    \begin{Verbatim}[commandchars=\\\{\}]
OrderedDict([('ADL', 0.681), ('XGB', 0.61), ('RF', 0.569), ('ADXG', 0.538),
('KNN', 0.536), ('RFSM', 0.522), ('DT', 0.512), ('IF', 0)])
    \end{Verbatim}

            \begin{tcolorbox}[breakable, size=fbox, boxrule=.5pt, pad at break*=1mm, opacityfill=0]
\prompt{Out}{outcolor}{202}{\boxspacing}
\begin{Verbatim}[commandchars=\\\{\}]
<BarContainer object of 8 artists>
\end{Verbatim}
\end{tcolorbox}
        
    \begin{center}
    \adjustimage{max size={0.9\linewidth}{0.9\paperheight}}{output_169_2.png}
    \end{center}
    { \hspace*{\fill} \\}
    
    \hypertarget{temps-de-calcul}{%
\subparagraph{Temps de calcul :}\label{temps-de-calcul}}

    \begin{tcolorbox}[breakable, size=fbox, boxrule=1pt, pad at break*=1mm,colback=cellbackground, colframe=cellborder]
\prompt{In}{incolor}{209}{\boxspacing}
\begin{Verbatim}[commandchars=\\\{\}]
\PY{c+c1}{\PYZsh{}barre plot test}
\PY{n}{tab\PYZus{}temps}\PY{o}{=}\PY{n}{OrderedDict}\PY{p}{(}\PY{n+nb}{sorted}\PY{p}{(}\PY{n}{time\PYZus{}result}\PY{o}{.}\PY{n}{items}\PY{p}{(}\PY{p}{)}\PY{p}{,} \PY{n}{key}\PY{o}{=}\PY{k}{lambda} \PY{n}{t}\PY{p}{:} \PY{n}{t}\PY{p}{[}\PY{l+m+mi}{1}\PY{p}{]}\PY{p}{,} \PY{n}{reverse}\PY{o}{=}\PY{k+kc}{True}\PY{p}{)}\PY{p}{)}
\PY{n+nb}{print}\PY{p}{(}\PY{n}{tab\PYZus{}temps}\PY{p}{)}
\PY{n}{plt}\PY{o}{.}\PY{n}{bar}\PY{p}{(}\PY{n}{tab\PYZus{}temps}\PY{o}{.}\PY{n}{keys}\PY{p}{(}\PY{p}{)}\PY{p}{,}\PY{n}{tab\PYZus{}temps}\PY{o}{.}\PY{n}{values}\PY{p}{(}\PY{p}{)}\PY{p}{,}\PY{n}{color}\PY{o}{=}\PY{p}{[}\PY{l+s+s1}{\PYZsq{}}\PY{l+s+s1}{\PYZsh{}0072BD}\PY{l+s+s1}{\PYZsq{}}\PY{p}{,}\PY{l+s+s1}{\PYZsq{}}\PY{l+s+s1}{\PYZsh{}D95319}\PY{l+s+s1}{\PYZsq{}}\PY{p}{,}\PY{l+s+s1}{\PYZsq{}}\PY{l+s+s1}{\PYZsh{}EDB120}\PY{l+s+s1}{\PYZsq{}}\PY{p}{,}\PY{l+s+s1}{\PYZsq{}}\PY{l+s+s1}{\PYZsh{}7E2F8E}\PY{l+s+s1}{\PYZsq{}}\PY{p}{,}\PY{l+s+s1}{\PYZsq{}}\PY{l+s+s1}{\PYZsh{}77AC30}\PY{l+s+s1}{\PYZsq{}}\PY{p}{,}\PY{l+s+s1}{\PYZsq{}}\PY{l+s+s1}{\PYZsh{}4DBEEE}\PY{l+s+s1}{\PYZsq{}}\PY{p}{,}\PY{l+s+s1}{\PYZsq{}}\PY{l+s+s1}{\PYZsh{}A2142F}\PY{l+s+s1}{\PYZsq{}}\PY{p}{]}\PY{p}{)}
\end{Verbatim}
\end{tcolorbox}

    \begin{Verbatim}[commandchars=\\\{\}]
OrderedDict([('KNN', 85.574), ('DT', 63.411), ('RFSM', 48.063), ('RF', 9.095),
('ADL', 5.1), ('ADXG', 1.934), ('XGB', 0.756), ('IF', 0.486)])
    \end{Verbatim}

            \begin{tcolorbox}[breakable, size=fbox, boxrule=.5pt, pad at break*=1mm, opacityfill=0]
\prompt{Out}{outcolor}{209}{\boxspacing}
\begin{Verbatim}[commandchars=\\\{\}]
<BarContainer object of 8 artists>
\end{Verbatim}
\end{tcolorbox}
        
    \begin{center}
    \adjustimage{max size={0.9\linewidth}{0.9\paperheight}}{output_171_2.png}
    \end{center}
    { \hspace*{\fill} \\}
    
    \hypertarget{conclusion-sur-f1}{%
\subsection{9. Conclusion sur F1}\label{conclusion-sur-f1}}

    En transformant le fromule de F1, nous avons:

F1 := 2 / (1/precision + 1/rapel).

Pourquoi le F1 est toujours petit ? Est ce que ca veut dire que notre
modèle est mauvais ?

Pour répondre à ces questions, nous examinons le meilleur score que nous
pouvons atteindre sans aucune connaissance, par exemple en lançant une
pièce. Cette pièce peut être injuste. Soit p la probabilité que la pièce
prédit un résultat positif, c'est-à-dire qu'une pièce parfaitement juste
aurait p = 0,5. Soit q la part des cas positifs réels. Dans ce scénario,
il n'est pas difficile de déduire des définitions que précision = q et
rappel = p.~Par conséquent, la précision n'est pas influencée par la
configuration de notre pièce. Et le rappel est meilleur si la pièce
prédit toujours positif (p = 1).Étonnamment, prédire toujours positif
est le mieux que nous puissions faire en termes de score F1 si nous
n'avons aucune information. Cela est dû au fait que le score F1 n'est
pas symétrique entre les cas positifs et négatifs. Il accorde plus
d'attention aux cas positifs. Nous avons le F1 max:

F1\_p = 2/((1/q + 1/1)) =2/((1+q)/q) = 2q/(1+q)

Nous allons voir le changement de F1 selon q :

    \begin{tcolorbox}[breakable, size=fbox, boxrule=1pt, pad at break*=1mm,colback=cellbackground, colframe=cellborder]
\prompt{In}{incolor}{214}{\boxspacing}
\begin{Verbatim}[commandchars=\\\{\}]
\PY{k+kn}{from} \PY{n+nn}{astropy}\PY{n+nn}{.}\PY{n+nn}{table} \PY{k+kn}{import} \PY{n}{QTable}\PY{p}{,} \PY{n}{Table}\PY{p}{,} \PY{n}{Column}
\PY{n}{row} \PY{o}{=} \PY{p}{[}\PY{l+s+s2}{\PYZdq{}}\PY{l+s+s2}{Tout negative}\PY{l+s+s2}{\PYZdq{}}\PY{p}{,}\PY{l+s+s2}{\PYZdq{}}\PY{l+s+s2}{Peu positif}\PY{l+s+s2}{\PYZdq{}}\PY{p}{,}\PY{l+s+s2}{\PYZdq{}}\PY{l+s+s2}{Balance}\PY{l+s+s2}{\PYZdq{}}\PY{p}{,}\PY{l+s+s2}{\PYZdq{}}\PY{l+s+s2}{Peu negatif}\PY{l+s+s2}{\PYZdq{}}\PY{p}{,}\PY{l+s+s2}{\PYZdq{}}\PY{l+s+s2}{Tout positif}\PY{l+s+s2}{\PYZdq{}}\PY{p}{]}
\PY{n}{col1} \PY{o}{=} \PY{p}{[}\PY{l+m+mi}{0}\PY{p}{,} \PY{l+m+mf}{0.1}\PY{p}{,} \PY{l+m+mf}{0.5}\PY{p}{,} \PY{l+m+mf}{0.9}\PY{p}{,} \PY{l+m+mi}{1}\PY{p}{]}
\PY{n}{col2} \PY{o}{=} \PY{p}{[}\PY{l+m+mi}{0}\PY{p}{,} \PY{l+m+mf}{0.19}\PY{p}{,} \PY{l+m+mf}{0.66}\PY{p}{,}\PY{l+m+mf}{0.95}\PY{p}{,}\PY{l+m+mi}{1}\PY{p}{]}
\PY{n}{t} \PY{o}{=} \PY{n}{Table}\PY{p}{(}\PY{p}{[}\PY{n}{row}\PY{p}{,}\PY{n}{col1}\PY{p}{,}\PY{n}{col2}\PY{p}{]}\PY{p}{,} \PY{n}{names}\PY{o}{=}\PY{p}{(}\PY{l+s+s1}{\PYZsq{}}\PY{l+s+s1}{Type}\PY{l+s+s1}{\PYZsq{}}\PY{p}{,}\PY{l+s+s1}{\PYZsq{}}\PY{l+s+s1}{Taux positif q}\PY{l+s+s1}{\PYZsq{}}\PY{p}{,} \PY{l+s+s1}{\PYZsq{}}\PY{l+s+s1}{F1}\PY{l+s+s1}{\PYZsq{}}\PY{p}{)}\PY{p}{)}
\PY{n+nb}{print}\PY{p}{(}\PY{n}{t}\PY{p}{)}
\end{Verbatim}
\end{tcolorbox}

    \begin{Verbatim}[commandchars=\\\{\}]
     Type     Taux positif q  F1
------------- -------------- ----
Tout negative            0.0  0.0
  Peu positif            0.1 0.19
      Balance            0.5 0.66
  Peu negatif            0.9 0.95
 Tout positif            1.0  1.0
    \end{Verbatim}

    \begin{tcolorbox}[breakable, size=fbox, boxrule=1pt, pad at break*=1mm,colback=cellbackground, colframe=cellborder]
\prompt{In}{incolor}{231}{\boxspacing}
\begin{Verbatim}[commandchars=\\\{\}]
\PY{n}{d} \PY{o}{=} \PY{p}{\PYZob{}}\PY{l+s+s1}{\PYZsq{}}\PY{l+s+s1}{q}\PY{l+s+s1}{\PYZsq{}}\PY{p}{:} \PY{n}{col1}\PY{p}{,} \PY{l+s+s1}{\PYZsq{}}\PY{l+s+s1}{F1}\PY{l+s+s1}{\PYZsq{}}\PY{p}{:} \PY{n}{col2}\PY{p}{\PYZcb{}}
\PY{n}{d} \PY{o}{=} \PY{n}{pd}\PY{o}{.}\PY{n}{DataFrame}\PY{p}{(}\PY{n}{d}\PY{p}{)}
\PY{n}{d}\PY{o}{.}\PY{n}{plot}\PY{p}{(}\PY{n}{x}\PY{o}{=}\PY{l+s+s1}{\PYZsq{}}\PY{l+s+s1}{q}\PY{l+s+s1}{\PYZsq{}}\PY{p}{,}\PY{n}{y}\PY{o}{=}\PY{l+s+s1}{\PYZsq{}}\PY{l+s+s1}{F1}\PY{l+s+s1}{\PYZsq{}}\PY{p}{)}
\end{Verbatim}
\end{tcolorbox}

            \begin{tcolorbox}[breakable, size=fbox, boxrule=.5pt, pad at break*=1mm, opacityfill=0]
\prompt{Out}{outcolor}{231}{\boxspacing}
\begin{Verbatim}[commandchars=\\\{\}]
<matplotlib.axes.\_subplots.AxesSubplot at 0x7fe23ec9d970>
\end{Verbatim}
\end{tcolorbox}
        
    \begin{center}
    \adjustimage{max size={0.9\linewidth}{0.9\paperheight}}{output_175_1.png}
    \end{center}
    { \hspace*{\fill} \\}
    
    Da graphique, ca montre que le F1 dépend fortement du déséquilibre de
notre jeu de données.

    \hypertarget{bonus}{%
\subsection{10. Bonus}\label{bonus}}

    Nous voudrons maximiser le chiffre d'affaire

    \hypertarget{analyse-univariuxe9}{%
\subsubsection{10.1 Analyse Univarié}\label{analyse-univariuxe9}}

    Une vue globale sur les 2 classes

    \begin{tcolorbox}[breakable, size=fbox, boxrule=1pt, pad at break*=1mm,colback=cellbackground, colframe=cellborder]
\prompt{In}{incolor}{67}{\boxspacing}
\begin{Verbatim}[commandchars=\\\{\}]
\PY{n+nb}{print} \PY{p}{(}\PY{l+s+s2}{\PYZdq{}}\PY{l+s+s2}{Fraud}\PY{l+s+s2}{\PYZdq{}}\PY{p}{)}
\PY{n+nb}{print} \PY{p}{(}\PY{n}{df}\PY{o}{.}\PY{n}{MontAnt}\PY{p}{[}\PY{n}{df}\PY{o}{.}\PY{n}{FlAgImpAye} \PY{o}{==} \PY{l+m+mi}{1}\PY{p}{]}\PY{o}{.}\PY{n}{describe}\PY{p}{(}\PY{p}{)}\PY{p}{)}
\PY{n+nb}{print} \PY{p}{(}\PY{p}{)}
\PY{n+nb}{print} \PY{p}{(}\PY{l+s+s2}{\PYZdq{}}\PY{l+s+s2}{Normal}\PY{l+s+s2}{\PYZdq{}}\PY{p}{)}
\PY{n+nb}{print} \PY{p}{(}\PY{n}{df}\PY{o}{.}\PY{n}{MontAnt}\PY{p}{[}\PY{n}{df}\PY{o}{.}\PY{n}{FlAgImpAye} \PY{o}{==} \PY{l+m+mi}{0}\PY{p}{]}\PY{o}{.}\PY{n}{describe}\PY{p}{(}\PY{p}{)}\PY{p}{)}
\end{Verbatim}
\end{tcolorbox}

    \begin{Verbatim}[commandchars=\\\{\}]
Fraud
count    6257.000000
mean       94.557721
std       133.659270
min         3.170000
25\%        31.160000
50\%        50.630000
75\%       103.240000
max      3825.610000
Name: MontAnt, dtype: float64

Normal
count    2.225112e+06
mean     5.982031e+01
std      7.958524e+01
min      1.000000e-02
25\%      2.500000e+01
50\%      4.233000e+01
75\%      7.200000e+01
max      1.698534e+04
Name: MontAnt, dtype: float64
    \end{Verbatim}

    \begin{tcolorbox}[breakable, size=fbox, boxrule=1pt, pad at break*=1mm,colback=cellbackground, colframe=cellborder]
\prompt{In}{incolor}{65}{\boxspacing}
\begin{Verbatim}[commandchars=\\\{\}]
\PY{n}{f}\PY{p}{,} \PY{p}{(}\PY{n}{ax1}\PY{p}{,} \PY{n}{ax2}\PY{p}{)} \PY{o}{=} \PY{n}{plt}\PY{o}{.}\PY{n}{subplots}\PY{p}{(}\PY{l+m+mi}{2}\PY{p}{,} \PY{l+m+mi}{1}\PY{p}{,} \PY{n}{sharex}\PY{o}{=}\PY{k+kc}{True}\PY{p}{,} \PY{n}{figsize}\PY{o}{=}\PY{p}{(}\PY{l+m+mi}{12}\PY{p}{,}\PY{l+m+mi}{4}\PY{p}{)}\PY{p}{)}

\PY{n}{bins} \PY{o}{=} \PY{l+m+mi}{30}

\PY{n}{ax1}\PY{o}{.}\PY{n}{hist}\PY{p}{(}\PY{n}{df}\PY{o}{.}\PY{n}{MontAnt}\PY{p}{[}\PY{n}{df}\PY{o}{.}\PY{n}{FlAgImpAye} \PY{o}{==} \PY{l+m+mi}{1}\PY{p}{]}\PY{p}{,} \PY{n}{bins} \PY{o}{=} \PY{n}{bins}\PY{p}{)}
\PY{n}{ax1}\PY{o}{.}\PY{n}{set\PYZus{}title}\PY{p}{(}\PY{l+s+s1}{\PYZsq{}}\PY{l+s+s1}{Fraud}\PY{l+s+s1}{\PYZsq{}}\PY{p}{)}

\PY{n}{ax2}\PY{o}{.}\PY{n}{hist}\PY{p}{(}\PY{n}{df}\PY{o}{.}\PY{n}{MontAnt}\PY{p}{[}\PY{n}{df}\PY{o}{.}\PY{n}{FlAgImpAye} \PY{o}{==} \PY{l+m+mi}{0}\PY{p}{]}\PY{p}{,} \PY{n}{bins} \PY{o}{=} \PY{n}{bins}\PY{p}{)}
\PY{n}{ax2}\PY{o}{.}\PY{n}{set\PYZus{}title}\PY{p}{(}\PY{l+s+s1}{\PYZsq{}}\PY{l+s+s1}{Normal}\PY{l+s+s1}{\PYZsq{}}\PY{p}{)}

\PY{n}{plt}\PY{o}{.}\PY{n}{xlabel}\PY{p}{(}\PY{l+s+s1}{\PYZsq{}}\PY{l+s+s1}{Amount (\PYZdl{})}\PY{l+s+s1}{\PYZsq{}}\PY{p}{)}
\PY{n}{plt}\PY{o}{.}\PY{n}{ylabel}\PY{p}{(}\PY{l+s+s1}{\PYZsq{}}\PY{l+s+s1}{Number of Transactions}\PY{l+s+s1}{\PYZsq{}}\PY{p}{)}
\PY{n}{plt}\PY{o}{.}\PY{n}{yscale}\PY{p}{(}\PY{l+s+s1}{\PYZsq{}}\PY{l+s+s1}{log}\PY{l+s+s1}{\PYZsq{}}\PY{p}{)}
\PY{n}{plt}\PY{o}{.}\PY{n}{show}\PY{p}{(}\PY{p}{)}
\end{Verbatim}
\end{tcolorbox}

    \begin{center}
    \adjustimage{max size={0.9\linewidth}{0.9\paperheight}}{output_182_0.png}
    \end{center}
    { \hspace*{\fill} \\}
    
    La moyenne des transactions frauduleax sont plus élevés que celle du
normal, cependant, la distribution du normal est beaucoup plus large. Le
montant maximum de la classe normale est clairement supérier que celui
du fraud.

    On va observer la distribution des 2 classes selon Heure :

    \begin{tcolorbox}[breakable, size=fbox, boxrule=1pt, pad at break*=1mm,colback=cellbackground, colframe=cellborder]
\prompt{In}{incolor}{62}{\boxspacing}
\begin{Verbatim}[commandchars=\\\{\}]
\PY{n}{f}\PY{p}{,} \PY{p}{(}\PY{n}{ax1}\PY{p}{,} \PY{n}{ax2}\PY{p}{)} \PY{o}{=} \PY{n}{plt}\PY{o}{.}\PY{n}{subplots}\PY{p}{(}\PY{l+m+mi}{2}\PY{p}{,} \PY{l+m+mi}{1}\PY{p}{,} \PY{n}{sharex}\PY{o}{=}\PY{k+kc}{True}\PY{p}{,} \PY{n}{figsize}\PY{o}{=}\PY{p}{(}\PY{l+m+mi}{12}\PY{p}{,}\PY{l+m+mi}{4}\PY{p}{)}\PY{p}{)}

\PY{n}{bins} \PY{o}{=} \PY{l+m+mi}{50}

\PY{n}{ax1}\PY{o}{.}\PY{n}{hist}\PY{p}{(}\PY{n}{df}\PY{o}{.}\PY{n}{Heure}\PY{p}{[}\PY{n}{df}\PY{o}{.}\PY{n}{FlAgImpAye}  \PY{o}{==} \PY{l+m+mi}{1}\PY{p}{]}\PY{p}{,} \PY{n}{bins} \PY{o}{=} \PY{n}{bins}\PY{p}{)}
\PY{n}{ax1}\PY{o}{.}\PY{n}{set\PYZus{}title}\PY{p}{(}\PY{l+s+s1}{\PYZsq{}}\PY{l+s+s1}{Fraud}\PY{l+s+s1}{\PYZsq{}}\PY{p}{)}

\PY{n}{ax2}\PY{o}{.}\PY{n}{hist}\PY{p}{(}\PY{n}{df}\PY{o}{.}\PY{n}{Heure}\PY{p}{[}\PY{n}{df}\PY{o}{.}\PY{n}{FlAgImpAye}  \PY{o}{==} \PY{l+m+mi}{0}\PY{p}{]}\PY{p}{,} \PY{n}{bins} \PY{o}{=} \PY{n}{bins}\PY{p}{)}
\PY{n}{ax2}\PY{o}{.}\PY{n}{set\PYZus{}title}\PY{p}{(}\PY{l+s+s1}{\PYZsq{}}\PY{l+s+s1}{Normal}\PY{l+s+s1}{\PYZsq{}}\PY{p}{)}

\PY{n}{plt}\PY{o}{.}\PY{n}{xlabel}\PY{p}{(}\PY{l+s+s1}{\PYZsq{}}\PY{l+s+s1}{Heure}\PY{l+s+s1}{\PYZsq{}}\PY{p}{)}
\PY{n}{plt}\PY{o}{.}\PY{n}{ylabel}\PY{p}{(}\PY{l+s+s1}{\PYZsq{}}\PY{l+s+s1}{Nombre de Transaction}\PY{l+s+s1}{\PYZsq{}}\PY{p}{)}
\PY{n}{plt}\PY{o}{.}\PY{n}{show}\PY{p}{(}\PY{p}{)}
\end{Verbatim}
\end{tcolorbox}

    \begin{center}
    \adjustimage{max size={0.9\linewidth}{0.9\paperheight}}{output_185_0.png}
    \end{center}
    { \hspace*{\fill} \\}
    
    La distribution des deux classe selon Heure sont similaire. On pourrais
assuler que les transations frauduleux sont plus gaussiens que les
transactions normaux.

    Les chiffre d'affaire des 2 classes selon Mois :

    \begin{tcolorbox}[breakable, size=fbox, boxrule=1pt, pad at break*=1mm,colback=cellbackground, colframe=cellborder]
\prompt{In}{incolor}{ }{\boxspacing}
\begin{Verbatim}[commandchars=\\\{\}]
\PY{n}{f}\PY{p}{,} \PY{p}{(}\PY{n}{ax1}\PY{p}{,} \PY{n}{ax2}\PY{p}{)} \PY{o}{=} \PY{n}{plt}\PY{o}{.}\PY{n}{subplots}\PY{p}{(}\PY{l+m+mi}{2}\PY{p}{,} \PY{l+m+mi}{1}\PY{p}{,} \PY{n}{sharex}\PY{o}{=}\PY{k+kc}{True}\PY{p}{,} \PY{n}{figsize}\PY{o}{=}\PY{p}{(}\PY{l+m+mi}{12}\PY{p}{,}\PY{l+m+mi}{4}\PY{p}{)}\PY{p}{)}

\PY{n}{bins} \PY{o}{=} \PY{l+m+mi}{50}

\PY{n}{ax1}\PY{o}{.}\PY{n}{hist}\PY{p}{(}\PY{n}{df}\PY{o}{.}\PY{n}{Mois}\PY{p}{[}\PY{n}{df}\PY{o}{.}\PY{n}{FlAgImpAye}  \PY{o}{==} \PY{l+m+mi}{1}\PY{p}{]}\PY{p}{,} \PY{n}{bins} \PY{o}{=} \PY{n}{bins}\PY{p}{)}
\PY{n}{ax1}\PY{o}{.}\PY{n}{set\PYZus{}title}\PY{p}{(}\PY{l+s+s1}{\PYZsq{}}\PY{l+s+s1}{Fraud}\PY{l+s+s1}{\PYZsq{}}\PY{p}{)}

\PY{n}{ax2}\PY{o}{.}\PY{n}{hist}\PY{p}{(}\PY{n}{df}\PY{o}{.}\PY{n}{Mois}\PY{p}{[}\PY{n}{df}\PY{o}{.}\PY{n}{FlAgImpAye}  \PY{o}{==} \PY{l+m+mi}{0}\PY{p}{]}\PY{p}{,} \PY{n}{bins} \PY{o}{=} \PY{n}{bins}\PY{p}{)}
\PY{n}{ax2}\PY{o}{.}\PY{n}{set\PYZus{}title}\PY{p}{(}\PY{l+s+s1}{\PYZsq{}}\PY{l+s+s1}{Normal}\PY{l+s+s1}{\PYZsq{}}\PY{p}{)}

\PY{n}{plt}\PY{o}{.}\PY{n}{xlabel}\PY{p}{(}\PY{l+s+s1}{\PYZsq{}}\PY{l+s+s1}{Mois}\PY{l+s+s1}{\PYZsq{}}\PY{p}{)}
\PY{n}{plt}\PY{o}{.}\PY{n}{ylabel}\PY{p}{(}\PY{l+s+s1}{\PYZsq{}}\PY{l+s+s1}{Nombre de Transaction}\PY{l+s+s1}{\PYZsq{}}\PY{p}{)}
\PY{n}{plt}\PY{o}{.}\PY{n}{show}\PY{p}{(}\PY{p}{)}
\end{Verbatim}
\end{tcolorbox}

    \hypertarget{chiffre-daffaire-origin}{%
\subsubsection{10.2. Chiffre d'affaire
origin}\label{chiffre-daffaire-origin}}

    Le CA origin est caculé basé sur tous les transactions normaux

    \begin{tcolorbox}[breakable, size=fbox, boxrule=1pt, pad at break*=1mm,colback=cellbackground, colframe=cellborder]
\prompt{In}{incolor}{233}{\boxspacing}
\begin{Verbatim}[commandchars=\\\{\}]
\PY{n}{b}\PY{o}{=}\PY{n}{df}\PY{o}{.}\PY{n}{loc}\PY{p}{[}\PY{n}{df}\PY{p}{[}\PY{l+s+s1}{\PYZsq{}}\PY{l+s+s1}{DAteTrAnsAction}\PY{l+s+s1}{\PYZsq{}}\PY{p}{]} \PY{o}{\PYZgt{}}\PY{o}{=} \PY{l+s+s1}{\PYZsq{}}\PY{l+s+s1}{2016\PYZhy{}09\PYZhy{}20}\PY{l+s+s1}{\PYZsq{}}\PY{p}{,}\PY{l+s+s1}{\PYZsq{}}\PY{l+s+s1}{MontAnt}\PY{l+s+s1}{\PYZsq{}}\PY{p}{]}\PY{o}{.}\PY{n}{to\PYZus{}numpy}\PY{p}{(}\PY{p}{)}
\PY{n}{CAorigin} \PY{o}{=} \PY{n}{np}\PY{o}{.}\PY{n}{sum}\PY{p}{(}\PY{n}{b}\PY{p}{[}\PY{n}{test\PYZus{}target}\PY{o}{==}\PY{l+m+mi}{0}\PY{p}{]}\PY{p}{)}
\PY{n+nb}{print}\PY{p}{(}\PY{n}{CAorigin}\PY{p}{)}
\end{Verbatim}
\end{tcolorbox}

    \begin{Verbatim}[commandchars=\\\{\}]
15436477.790000001
    \end{Verbatim}

    \hypertarget{les-moduxe8les-pour-maximiser-le-ca}{%
\subsubsection{10.3. Les Modèles pour maximiser le
CA}\label{les-moduxe8les-pour-maximiser-le-ca}}

    \begin{itemize}
\tightlist
\item
  si on accepte une bonne transaction (TN) : le chiffre d'affaire généré
  est égal au montant de la transaction f(m) = m
\item
  si on accepte une mauvaise transaction (FN) : le chiffre d'affaire
  perdu est proportionnel au montant à f(m) = m(1 − exp(-m)) Plus le
  montant de la transaction est élevé, plus la perte est importante.
\item
  lorsque vous refusez une bonne transaction (FP) : vous générez un
  chiffre d'affaire égal à 80\% du montant de la transaction, f(m) =
  0.8m
\item
  lorsque vous refusez une transaction frauduleuse, le chiffre d'affaire
  est nul f(m) = 0
\end{itemize}

    On a le formule de chiffre d'affaire :

    \begin{verbatim}
                            CA_Modèle = TN -FN*(1-exp(-FN))+0.8*FP+0*TP
\end{verbatim}

    Afin d'obtenir le resultat selon les méthodes proposé, nous avons
développé 2 formules qui sont pris l'advantage de la class 0 et 1.

    a=df.loc{[}df{[}`DAteTrAnsAction'{]} \textgreater=
`2016-09-20',`MontAnt'{]}.to\_numpy()

a égale les valeurs de Montant de Test

a=(1-test\_target.to\_numpy())\emph{0.2}a +
test\_target.to\_numpy()\emph{a}(1-np.exp(-a)) print(a)

On a test\_target = 1 ou 0 :

\begin{itemize}
\item
  Cas 1 : test\_target = 1 =\textgreater{} Fraud

  a = (1 - 1)\emph{0.2}a + 1\emph{a}(1-exp(-a)) = 0 + a(1-exp(-a)) =
  a(1-exp(-a))
\item
  Cas 2 : test\_target = 0 =\textgreater{} Normal

  a = (1 - 0)\emph{0.2}a + 0\emph{a}(1-exp(-a)) = 0.2*a
\end{itemize}

b=df.loc{[}df{[}`DAteTrAnsAction'{]} \textgreater=
`2016-09-20',`MontAnt'{]}.to\_numpy()

b égale les valeurs de Montant de Test

CA=np.sum(b{[}test\_target==0{]}) -
np.sum(((RDbest4.predict(test\_features)-test\_target)**2)*a)

np.sum(b{[}test\_target==0{]}) = Chiffre d'affaire origin

Donc maintenant, on va étudier les cas de la fonction suivant qui
suppose d'être les pertes.

RDbest4.predict(test\_features)-test\_target

\begin{itemize}
\item
  Cas 1 : test\_target = 1 =\textgreater{} Fraud a = a(1-exp(-a))

  \begin{itemize}
  \item
    Cas 1.1 : test\_feature = predict= 1 =\textgreater{} Vrai Positif

    (((RDbest4.predict(test\_features)-test\_target)**2)\emph{a)
    ((1-1)\^{}2)}a =0

    Donc on enleve rien
  \item
    Cas 1.2 : test\_feature = predict= 0 =\textgreater{} Faux Negatif
    ((RDbest4.predict(test\_features)-test\_target)**2)\emph{a)
    ((0-1)\^{}2)}a =1*a = a = a(1-exp(-a)) Donc on enleve: a(1-exp(-a))
  \end{itemize}
\item
  Cas 2 : test\_target = Flag = 0 =\textgreater{} Normal a = 0.2*a

  \begin{itemize}
  \item
    Cas 2.1 : test\_feature = predict= 1 =\textgreater{} Faux Positif
    (((RDbest4.predict(test\_features)-test\_target)**2)\emph{a)
    ((1-0)\^{}2)}a =1\emph{a = a =0.2}a Donc on enleve 0.2*a
  \item
    Cas 2.2 : test\_feature = predict= 0 =\textgreater{} VN
    ((RDbest4.predict(test\_features)-test\_target)**2)\emph{a)
    ((0-0)\^{}2)}a =0 Donc on enleve rien
  \end{itemize}
\end{itemize}

    \hypertarget{random-forest---foruxeat-aluxe9atoire}{%
\paragraph{10.3.1. Random Forest - Forêt
aléatoire}\label{random-forest---foruxeat-aluxe9atoire}}

    Comme le premier partie, avec le meilleur modèle pour F1, nous allons
tester pour savoir la difference de la prédiction.

    \begin{tcolorbox}[breakable, size=fbox, boxrule=1pt, pad at break*=1mm,colback=cellbackground, colframe=cellborder]
\prompt{In}{incolor}{71}{\boxspacing}
\begin{Verbatim}[commandchars=\\\{\}]
\PY{n}{startRF} \PY{o}{=} \PY{n}{time}\PY{p}{(}\PY{p}{)}
\PY{n}{RDbestCA} \PY{o}{=} \PY{n}{make\PYZus{}pipeline}\PY{p}{(}\PY{n}{StandardScaler}\PY{p}{(}\PY{p}{)}\PY{p}{,}\PY{n}{RandomForestClassifier}\PY{p}{(}\PY{n}{bootstrap}\PY{o}{=}\PY{k+kc}{True}\PY{p}{,} \PY{n}{class\PYZus{}weight}\PY{o}{=}\PY{p}{\PYZob{}}\PY{l+m+mi}{0}\PY{p}{:}\PY{l+m+mi}{1}\PY{p}{,} \PY{l+m+mi}{1}\PY{p}{:}\PY{l+m+mi}{375}\PY{p}{\PYZcb{}}\PY{p}{,} \PY{n}{criterion}\PY{o}{=}\PY{l+s+s1}{\PYZsq{}}\PY{l+s+s1}{entropy}\PY{l+s+s1}{\PYZsq{}}\PY{p}{,} \PY{n}{max\PYZus{}depth}\PY{o}{=}\PY{l+m+mi}{200}\PY{p}{,}\PY{n}{min\PYZus{}samples\PYZus{}leaf}\PY{o}{=}\PY{l+m+mi}{20}\PY{p}{,}
\PY{n}{n\PYZus{}estimators}\PY{o}{=}\PY{l+m+mi}{500}\PY{p}{,} \PY{n}{n\PYZus{}jobs}\PY{o}{=}\PY{o}{\PYZhy{}}\PY{l+m+mi}{1}\PY{p}{,} \PY{n}{random\PYZus{}state}\PY{o}{=}\PY{l+m+mi}{5}\PY{p}{)}\PY{p}{)}
\PY{n}{RDbestCA}\PY{o}{.}\PY{n}{fit}\PY{p}{(}\PY{n}{x\PYZus{}train}\PY{p}{,} \PY{n}{y\PYZus{}train}\PY{p}{)} 
\PY{n}{doneRF} \PY{o}{=} \PY{n}{time}\PY{p}{(}\PY{p}{)}

\PY{n}{tpsRF} \PY{o}{=} \PY{n+nb}{round}\PY{p}{(}\PY{n}{doneRF} \PY{o}{\PYZhy{}} \PY{n}{startRF}\PY{p}{,}\PY{l+m+mi}{3}\PY{p}{)}
\PY{n+nb}{print}\PY{p}{(}\PY{l+s+s2}{\PYZdq{}}\PY{l+s+s2}{temps de calcul}\PY{l+s+s2}{\PYZdq{}} \PY{p}{,} \PY{n}{tpsRF}\PY{o}{/}\PY{l+m+mi}{60}\PY{p}{)}

\PY{n+nb}{print}\PY{p}{(}\PY{n}{metrics}\PY{o}{.}\PY{n}{classification\PYZus{}report}\PY{p}{(}\PY{n}{test\PYZus{}target}\PY{p}{,} \PY{n}{RDbestCA}\PY{o}{.}\PY{n}{predict}\PY{p}{(}\PY{n}{test\PYZus{}features}\PY{p}{)}\PY{p}{)}\PY{p}{)}
\PY{n+nb}{print}\PY{p}{(}\PY{l+s+s2}{\PYZdq{}}\PY{l+s+s2}{Score f1:}\PY{l+s+s2}{\PYZdq{}}\PY{p}{,}\PY{n}{f1\PYZus{}score}\PY{p}{(}\PY{n}{test\PYZus{}target}\PY{p}{,}\PY{n}{RDbestCA}\PY{o}{.}\PY{n}{predict}\PY{p}{(}\PY{n}{test\PYZus{}features}\PY{p}{)} \PY{p}{)}\PY{p}{)}
\PY{n+nb}{print}\PY{p}{(}\PY{n}{metrics}\PY{o}{.}\PY{n}{classification\PYZus{}report}\PY{p}{(}\PY{n}{y\PYZus{}val}\PY{p}{,} \PY{n}{RDbestCA}\PY{o}{.}\PY{n}{predict}\PY{p}{(}\PY{n}{x\PYZus{}val}\PY{p}{)}\PY{p}{)}\PY{p}{)}
\PY{n+nb}{print} \PY{p}{(}\PY{l+s+s2}{\PYZdq{}}\PY{l+s+s2}{Score f1:}\PY{l+s+s2}{\PYZdq{}}\PY{p}{,}\PY{n}{f1\PYZus{}score}\PY{p}{(}\PY{n}{y\PYZus{}val}\PY{p}{,} \PY{n}{RDbestCA}\PY{o}{.}\PY{n}{predict}\PY{p}{(}\PY{n}{x\PYZus{}val}\PY{p}{)}\PY{p}{)}\PY{p}{)}
\end{Verbatim}
\end{tcolorbox}

    \begin{Verbatim}[commandchars=\\\{\}]
temps de calcul 11.671016666666667
              precision    recall  f1-score   support

           0       1.00      1.00      1.00    263288
           1       0.16      0.14      0.15       855

    accuracy                           0.99    264143
   macro avg       0.58      0.57      0.57    264143
weighted avg       0.99      0.99      0.99    264143

Score f1: 0.14606741573033707
              precision    recall  f1-score   support

           0       1.00      1.00      1.00    288737
           1       0.16      0.16      0.16       948

    accuracy                           0.99    289685
   macro avg       0.58      0.58      0.58    289685
weighted avg       0.99      0.99      0.99    289685

Score f1: 0.15849843587069862
    \end{Verbatim}

    \begin{tcolorbox}[breakable, size=fbox, boxrule=1pt, pad at break*=1mm,colback=cellbackground, colframe=cellborder]
\prompt{In}{incolor}{73}{\boxspacing}
\begin{Verbatim}[commandchars=\\\{\}]
\PY{n}{a}\PY{o}{=}\PY{n}{df}\PY{o}{.}\PY{n}{loc}\PY{p}{[}\PY{n}{df}\PY{p}{[}\PY{l+s+s1}{\PYZsq{}}\PY{l+s+s1}{DAteTrAnsAction}\PY{l+s+s1}{\PYZsq{}}\PY{p}{]} \PY{o}{\PYZgt{}}\PY{o}{=} \PY{l+s+s1}{\PYZsq{}}\PY{l+s+s1}{2016\PYZhy{}09\PYZhy{}20}\PY{l+s+s1}{\PYZsq{}}\PY{p}{,}\PY{l+s+s1}{\PYZsq{}}\PY{l+s+s1}{MontAnt}\PY{l+s+s1}{\PYZsq{}}\PY{p}{]}\PY{o}{.}\PY{n}{to\PYZus{}numpy}\PY{p}{(}\PY{p}{)}
\PY{n}{a}\PY{o}{=}\PY{p}{(}\PY{l+m+mi}{1}\PY{o}{\PYZhy{}}\PY{n}{test\PYZus{}target}\PY{o}{.}\PY{n}{to\PYZus{}numpy}\PY{p}{(}\PY{p}{)}\PY{p}{)}\PY{o}{*}\PY{l+m+mf}{0.2}\PY{o}{*}\PY{n}{a} \PY{o}{+} \PY{n}{test\PYZus{}target}\PY{o}{.}\PY{n}{to\PYZus{}numpy}\PY{p}{(}\PY{p}{)}\PY{o}{*}\PY{n}{a}\PY{o}{*}\PY{p}{(}\PY{l+m+mi}{1}\PY{o}{\PYZhy{}}\PY{n}{np}\PY{o}{.}\PY{n}{exp}\PY{p}{(}\PY{o}{\PYZhy{}}\PY{n}{a}\PY{p}{)}\PY{p}{)}
\PY{n}{b}\PY{o}{=}\PY{n}{df}\PY{o}{.}\PY{n}{loc}\PY{p}{[}\PY{n}{df}\PY{p}{[}\PY{l+s+s1}{\PYZsq{}}\PY{l+s+s1}{DAteTrAnsAction}\PY{l+s+s1}{\PYZsq{}}\PY{p}{]} \PY{o}{\PYZgt{}}\PY{o}{=} \PY{l+s+s1}{\PYZsq{}}\PY{l+s+s1}{2016\PYZhy{}09\PYZhy{}20}\PY{l+s+s1}{\PYZsq{}}\PY{p}{,}\PY{l+s+s1}{\PYZsq{}}\PY{l+s+s1}{MontAnt}\PY{l+s+s1}{\PYZsq{}}\PY{p}{]}\PY{o}{.}\PY{n}{to\PYZus{}numpy}\PY{p}{(}\PY{p}{)}
\PY{n}{CA\PYZus{}RD}\PY{o}{=}\PY{n}{np}\PY{o}{.}\PY{n}{sum}\PY{p}{(}\PY{n}{b}\PY{p}{[}\PY{n}{test\PYZus{}target}\PY{o}{==}\PY{l+m+mi}{0}\PY{p}{]}\PY{p}{)}\PY{o}{\PYZhy{}}\PY{n}{np}\PY{o}{.}\PY{n}{sum}\PY{p}{(}\PY{p}{(}\PY{p}{(}\PY{n}{RDbestCA}\PY{o}{.}\PY{n}{predict}\PY{p}{(}\PY{n}{test\PYZus{}features}\PY{p}{)}\PY{o}{\PYZhy{}}\PY{n}{test\PYZus{}target}\PY{p}{)}\PY{o}{*}\PY{o}{*}\PY{l+m+mi}{2}\PY{p}{)}\PY{o}{*}\PY{n}{a}\PY{p}{)}
\PY{n+nb}{print}\PY{p}{(}\PY{n}{CA\PYZus{}RD}\PY{p}{)}
\end{Verbatim}
\end{tcolorbox}

    \begin{Verbatim}[commandchars=\\\{\}]
[ 45.41  62.06  16.43 {\ldots} 186.51  38.54  13.93]
15353024.300848942
    \end{Verbatim}

    \begin{tcolorbox}[breakable, size=fbox, boxrule=1pt, pad at break*=1mm,colback=cellbackground, colframe=cellborder]
\prompt{In}{incolor}{76}{\boxspacing}
\begin{Verbatim}[commandchars=\\\{\}]
\PY{n+nb}{print}\PY{p}{(}\PY{n}{CAorigin}\PY{o}{\PYZhy{}}\PY{n}{CA\PYZus{}RD}\PY{p}{)}
\end{Verbatim}
\end{tcolorbox}

    \begin{Verbatim}[commandchars=\\\{\}]
83453.48915105872
    \end{Verbatim}

    Donc notre resultat est environ 83k\$ moins que par rapport le CA
original. Le but est de minimiser ce différent.

    \hypertarget{random-forest-avec-sample_weight}{%
\paragraph{10.3.2 Random Forest avec
Sample\_weight}\label{random-forest-avec-sample_weight}}

    On va utiliser maitenant l'option sample\_weight: plus le montant perdu
pour une transaction élevé, plus de chance d'être bon classé pour cette
transaction.

    Donc on essaie de fit le meilleur modele avec sample\_weight =
montant\_lost ou montant\_lost est le montant perdu de test

    \begin{tcolorbox}[breakable, size=fbox, boxrule=1pt, pad at break*=1mm,colback=cellbackground, colframe=cellborder]
\prompt{In}{incolor}{77}{\boxspacing}
\begin{Verbatim}[commandchars=\\\{\}]
\PY{n}{entrainement} \PY{o}{=} \PY{n}{Apprenti}\PY{o}{.}\PY{n}{loc}\PY{p}{[}\PY{n}{df}\PY{p}{[}\PY{l+s+s1}{\PYZsq{}}\PY{l+s+s1}{DAteTrAnsAction}\PY{l+s+s1}{\PYZsq{}}\PY{p}{]} \PY{o}{\PYZlt{}} \PY{l+s+s1}{\PYZsq{}}\PY{l+s+s1}{2016\PYZhy{}08\PYZhy{}20}\PY{l+s+s1}{\PYZsq{}}\PY{p}{]} 
\PY{n}{data}\PY{o}{=} \PY{n}{entrainement}\PY{o}{.}\PY{n}{drop}\PY{p}{(}\PY{n}{Apprenti}\PY{o}{.}\PY{n}{columns}\PY{p}{[}\PY{p}{[}\PY{l+m+mi}{0}\PY{p}{,} \PY{l+m+mi}{1}\PY{p}{,} \PY{l+m+mi}{3}\PY{p}{,}\PY{l+m+mi}{21}\PY{p}{]}\PY{p}{]}\PY{p}{,} \PY{n}{axis}\PY{o}{=}\PY{l+m+mi}{1}\PY{p}{)} 
\PY{n}{sc}\PY{o}{=}\PY{n}{StandardScaler}\PY{p}{(}\PY{p}{)}
\PY{n}{sc}\PY{o}{.}\PY{n}{fit}\PY{p}{(}\PY{n}{data}\PY{p}{)}
\PY{n}{NormalizeY} \PY{o}{=} \PY{n}{data}\PY{p}{[}\PY{l+s+s1}{\PYZsq{}}\PY{l+s+s1}{FlAgImpAye}\PY{l+s+s1}{\PYZsq{}}\PY{p}{]}
\PY{n}{NormalizeX} \PY{o}{=} \PY{n}{data}\PY{o}{.}\PY{n}{drop}\PY{p}{(}\PY{n}{data}\PY{o}{.}\PY{n}{columns}\PY{p}{[}\PY{p}{[}\PY{l+m+mi}{18}\PY{p}{]}\PY{p}{]}\PY{p}{,} \PY{n}{axis}\PY{o}{=}\PY{l+m+mi}{1}\PY{p}{)} 
\end{Verbatim}
\end{tcolorbox}

    \begin{tcolorbox}[breakable, size=fbox, boxrule=1pt, pad at break*=1mm,colback=cellbackground, colframe=cellborder]
\prompt{In}{incolor}{80}{\boxspacing}
\begin{Verbatim}[commandchars=\\\{\}]
\PY{n}{startRF} \PY{o}{=} \PY{n}{time}\PY{p}{(}\PY{p}{)}

\PY{n}{RDSW} \PY{o}{=} \PY{n}{RandomForestClassifier}\PY{p}{(}\PY{n}{bootstrap}\PY{o}{=}\PY{k+kc}{True}\PY{p}{,} \PY{n}{class\PYZus{}weight}\PY{o}{=}\PY{p}{\PYZob{}}\PY{l+m+mi}{0}\PY{p}{:}\PY{l+m+mi}{1}\PY{p}{,} \PY{l+m+mi}{1}\PY{p}{:}\PY{l+m+mi}{375}\PY{p}{\PYZcb{}}\PY{p}{,} \PY{n}{criterion}\PY{o}{=}\PY{l+s+s1}{\PYZsq{}}\PY{l+s+s1}{entropy}\PY{l+s+s1}{\PYZsq{}}\PY{p}{,} \PY{n}{max\PYZus{}depth}\PY{o}{=}\PY{l+m+mi}{200}\PY{p}{,}\PY{n}{min\PYZus{}samples\PYZus{}leaf}\PY{o}{=}\PY{l+m+mi}{20}\PY{p}{,}
\PY{n}{n\PYZus{}estimators}\PY{o}{=}\PY{l+m+mi}{500}\PY{p}{,} \PY{n}{n\PYZus{}jobs}\PY{o}{=}\PY{o}{\PYZhy{}}\PY{l+m+mi}{1}\PY{p}{,} \PY{n}{random\PYZus{}state}\PY{o}{=}\PY{l+m+mi}{5}\PY{p}{)}


\PY{n}{montant}\PY{o}{=}\PY{n}{x\PYZus{}train}\PY{p}{[}\PY{l+s+s1}{\PYZsq{}}\PY{l+s+s1}{MontAnt}\PY{l+s+s1}{\PYZsq{}}\PY{p}{]}\PY{o}{.}\PY{n}{to\PYZus{}numpy}\PY{p}{(}\PY{p}{)}
\PY{n}{montant\PYZus{}loss}\PY{o}{=}\PY{p}{(}\PY{l+m+mi}{1}\PY{o}{\PYZhy{}}\PY{n}{y\PYZus{}train}\PY{o}{.}\PY{n}{to\PYZus{}numpy}\PY{p}{(}\PY{p}{)}\PY{p}{)}\PY{o}{*}\PY{l+m+mf}{0.2}\PY{o}{*}\PY{n}{montant}\PY{o}{+}\PY{n}{y\PYZus{}train}\PY{o}{.}\PY{n}{to\PYZus{}numpy}\PY{p}{(}\PY{p}{)}\PY{o}{*}\PY{n}{montant}\PY{o}{*}\PY{p}{(}\PY{l+m+mi}{1}\PY{o}{\PYZhy{}}\PY{n}{np}\PY{o}{.}\PY{n}{exp}\PY{p}{(}\PY{o}{\PYZhy{}}\PY{n}{montant}\PY{p}{)}\PY{p}{)}

\PY{n}{RDSW}\PY{o}{.}\PY{n}{fit}\PY{p}{(}\PY{n}{NormalizeX}\PY{p}{,} \PY{n}{NormalizeY}\PY{p}{,} \PY{n}{sample\PYZus{}weight}\PY{o}{=} \PY{n}{montant\PYZus{}loss}\PY{p}{)} 
\PY{n}{doneRF} \PY{o}{=} \PY{n}{time}\PY{p}{(}\PY{p}{)}

\PY{n}{tpsRF} \PY{o}{=} \PY{n+nb}{round}\PY{p}{(}\PY{n}{doneRF} \PY{o}{\PYZhy{}} \PY{n}{startRF}\PY{p}{,}\PY{l+m+mi}{3}\PY{p}{)}
\PY{n+nb}{print}\PY{p}{(}\PY{l+s+s2}{\PYZdq{}}\PY{l+s+s2}{temps de calcul}\PY{l+s+s2}{\PYZdq{}} \PY{p}{,} \PY{n}{tpsRF}\PY{o}{/}\PY{l+m+mi}{60}\PY{p}{)}
\end{Verbatim}
\end{tcolorbox}

    \begin{Verbatim}[commandchars=\\\{\}]
temps de calcul 11.549566666666667
    \end{Verbatim}

    \begin{tcolorbox}[breakable, size=fbox, boxrule=1pt, pad at break*=1mm,colback=cellbackground, colframe=cellborder]
\prompt{In}{incolor}{81}{\boxspacing}
\begin{Verbatim}[commandchars=\\\{\}]
\PY{n+nb}{print}\PY{p}{(}\PY{n}{metrics}\PY{o}{.}\PY{n}{classification\PYZus{}report}\PY{p}{(}\PY{n}{test\PYZus{}target}\PY{p}{,} \PY{n}{RDSW}\PY{o}{.}\PY{n}{predict}\PY{p}{(}\PY{n}{test\PYZus{}features}\PY{p}{)}\PY{p}{)}\PY{p}{)}
\PY{n+nb}{print}\PY{p}{(}\PY{l+s+s2}{\PYZdq{}}\PY{l+s+s2}{Score f1:}\PY{l+s+s2}{\PYZdq{}}\PY{p}{,}\PY{n}{f1\PYZus{}score}\PY{p}{(}\PY{n}{test\PYZus{}target}\PY{p}{,}\PY{n}{RDSW}\PY{o}{.}\PY{n}{predict}\PY{p}{(}\PY{n}{test\PYZus{}features}\PY{p}{)} \PY{p}{)}\PY{p}{)}
\PY{n+nb}{print}\PY{p}{(}\PY{n}{metrics}\PY{o}{.}\PY{n}{classification\PYZus{}report}\PY{p}{(}\PY{n}{y\PYZus{}val}\PY{p}{,} \PY{n}{RDSW}\PY{o}{.}\PY{n}{predict}\PY{p}{(}\PY{n}{x\PYZus{}val}\PY{p}{)}\PY{p}{)}\PY{p}{)}
\PY{n+nb}{print} \PY{p}{(}\PY{l+s+s2}{\PYZdq{}}\PY{l+s+s2}{Score f1:}\PY{l+s+s2}{\PYZdq{}}\PY{p}{,}\PY{n}{f1\PYZus{}score}\PY{p}{(}\PY{n}{y\PYZus{}val}\PY{p}{,} \PY{n}{RDSW}\PY{o}{.}\PY{n}{predict}\PY{p}{(}\PY{n}{x\PYZus{}val}\PY{p}{)}\PY{p}{)}\PY{p}{)}
\end{Verbatim}
\end{tcolorbox}

    \begin{Verbatim}[commandchars=\\\{\}]
              precision    recall  f1-score   support

           0       1.00      1.00      1.00    263288
           1       0.09      0.14      0.11       855

    accuracy                           0.99    264143
   macro avg       0.54      0.57      0.55    264143
weighted avg       0.99      0.99      0.99    264143

Score f1: 0.11044638748274273
              precision    recall  f1-score   support

           0       1.00      0.99      1.00    288737
           1       0.10      0.17      0.12       948

    accuracy                           0.99    289685
   macro avg       0.55      0.58      0.56    289685
weighted avg       0.99      0.99      0.99    289685

Score f1: 0.12260835611089418
    \end{Verbatim}

    \begin{tcolorbox}[breakable, size=fbox, boxrule=1pt, pad at break*=1mm,colback=cellbackground, colframe=cellborder]
\prompt{In}{incolor}{82}{\boxspacing}
\begin{Verbatim}[commandchars=\\\{\}]
\PY{n}{a}\PY{o}{=}\PY{n}{df}\PY{o}{.}\PY{n}{loc}\PY{p}{[}\PY{n}{df}\PY{p}{[}\PY{l+s+s1}{\PYZsq{}}\PY{l+s+s1}{DAteTrAnsAction}\PY{l+s+s1}{\PYZsq{}}\PY{p}{]} \PY{o}{\PYZgt{}}\PY{o}{=} \PY{l+s+s1}{\PYZsq{}}\PY{l+s+s1}{2016\PYZhy{}09\PYZhy{}20}\PY{l+s+s1}{\PYZsq{}}\PY{p}{,}\PY{l+s+s1}{\PYZsq{}}\PY{l+s+s1}{MontAnt}\PY{l+s+s1}{\PYZsq{}}\PY{p}{]}\PY{o}{.}\PY{n}{to\PYZus{}numpy}\PY{p}{(}\PY{p}{)}
\PY{n}{a}\PY{o}{=}\PY{p}{(}\PY{l+m+mi}{1}\PY{o}{\PYZhy{}}\PY{n}{test\PYZus{}target}\PY{o}{.}\PY{n}{to\PYZus{}numpy}\PY{p}{(}\PY{p}{)}\PY{p}{)}\PY{o}{*}\PY{l+m+mf}{0.2}\PY{o}{*}\PY{n}{a} \PY{o}{+} \PY{n}{test\PYZus{}target}\PY{o}{.}\PY{n}{to\PYZus{}numpy}\PY{p}{(}\PY{p}{)}\PY{o}{*}\PY{n}{a}\PY{o}{*}\PY{p}{(}\PY{l+m+mi}{1}\PY{o}{\PYZhy{}}\PY{n}{np}\PY{o}{.}\PY{n}{exp}\PY{p}{(}\PY{o}{\PYZhy{}}\PY{n}{a}\PY{p}{)}\PY{p}{)}
\PY{n}{b}\PY{o}{=}\PY{n}{df}\PY{o}{.}\PY{n}{loc}\PY{p}{[}\PY{n}{df}\PY{p}{[}\PY{l+s+s1}{\PYZsq{}}\PY{l+s+s1}{DAteTrAnsAction}\PY{l+s+s1}{\PYZsq{}}\PY{p}{]} \PY{o}{\PYZgt{}}\PY{o}{=} \PY{l+s+s1}{\PYZsq{}}\PY{l+s+s1}{2016\PYZhy{}09\PYZhy{}20}\PY{l+s+s1}{\PYZsq{}}\PY{p}{,}\PY{l+s+s1}{\PYZsq{}}\PY{l+s+s1}{MontAnt}\PY{l+s+s1}{\PYZsq{}}\PY{p}{]}\PY{o}{.}\PY{n}{to\PYZus{}numpy}\PY{p}{(}\PY{p}{)}
\PY{n}{CA\PYZus{}RDSW}\PY{o}{=}\PY{n}{np}\PY{o}{.}\PY{n}{sum}\PY{p}{(}\PY{n}{b}\PY{p}{[}\PY{n}{test\PYZus{}target}\PY{o}{==}\PY{l+m+mi}{0}\PY{p}{]}\PY{p}{)}\PY{o}{\PYZhy{}}\PY{n}{np}\PY{o}{.}\PY{n}{sum}\PY{p}{(}\PY{p}{(}\PY{p}{(}\PY{n}{RDSW}\PY{o}{.}\PY{n}{predict}\PY{p}{(}\PY{n}{test\PYZus{}features}\PY{p}{)}\PY{o}{\PYZhy{}}\PY{n}{test\PYZus{}target}\PY{p}{)}\PY{o}{*}\PY{o}{*}\PY{l+m+mi}{2}\PY{p}{)}\PY{o}{*}\PY{n}{a}\PY{p}{)}
\PY{n+nb}{print}\PY{p}{(}\PY{n}{CA\PYZus{}RDSW}\PY{p}{)}
\end{Verbatim}
\end{tcolorbox}

    \begin{Verbatim}[commandchars=\\\{\}]
15353730.078848733
    \end{Verbatim}

    \begin{tcolorbox}[breakable, size=fbox, boxrule=1pt, pad at break*=1mm,colback=cellbackground, colframe=cellborder]
\prompt{In}{incolor}{83}{\boxspacing}
\begin{Verbatim}[commandchars=\\\{\}]
\PY{n+nb}{print}\PY{p}{(}\PY{n}{CAorigin}\PY{o}{\PYZhy{}}\PY{n}{CA\PYZus{}RDSW}\PY{p}{)}
\end{Verbatim}
\end{tcolorbox}

    \begin{Verbatim}[commandchars=\\\{\}]
82747.71115126833
    \end{Verbatim}

    Nous avons 82k\$ en différence, donc on avance 1k de plus par rapport au
modèle simple de Random Forest.

    \hypertarget{abre-de-duxe9cision}{%
\paragraph{10.3.3. Abre de décision}\label{abre-de-duxe9cision}}

    \begin{tcolorbox}[breakable, size=fbox, boxrule=1pt, pad at break*=1mm,colback=cellbackground, colframe=cellborder]
\prompt{In}{incolor}{234}{\boxspacing}
\begin{Verbatim}[commandchars=\\\{\}]
\PY{n}{a}\PY{o}{=}\PY{n}{df}\PY{o}{.}\PY{n}{loc}\PY{p}{[}\PY{n}{df}\PY{p}{[}\PY{l+s+s1}{\PYZsq{}}\PY{l+s+s1}{DAteTrAnsAction}\PY{l+s+s1}{\PYZsq{}}\PY{p}{]} \PY{o}{\PYZgt{}}\PY{o}{=} \PY{l+s+s1}{\PYZsq{}}\PY{l+s+s1}{2016\PYZhy{}09\PYZhy{}20}\PY{l+s+s1}{\PYZsq{}}\PY{p}{,}\PY{l+s+s1}{\PYZsq{}}\PY{l+s+s1}{MontAnt}\PY{l+s+s1}{\PYZsq{}}\PY{p}{]}\PY{o}{.}\PY{n}{to\PYZus{}numpy}\PY{p}{(}\PY{p}{)}
\PY{n}{a}\PY{o}{=}\PY{p}{(}\PY{l+m+mi}{1}\PY{o}{\PYZhy{}}\PY{n}{test\PYZus{}target}\PY{o}{.}\PY{n}{to\PYZus{}numpy}\PY{p}{(}\PY{p}{)}\PY{p}{)}\PY{o}{*}\PY{l+m+mf}{0.2}\PY{o}{*}\PY{n}{a} \PY{o}{+} \PY{n}{test\PYZus{}target}\PY{o}{.}\PY{n}{to\PYZus{}numpy}\PY{p}{(}\PY{p}{)}\PY{o}{*}\PY{n}{a}\PY{o}{*}\PY{p}{(}\PY{l+m+mi}{1}\PY{o}{\PYZhy{}}\PY{n}{np}\PY{o}{.}\PY{n}{exp}\PY{p}{(}\PY{o}{\PYZhy{}}\PY{n}{a}\PY{p}{)}\PY{p}{)}
\PY{n}{b}\PY{o}{=}\PY{n}{df}\PY{o}{.}\PY{n}{loc}\PY{p}{[}\PY{n}{df}\PY{p}{[}\PY{l+s+s1}{\PYZsq{}}\PY{l+s+s1}{DAteTrAnsAction}\PY{l+s+s1}{\PYZsq{}}\PY{p}{]} \PY{o}{\PYZgt{}}\PY{o}{=} \PY{l+s+s1}{\PYZsq{}}\PY{l+s+s1}{2016\PYZhy{}09\PYZhy{}20}\PY{l+s+s1}{\PYZsq{}}\PY{p}{,}\PY{l+s+s1}{\PYZsq{}}\PY{l+s+s1}{MontAnt}\PY{l+s+s1}{\PYZsq{}}\PY{p}{]}\PY{o}{.}\PY{n}{to\PYZus{}numpy}\PY{p}{(}\PY{p}{)}
\PY{n}{CA\PYZus{}DT}\PY{o}{=}\PY{n}{np}\PY{o}{.}\PY{n}{sum}\PY{p}{(}\PY{n}{b}\PY{p}{[}\PY{n}{test\PYZus{}target}\PY{o}{==}\PY{l+m+mi}{0}\PY{p}{]}\PY{p}{)}\PY{o}{\PYZhy{}}\PY{n}{np}\PY{o}{.}\PY{n}{sum}\PY{p}{(}\PY{p}{(}\PY{p}{(}\PY{n}{gridsearchDT}\PY{o}{.}\PY{n}{predict}\PY{p}{(}\PY{n}{test\PYZus{}features}\PY{p}{)}\PY{o}{\PYZhy{}}\PY{n}{test\PYZus{}target}\PY{p}{)}\PY{o}{*}\PY{o}{*}\PY{l+m+mi}{2}\PY{p}{)}\PY{o}{*}\PY{n}{a}\PY{p}{)}
\PY{n+nb}{print}\PY{p}{(}\PY{n}{CA\PYZus{}DT}\PY{p}{)}
\PY{n+nb}{print}\PY{p}{(}\PY{n}{CAorigin}\PY{o}{\PYZhy{}}\PY{n}{CA\PYZus{}DT}\PY{p}{)}
\end{Verbatim}
\end{tcolorbox}

    \begin{Verbatim}[commandchars=\\\{\}]
15338705.322848309
97772.46715169214
    \end{Verbatim}

    \hypertarget{adl}{%
\paragraph{10.3.4. ADL}\label{adl}}

    \begin{tcolorbox}[breakable, size=fbox, boxrule=1pt, pad at break*=1mm,colback=cellbackground, colframe=cellborder]
\prompt{In}{incolor}{235}{\boxspacing}
\begin{Verbatim}[commandchars=\\\{\}]
\PY{n}{a}\PY{o}{=}\PY{n}{df}\PY{o}{.}\PY{n}{loc}\PY{p}{[}\PY{n}{df}\PY{p}{[}\PY{l+s+s1}{\PYZsq{}}\PY{l+s+s1}{DAteTrAnsAction}\PY{l+s+s1}{\PYZsq{}}\PY{p}{]} \PY{o}{\PYZgt{}}\PY{o}{=} \PY{l+s+s1}{\PYZsq{}}\PY{l+s+s1}{2016\PYZhy{}09\PYZhy{}20}\PY{l+s+s1}{\PYZsq{}}\PY{p}{,}\PY{l+s+s1}{\PYZsq{}}\PY{l+s+s1}{MontAnt}\PY{l+s+s1}{\PYZsq{}}\PY{p}{]}\PY{o}{.}\PY{n}{to\PYZus{}numpy}\PY{p}{(}\PY{p}{)}
\PY{n}{a}\PY{o}{=}\PY{p}{(}\PY{l+m+mi}{1}\PY{o}{\PYZhy{}}\PY{n}{test\PYZus{}target}\PY{o}{.}\PY{n}{to\PYZus{}numpy}\PY{p}{(}\PY{p}{)}\PY{p}{)}\PY{o}{*}\PY{l+m+mf}{0.2}\PY{o}{*}\PY{n}{a} \PY{o}{+} \PY{n}{test\PYZus{}target}\PY{o}{.}\PY{n}{to\PYZus{}numpy}\PY{p}{(}\PY{p}{)}\PY{o}{*}\PY{n}{a}\PY{o}{*}\PY{p}{(}\PY{l+m+mi}{1}\PY{o}{\PYZhy{}}\PY{n}{np}\PY{o}{.}\PY{n}{exp}\PY{p}{(}\PY{o}{\PYZhy{}}\PY{n}{a}\PY{p}{)}\PY{p}{)}
\PY{n}{b}\PY{o}{=}\PY{n}{df}\PY{o}{.}\PY{n}{loc}\PY{p}{[}\PY{n}{df}\PY{p}{[}\PY{l+s+s1}{\PYZsq{}}\PY{l+s+s1}{DAteTrAnsAction}\PY{l+s+s1}{\PYZsq{}}\PY{p}{]} \PY{o}{\PYZgt{}}\PY{o}{=} \PY{l+s+s1}{\PYZsq{}}\PY{l+s+s1}{2016\PYZhy{}09\PYZhy{}20}\PY{l+s+s1}{\PYZsq{}}\PY{p}{,}\PY{l+s+s1}{\PYZsq{}}\PY{l+s+s1}{MontAnt}\PY{l+s+s1}{\PYZsq{}}\PY{p}{]}\PY{o}{.}\PY{n}{to\PYZus{}numpy}\PY{p}{(}\PY{p}{)}
\PY{n}{CA\PYZus{}DT}\PY{o}{=}\PY{n}{np}\PY{o}{.}\PY{n}{sum}\PY{p}{(}\PY{n}{b}\PY{p}{[}\PY{n}{test\PYZus{}target}\PY{o}{==}\PY{l+m+mi}{0}\PY{p}{]}\PY{p}{)}\PY{o}{\PYZhy{}}\PY{n}{np}\PY{o}{.}\PY{n}{sum}\PY{p}{(}\PY{p}{(}\PY{p}{(}\PY{n}{gridsearchADL}\PY{o}{.}\PY{n}{predict}\PY{p}{(}\PY{n}{test\PYZus{}features}\PY{p}{)}\PY{o}{\PYZhy{}}\PY{n}{test\PYZus{}target}\PY{p}{)}\PY{o}{*}\PY{o}{*}\PY{l+m+mi}{2}\PY{p}{)}\PY{o}{*}\PY{n}{a}\PY{p}{)}
\PY{n+nb}{print}\PY{p}{(}\PY{n}{CA\PYZus{}DT}\PY{p}{)}
\PY{n+nb}{print}\PY{p}{(}\PY{n}{CAorigin}\PY{o}{\PYZhy{}}\PY{n}{CA\PYZus{}DT}\PY{p}{)}
\end{Verbatim}
\end{tcolorbox}

    \begin{Verbatim}[commandchars=\\\{\}]
15012182.070750276
424295.71924972534
    \end{Verbatim}

    \hypertarget{xgboost}{%
\paragraph{10.3.5. XGBoost}\label{xgboost}}

    \begin{tcolorbox}[breakable, size=fbox, boxrule=1pt, pad at break*=1mm,colback=cellbackground, colframe=cellborder]
\prompt{In}{incolor}{236}{\boxspacing}
\begin{Verbatim}[commandchars=\\\{\}]
\PY{n}{a}\PY{o}{=}\PY{n}{df}\PY{o}{.}\PY{n}{loc}\PY{p}{[}\PY{n}{df}\PY{p}{[}\PY{l+s+s1}{\PYZsq{}}\PY{l+s+s1}{DAteTrAnsAction}\PY{l+s+s1}{\PYZsq{}}\PY{p}{]} \PY{o}{\PYZgt{}}\PY{o}{=} \PY{l+s+s1}{\PYZsq{}}\PY{l+s+s1}{2016\PYZhy{}09\PYZhy{}20}\PY{l+s+s1}{\PYZsq{}}\PY{p}{,}\PY{l+s+s1}{\PYZsq{}}\PY{l+s+s1}{MontAnt}\PY{l+s+s1}{\PYZsq{}}\PY{p}{]}\PY{o}{.}\PY{n}{to\PYZus{}numpy}\PY{p}{(}\PY{p}{)}
\PY{n}{a}\PY{o}{=}\PY{p}{(}\PY{l+m+mi}{1}\PY{o}{\PYZhy{}}\PY{n}{test\PYZus{}target}\PY{o}{.}\PY{n}{to\PYZus{}numpy}\PY{p}{(}\PY{p}{)}\PY{p}{)}\PY{o}{*}\PY{l+m+mf}{0.2}\PY{o}{*}\PY{n}{a} \PY{o}{+} \PY{n}{test\PYZus{}target}\PY{o}{.}\PY{n}{to\PYZus{}numpy}\PY{p}{(}\PY{p}{)}\PY{o}{*}\PY{n}{a}\PY{o}{*}\PY{p}{(}\PY{l+m+mi}{1}\PY{o}{\PYZhy{}}\PY{n}{np}\PY{o}{.}\PY{n}{exp}\PY{p}{(}\PY{o}{\PYZhy{}}\PY{n}{a}\PY{p}{)}\PY{p}{)}
\PY{n}{b}\PY{o}{=}\PY{n}{df}\PY{o}{.}\PY{n}{loc}\PY{p}{[}\PY{n}{df}\PY{p}{[}\PY{l+s+s1}{\PYZsq{}}\PY{l+s+s1}{DAteTrAnsAction}\PY{l+s+s1}{\PYZsq{}}\PY{p}{]} \PY{o}{\PYZgt{}}\PY{o}{=} \PY{l+s+s1}{\PYZsq{}}\PY{l+s+s1}{2016\PYZhy{}09\PYZhy{}20}\PY{l+s+s1}{\PYZsq{}}\PY{p}{,}\PY{l+s+s1}{\PYZsq{}}\PY{l+s+s1}{MontAnt}\PY{l+s+s1}{\PYZsq{}}\PY{p}{]}\PY{o}{.}\PY{n}{to\PYZus{}numpy}\PY{p}{(}\PY{p}{)}
\PY{n}{CA\PYZus{}XG}\PY{o}{=}\PY{n}{np}\PY{o}{.}\PY{n}{sum}\PY{p}{(}\PY{n}{b}\PY{p}{[}\PY{n}{test\PYZus{}target}\PY{o}{==}\PY{l+m+mi}{0}\PY{p}{]}\PY{p}{)}\PY{o}{\PYZhy{}}\PY{n}{np}\PY{o}{.}\PY{n}{sum}\PY{p}{(}\PY{p}{(}\PY{p}{(}\PY{n}{xg\PYZus{}clas}\PY{o}{.}\PY{n}{predict}\PY{p}{(}\PY{n}{test\PYZus{}features}\PY{p}{)}\PY{o}{\PYZhy{}}\PY{n}{test\PYZus{}target}\PY{p}{)}\PY{o}{*}\PY{o}{*}\PY{l+m+mi}{2}\PY{p}{)}\PY{o}{*}\PY{n}{a}\PY{p}{)}
\PY{n+nb}{print}\PY{p}{(}\PY{n}{CA\PYZus{}XG}\PY{p}{)}
\PY{n+nb}{print}\PY{p}{(}\PY{n}{CAorigin}\PY{o}{\PYZhy{}}\PY{n}{CA\PYZus{}XG}\PY{p}{)}
\end{Verbatim}
\end{tcolorbox}

    \begin{Verbatim}[commandchars=\\\{\}]
15281358.332843106
155119.45715689473
    \end{Verbatim}

    \hypertarget{adxxg}{%
\paragraph{10.3.6. ADxXG}\label{adxxg}}

    \begin{tcolorbox}[breakable, size=fbox, boxrule=1pt, pad at break*=1mm,colback=cellbackground, colframe=cellborder]
\prompt{In}{incolor}{237}{\boxspacing}
\begin{Verbatim}[commandchars=\\\{\}]
\PY{n}{a}\PY{o}{=}\PY{n}{df}\PY{o}{.}\PY{n}{loc}\PY{p}{[}\PY{n}{df}\PY{p}{[}\PY{l+s+s1}{\PYZsq{}}\PY{l+s+s1}{DAteTrAnsAction}\PY{l+s+s1}{\PYZsq{}}\PY{p}{]} \PY{o}{\PYZgt{}}\PY{o}{=} \PY{l+s+s1}{\PYZsq{}}\PY{l+s+s1}{2016\PYZhy{}09\PYZhy{}20}\PY{l+s+s1}{\PYZsq{}}\PY{p}{,}\PY{l+s+s1}{\PYZsq{}}\PY{l+s+s1}{MontAnt}\PY{l+s+s1}{\PYZsq{}}\PY{p}{]}\PY{o}{.}\PY{n}{to\PYZus{}numpy}\PY{p}{(}\PY{p}{)}
\PY{n}{a}\PY{o}{=}\PY{p}{(}\PY{l+m+mi}{1}\PY{o}{\PYZhy{}}\PY{n}{test\PYZus{}target}\PY{o}{.}\PY{n}{to\PYZus{}numpy}\PY{p}{(}\PY{p}{)}\PY{p}{)}\PY{o}{*}\PY{l+m+mf}{0.2}\PY{o}{*}\PY{n}{a} \PY{o}{+} \PY{n}{test\PYZus{}target}\PY{o}{.}\PY{n}{to\PYZus{}numpy}\PY{p}{(}\PY{p}{)}\PY{o}{*}\PY{n}{a}\PY{o}{*}\PY{p}{(}\PY{l+m+mi}{1}\PY{o}{\PYZhy{}}\PY{n}{np}\PY{o}{.}\PY{n}{exp}\PY{p}{(}\PY{o}{\PYZhy{}}\PY{n}{a}\PY{p}{)}\PY{p}{)}
\PY{n}{b}\PY{o}{=}\PY{n}{df}\PY{o}{.}\PY{n}{loc}\PY{p}{[}\PY{n}{df}\PY{p}{[}\PY{l+s+s1}{\PYZsq{}}\PY{l+s+s1}{DAteTrAnsAction}\PY{l+s+s1}{\PYZsq{}}\PY{p}{]} \PY{o}{\PYZgt{}}\PY{o}{=} \PY{l+s+s1}{\PYZsq{}}\PY{l+s+s1}{2016\PYZhy{}09\PYZhy{}20}\PY{l+s+s1}{\PYZsq{}}\PY{p}{,}\PY{l+s+s1}{\PYZsq{}}\PY{l+s+s1}{MontAnt}\PY{l+s+s1}{\PYZsq{}}\PY{p}{]}\PY{o}{.}\PY{n}{to\PYZus{}numpy}\PY{p}{(}\PY{p}{)}
\PY{n}{CA\PYZus{}ADXG}\PY{o}{=}\PY{n}{np}\PY{o}{.}\PY{n}{sum}\PY{p}{(}\PY{n}{b}\PY{p}{[}\PY{n}{test\PYZus{}target}\PY{o}{==}\PY{l+m+mi}{0}\PY{p}{]}\PY{p}{)}\PY{o}{\PYZhy{}}\PY{n}{np}\PY{o}{.}\PY{n}{sum}\PY{p}{(}\PY{p}{(}\PY{p}{(}\PY{n}{VotingPredictorAR}\PY{o}{.}\PY{n}{predict}\PY{p}{(}\PY{n}{test\PYZus{}features}\PY{p}{)}\PY{o}{\PYZhy{}}\PY{n}{test\PYZus{}target}\PY{p}{)}\PY{o}{*}\PY{o}{*}\PY{l+m+mi}{2}\PY{p}{)}\PY{o}{*}\PY{n}{a}\PY{p}{)}
\PY{n+nb}{print}\PY{p}{(}\PY{n}{CA\PYZus{}ADXG}\PY{p}{)}
\PY{n+nb}{print}\PY{p}{(}\PY{n}{CAorigin}\PY{o}{\PYZhy{}}\PY{n}{CA\PYZus{}ADXG}\PY{p}{)}
\end{Verbatim}
\end{tcolorbox}

    \begin{Verbatim}[commandchars=\\\{\}]
15320522.120849116
115955.6691508852
    \end{Verbatim}

    \hypertarget{conclusion}{%
\subsection{11. Conclusion}\label{conclusion}}

    \begin{tcolorbox}[breakable, size=fbox, boxrule=1pt, pad at break*=1mm,colback=cellbackground, colframe=cellborder]
\prompt{In}{incolor}{ }{\boxspacing}
\begin{Verbatim}[commandchars=\\\{\}]

\end{Verbatim}
\end{tcolorbox}

    \hypertarget{exporter-and-save}{%
\section{Exporter and save}\label{exporter-and-save}}

    \begin{tcolorbox}[breakable, size=fbox, boxrule=1pt, pad at break*=1mm,colback=cellbackground, colframe=cellborder]
\prompt{In}{incolor}{121}{\boxspacing}
\begin{Verbatim}[commandchars=\\\{\}]
  \PY{c+c1}{\PYZsh{}\PYZpc{}notebook \PYZsq{}/Users/hoangkhanhle/Desktop/School/Master 2/Big Data/ProjetFraud/BIG DATA.ipynb\PYZsq{}}
\end{Verbatim}
\end{tcolorbox}


    % Add a bibliography block to the postdoc
    
    
    
\end{document}
